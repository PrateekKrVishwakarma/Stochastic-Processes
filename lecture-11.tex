<<<<<<< HEAD
\documentclass[a4paper,10pt]{article}
\usepackage{%
amsmath,%
amsfonts,%
amssymb,%
amsthm,%
hyperref,%
url,%
latexsym,%
epsfig,%
graphicx,%
psfrag,%
subfigure,%
color,%
tikz,%
pgf,%
pgfplots,%
pgfplotstable,%
pgfpages%
}
\usepgflibrary{shapes}
\usetikzlibrary{%
arrows,%
backgrounds,%
chains,%
decorations.pathmorphing,% /pgf/decoration/random steps | erste Graphik
decorations.text,%
matrix,%
positioning,% wg. " of "
fit,%
patterns,%
petri,%
plotmarks,%
scopes,%
shadows,%
shapes.misc,% wg. rounded rectangle
shapes.arrows,%
shapes.callouts,%
shapes%
}
\theoremstyle{plain}
\newtheorem{thm}{Theorem}[section]
\newtheorem{lem}[thm]{Lemma}
\newtheorem{prop}[thm]{Proposition}
\newtheorem{cor}[thm]{Corollary}
\theoremstyle{definition}
\newtheorem{defn}[thm]{Definition}
\newtheorem{conj}[thm]{Conjecture}
\newtheorem{exmp}[thm]{Example}
\theoremstyle{remark}
\newtheorem{rem}[thm]{Remark}
\newtheorem{note}[thm]{Note}

\usepackage{%
	amsfonts,%
	amsmath,%	
	etex,%
	amssymb,%
	amsthm,%
	babel,%
	bbm,%
	%biblatex,%
	caption,%
	centernot,%
	color,%
	enumerate,%
	epsfig,%
	epstopdf,%
	geometry,%
	graphicx,%
	hyperref,%
	latexsym,%
	mathtools,%
	multicol,%
	pgf,%
	pgfplots,%
	pgfplotstable,%
	pgfpages,%
	proof,%
	psfrag,%
	subfigure,%	
	tikz,%
	ulem,%
	url%
}	

\usepackage[mathscr]{eucal}
\usepgflibrary{shapes}
\usetikzlibrary{%
  arrows,%
	backgrounds,%
	chains,%
	decorations.pathmorphing,% /pgf/decoration/random steps | erste Graphik
	decorations.text,%
	matrix,%
  	positioning,% wg. " of "
  	fit,%
	patterns,%
  	petri,%
	plotmarks,%
  	scopes,%
	shadows,%
  	shapes.misc,% wg. rounded rectangle
  	shapes.arrows,%
	shapes.callouts,%
  	shapes%
}

\theoremstyle{plain}
\newtheorem{thm}{Theorem}[section]
\newtheorem{lem}[thm]{Lemma}
\newtheorem{prop}[thm]{Proposition}
\newtheorem{cor}[thm]{Corollary}

\theoremstyle{definition}
\newtheorem{defn}[thm]{Definition}
\newtheorem{conj}[thm]{Conjecture}
\newtheorem{exmp}[thm]{Example}
\newtheorem{assum}[thm]{Assumptions}
\newtheorem{axiom}[thm]{Axiom}

\theoremstyle{remark}
\newtheorem{rem}{Remark}
\newtheorem{note}{Note}

\newcommand{\norm}[1]{\left\lVert#1\right\rVert}
\newcommand{\indep}{\!\perp\!\!\!\perp}
\DeclarePairedDelimiter\abs{\lvert}{\rvert}%
%\DeclarePairedDelimiter\norm{\lVert}{\rVert}%
\newcommand{\tr}{\operatorname{tr}}
\newcommand{\R}{\mathbb{R}}
\newcommand{\Q}{\mathbb{Q}}
\newcommand{\N}{\mathbb{N}}
\newcommand{\E}{\mathbb{E}}
\newcommand{\Z}{\mathbb{Z}}
\newcommand{\B}{\mathscr{B}}
\newcommand{\C}{\mathcal{C}}
\newcommand{\T}{\mathscr{T}}
\newcommand{\F}{\mathcal{F}}
\newcommand{\G}{\mathcal{G}}
%\newcommand{\ba}{\begin{align*}}
%\newcommand{\ea}{\end{align*}}

\makeatletter
\def\th@plain{%
  \thm@notefont{}% same as heading font
  \itshape % body font
}
\def\th@definition{%
  \thm@notefont{}% same as heading font
  \normalfont % body font
}
\makeatother
\date{}
\title{Lecture 11 : Discrete Time Markov Chains}
\author{Sireesha Madabhushi}
\begin{document}
\maketitle
\section{Discrete Time Markov Chains Contd}
\subsection{Theorem}
\textbf{Statement:}\\
An irreducible, aperiodic MC is of one of the following types:
\begin{itemize}
	\item All the states are either transient or null recurrent.\\
	$P_{ij}^n \longrightarrow 0$ as $n \longrightarrow \infty$ and there exists no stationary distribution
	\item All the states are positive recurrent. There exists a unique stationary distribution ${\pi_j, j = 0,1,2,3...}$.\\
    $\pi_j = \underset{n\rightarrow \infty}{\text{lim}} P^n_{ij} > 0$
\end{itemize}
\begin{proof}
We will initially prove the second case.
From renewal reward theorem, $\lim_{n \to \infty} \sum_{j=0}^{\infty} P_{ij}^{(n)} = 1/\mu_{jj}$ exists, which we take as $\pi_j$.\\
\begin{flalign*}
&\sum_{j=0}^{M} P_{ij}^n \leq \sum_{j=0}^{\infty} P_{ij}^n = 1, \forall M\\
&n \longrightarrow \infty \Rightarrow \sum_{j=0}^{M}\pi_j  \leq 1, \forall M\\	
&\Rightarrow \sum_{j=0}^{\infty}\pi_j  \leq 1\\
&P_{ij}^{n+1} = \sum_{k=0}^{\infty}P_{ik}^nP_{kj} \geq \sum_{k=0}^{M}P_{ik}^nP_{kj},\forall M\\
\end{flalign*}
Applying $n \to \infty$ on $P_{ij}^{n+1} \geq \sum_{k=0}^{M}P_{ik}^nP_{kj}$, we get\\
\begin{flalign*}
&n \longrightarrow \infty \Rightarrow \pi_j \geq \sum_{k=0}^{M} \pi_kP_{kj},\forall M\\
&\Rightarrow \pi_j \geq \sum_{k=0}^{\infty} \pi_kP_{kj},j \geq 0\\
\end{flalign*}	
Assuming that the inequality is strict for some $j$ and adding the inequalities, we get
\begin{flalign*}
&\sum_{j=0}^{\infty} \pi_j > \sum_{j=0}^{\infty} \sum_{k=0}^{\infty} \pi_kP_{kj} = \sum_{k=0}^{\infty} \pi_k \sum_{j=0}^{\infty} P_{kj} = \sum_{k=0}^{\infty} \pi_k\\
&(summation \: can\: be\: interchanged \:since\: \pi_k\: and\: P_{kj}\: are\: non-negative.)\\
\end{flalign*}	
This is a contradiction. Therefore,\\
\begin{flalign*}
&\pi_j = \sum_{k=0}^{\infty} \pi_k P_{kj}, j = 0,1,2....\\
\end{flalign*}
Putting $P_j = \dfrac{\pi_j}{\sum_{0}^{\infty}\pi_k}$ (normalization), we see that $\{P_j,j=0,1,2...\}$ is a stationary distribution and so atleast one stationary distribution exists. Let $\{P_j,j=0,1,2...\}$ be any stationary distribution. Then if $\{P_j,j=0,1,2...\}$ is the probability distribution of $X_0$, then\\
\begin{flalign*}
&P_j = P\{X_n = j\}\\
&P_j = \sum_{i=0}^{\infty} P\{X_n = j|X_0 = i\} P\{X_0 = i\}\\
&P_j = \sum_{i=0}^{\infty} P_{ij}^nP_i\\
&\Rightarrow P_j \geq \sum_{i=0}^{M} P_{ij}^nP_i, \forall M\\
&Applying \: n \longrightarrow \infty and then M \longrightarrow \infty \Rightarrow P_j \geq \sum_{i=0}^{\infty} \pi_j P_i = \pi_j\\
&P_j \geq \pi_j\\
\end{flalign*}
To show $P_j \leq \pi_j$, we use the fact that $P_{ij}^n \leq 1$.\\
\begin{flalign*}
&P_j = \sum_{i=0}^{\infty} P_{ij}^nP_i\\
&P_j = \sum_{i=0}^{M}P_{ij}^nP_i + \sum_{i=M+1}^{\infty}P_{ij}^nP_i\\
&P_j \leq \sum_{i=0}^{M} P_{ij}^{n}P_i + \sum_{i=M+1}^{\infty} P_i, \forall M\\
&n \longrightarrow \infty \Rightarrow P_j \leq \sum_{i=0}^{M}\pi_jP_i + \sum_{i=M+1}^{\infty} P_i, \forall M\\
&\because \sum_{0}^{\infty}P_i = 1, M \longrightarrow \infty \Rightarrow P_j \leq \sum_{i=0}^{\infty} \pi_j P_i = \pi_j\\
&P_j \leq \pi_j\\
\end{flalign*}
If the states are transient or null recurrent and $\{P_j,j=0,1,2...\}$ is a stationary distribution, then $P_j = \sum_{i=0}^{\infty} P\{X_n = j|X_0 = i\} P\{X_0 = i\}$ holds.\\
Then, $n \to \infty$ in $P_j= \sum_{i=0}^{n} \pi_j P_{ij}^n$, gives $P_j=0$ for every $j$. This contradicts the assumption that $P_j$ is a probability distribution.\\
Thus for the first case, no stationary distribution exists and the proof is complete.
\end{proof}
\textbf{Corollary 1:} An irreducible, aperiodic DTMC defined on a finite state space will be positive recurrent.
\begin{proof}
	Suppose that the DTMC is not positive recurrent, then \\
	\begin{flalign*}
		&\underset{n\rightarrow \infty}{\text{lim}} p_{ij}^{(n)} = 0\\
    \end{flalign*}
		Interchanging limit and finite summation gives
    \begin{flalign*}
		&\underset{n\rightarrow \infty}{\text{lim}} (\sum_{j \in S}p_{ij}) = \underset{n\rightarrow \infty}{\text{lim}} (1) = 1\\
	\end{flalign*}
	This is a contradiction.
	Hence the DTMC mentioned above is positive recurrent.
\end{proof}

\textbf{Corollary 2:} For an irreducible and aperiodic DTMC, $E_i[T_i] = 1/\pi_i$, where $E_i[T_i]$ denotes the expected number of time steps to return to the state $i$ and $\pi_i$ is its stationary distribution.\\

\subsection{Ergodic theorem for DTMCs}

\textbf{Claim:} Let $\{X_n\}$ be an irreducible, aperiodic and positive recurrent DTMC on $N_0$ with stationary distribution $\Pi$.\\
Let $f : N \longrightarrow R$, such that $\sum_{i \in N_0} \arrowvert f(i) \arrowvert \Pi_i < \infty$ (f is integrable over $N_0$ w.r.t $\Pi$).\\
Then for any initial distribution of $X_0$,\\
$\underset{n\rightarrow \infty}{\text{lim}} \dfrac{1}{n} \sum_{i=1}^{n} f(X_i) = \sum_{i \in N_0} \Pi(i)f(i)$ almost surely.\\

\begin{proof}

Fix $0 \in N_0$	.\\
Let $\tau_1, \tau_2,....$ be successive instants at which '0' is visited ($\tau_0 = 0$).\\
Let $\forall p \geq 0, R_{p+1} = \sum_{n = \tau_p + 1}^{\tau_{(p + 1)}} f(X_n) = $net reward earned at the end of cycle $(p+1)$.\\
Each cycle forms a renewal.By the strong Markov property, cycles are independent. 
At stopping time, state is $0$.\\
Average reward earned during the first $k$ cycles is given as follows:
\begin{flalign*}	
&\dfrac{1}{n} \sum_{k=1}^{n} R_k = \dfrac{1}{n} \sum_{k=1}^{n} \sum_{\tau_{k-1} + 1}^{\tau_k} f(X_i)\\
&\dfrac{1}{n} \sum_{k=1}^{n} R_k = \dfrac{1}{n} \sum_{t=0}^{\tau_n} f(X_t)\\
&\stackrel {n \longrightarrow \infty}{\longrightarrow} \dfrac{E_0[cycle \:  renewal]}{E_0[cycle \: length]} \: (from \: Renewal \: Reward \: Theorem)\\
&E_0[cycle \: length] = \dfrac{1}{\pi_0}\: (from \: Corollary \: 2)\\
&E_0[cycle \: reward] = E_0[\sum_{n=0}^{\tau_1}f(X_n)]\\
&E_0[cycle \: reward] = E_0[\sum_{n=0}^{\tau_1} \sum_{i \in N_0} f(i) \textbf{1}[X_n = i]]\\
&E_0[cycle \: reward] = \sum_{i \in N_0} f(i) E_0[\sum_{n=0}^{\tau_i} \textbf{1}[X_n = i]]\: (assuming\: f(i) > 0)\\
\end{flalign*}
To calculate the reward given above, we need to find the expected number of visits to state $i$ between successive visits to state 0
Let us consider the average number of visits to state $i$ i.e.,\\
$\dfrac{\underset{n\rightarrow \infty}{\text{lim}} \sum_{k=1}^{n} \textbf{1}[X_t = i]}{n}$. Let us call it $\circledR$.\\
We can calculate the above quantity in two ways.\\
\begin{enumerate}
	\item DTMC undergoes a renewal each time it hits state $i$.\\
	So, from elementary renewal theorem, \\
	$\circledR = 1/(inter\:arrival\: times\: of\: hitting\: i) = 1/(1/\pi_i) = \pi_i$
	\item Without loss of generality, renewal occurs whenever DTMC hits state $0$.\\
	By renewal reward theorem, $\circledR = \dfrac{E_0[\sum_{n=0}^{\tau_i} \textbf{1}[X_n = i]]}{1/\pi_0}$
\end{enumerate}
From (1) and (2), $E_0[\sum_{n=0}^{\tau_i} \textbf{1}[X_n = i]] = \dfrac{\pi_i}{\pi_0}$\\
$\dfrac{1}{\tau_n} \sum_{t=0}^{\tau_n} f(X_t) \stackrel {n \longrightarrow \infty}{\longrightarrow} \dfrac{\sum_{i \in N_0} f(i) \pi_i/\pi_0}{1/\pi_0}$

\end{proof}

\end{document}
=======
% !TEX spellcheck = en_US
% !TEX spellcheck = LaTeX
\documentclass[a4paper,10pt,english]{article}
\usepackage{%
	amsfonts,%
	amsmath,%	
	etex,%
	amssymb,%
	amsthm,%
	babel,%
	bbm,%
	%biblatex,%
	caption,%
	centernot,%
	color,%
	enumerate,%
	epsfig,%
	epstopdf,%
	geometry,%
	graphicx,%
	hyperref,%
	latexsym,%
	mathtools,%
	multicol,%
	pgf,%
	pgfplots,%
	pgfplotstable,%
	pgfpages,%
	proof,%
	psfrag,%
	subfigure,%	
	tikz,%
	ulem,%
	url%
}	

\usepackage[mathscr]{eucal}
\usepgflibrary{shapes}
\usetikzlibrary{%
  arrows,%
	backgrounds,%
	chains,%
	decorations.pathmorphing,% /pgf/decoration/random steps | erste Graphik
	decorations.text,%
	matrix,%
  	positioning,% wg. " of "
  	fit,%
	patterns,%
  	petri,%
	plotmarks,%
  	scopes,%
	shadows,%
  	shapes.misc,% wg. rounded rectangle
  	shapes.arrows,%
	shapes.callouts,%
  	shapes%
}

\theoremstyle{plain}
\newtheorem{thm}{Theorem}[section]
\newtheorem{lem}[thm]{Lemma}
\newtheorem{prop}[thm]{Proposition}
\newtheorem{cor}[thm]{Corollary}

\theoremstyle{definition}
\newtheorem{defn}[thm]{Definition}
\newtheorem{conj}[thm]{Conjecture}
\newtheorem{exmp}[thm]{Example}
\newtheorem{assum}[thm]{Assumptions}
\newtheorem{axiom}[thm]{Axiom}

\theoremstyle{remark}
\newtheorem{rem}{Remark}
\newtheorem{note}{Note}

\newcommand{\norm}[1]{\left\lVert#1\right\rVert}
\newcommand{\indep}{\!\perp\!\!\!\perp}
\DeclarePairedDelimiter\abs{\lvert}{\rvert}%
%\DeclarePairedDelimiter\norm{\lVert}{\rVert}%
\newcommand{\tr}{\operatorname{tr}}
\newcommand{\R}{\mathbb{R}}
\newcommand{\Q}{\mathbb{Q}}
\newcommand{\N}{\mathbb{N}}
\newcommand{\E}{\mathbb{E}}
\newcommand{\Z}{\mathbb{Z}}
\newcommand{\B}{\mathscr{B}}
\newcommand{\C}{\mathcal{C}}
\newcommand{\T}{\mathscr{T}}
\newcommand{\F}{\mathcal{F}}
\newcommand{\G}{\mathcal{G}}
%\newcommand{\ba}{\begin{align*}}
%\newcommand{\ea}{\end{align*}}

\makeatletter
\def\th@plain{%
  \thm@notefont{}% same as heading font
  \itshape % body font
}
\def\th@definition{%
  \thm@notefont{}% same as heading font
  \normalfont % body font
}
\makeatother
\date{}
\title{Lecture 11 : Time Reversibility of Discrete Time Markov Chains}
\author{}
\begin{document}
\maketitle
\section{Discrete Time Markov Chains Contd.}

\subsection{Time Reversibility of Discrete Time Markov Chains}

Consider a discrete time Markov chain with transition probability matrix $P$ and stationary probability vector $\pi$.\\
\textbf{Claim:} The reversed process is a Markov chain.
\begin{proof}
\begin{flalign*}
&P(X_{m-1}=i|X_m=j,X_{m+1}=i_{m+1} \hdots )=\frac{P(X_{m-1}=i,X_m=j, \hdots )}{P(X_m=j, X_{m+1}=i_{m+1}\hdots )}\\
&=\frac{P(X_{m-1}=i,X_m=j)P(X_{m+1}=i_{m+1}\hdots |X_{m-1}=i,X_m=j  )}{P(X_m=j)P(X_{m+1}=i_{m+1} \hdots |X_m=j)}\\
&\stackrel{(a)}{=}\frac{P(X_{m-1}=i,X_m=j)P(X_{m+1}=i_{m+1}\hdots |X_m=j  )}{P(X_m=j)P(X_{m+1}=i_{m+1} \hdots |X_m=j)}\\
&=P(X_{m-1}=i|X_m=j).\\
\end{flalign*}
where $(a)$ follows from the Markov property.
\end{proof}
The sequence $\{X_n,X_{n-1} \hdots \}$ is called reverse process. Let $P^*$ denote the transition probability matrix. 
\begin{flalign*}
P_{ij}^*&=P(X_{n-1}=j|X_n=i)=\frac{P(X_{n-1}=j,X_n=i)}{P(X_n=i)}\\
&=\frac{P(X_{n-1}=j)P(X_n=i|X_{n-1}=j)}{P(X_{n}=i)}\\
&=\frac{P(X_{n-1}=j)}{P(X_{n}=i)}P_{ji}
\end{flalign*}
suppose we are considering a stationary Markov chain, $P(X_n=l)=P(X_{n-1}=l)=\pi(l),~\forall l$, $\pi(i){P^*}_{ij}=\pi(j)(P)_{ji}$. If $P^*=P^T$ then the Markov chain is called time reversible. Thus the condition for time reversibility is given by $\pi P_{ij}=\pi_jP_{ji}$. Any non-negative vector $X$ satisfying $X_iP_{ij}=X_jP_{ji},~\forall i,j$ and $\sum_{j \in \mathcal{N}_0}X_j=1$ is stationary distribution of time-reversible Markov chain. This is true because,
\begin{flalign*}
\sum_{i} X_i P_{ij}=\sum_i X_j P_{ji}=X_j \sum X_i = 1.
\end{flalign*}
Since stationary probabilities are the unique solution of the above, the claim follows.\\

\textbf{Example 4.7(A) An Ergodic Random Walk}: Any ergodic, positive recurrent random walk is time reversible. The transition probability matrix is $P_{i,i+1}+P_{i-1,i}=1$. For every two transitions from $i+1$ to $i$, there must be one transition from $i$ to $i+1$. The rate of transitions from $i+1$ to $i$ must hence be same as the number of transitions from $i$ to $i+1$. So the process is time reversible.

\textbf{Example 4.7(B) The Metropolis Algorithm: } Let $a_j,~ j=1, \hdots m$ be positive numbers and let $A=\sum_{i=1}^{m}a_i$. Suppose that $m$ is large and $A$ is difficult to compute. One way to compute it by simulating a sequence of random variables is by generating a Markov chain whose limiting probabilities are $p_j$s. The Metropolis algorithm provides a convenient approach. \\
Let $Q$ be an irreducible transition probability matrix on the integers $1, \hdots n$ such that $q_{ij}=q_{ji}$ and for all $i$ and $j$. Generate a Markov chain $\{X_n\}$ such that the transition probabilities are given by 
\begin{flalign*}
P_{ij} = \left\{
     \begin{array}{lr}
       q_{ij}\min(1,a_j/a_i) & : j \neq i\\
       q_{ii}+\sum_{j \neq i}q_{ij}\{1-\min(1,a_j/a_i)\} & : j = i.
     \end{array}
   \right.
\end{flalign*} 
It can be directly verified that the chain is irreducible and hence verify that $p_j$s are the limiting probabilities.

\textbf{Edge weighted graphs:} Consider a graph having a positive number $w_{ij}$ associated with each edge $(i,j)$, and suppose that a particle moves from vertex to vertex in the following manner: If the particle is presently at vertex  $i$ then it will next move to vertex $j$ with probability
\begin{equation*}
P_{ij}=\frac{w_{ij}}{\sum_{j}w_{ij}}
\end{equation*}
where $w_{ij}$ is 0 if $(i,j)$ is not an edge of the graph. The Markov chain describing the sequence of vertices visited by the particle is called a random walk on an edge weighted graph. 
\begin{prop}
Consider a random walk on an edge weighted graph witha finite number of vertices. If this Markov chain is irreducible then it is, in steady state, time reversible with stationary probabilities given by 
\begin{equation*}
\pi_i = \frac{\sum_{i}w_{ij}}{\sum_{j}\sum_{i}w_{ij}}.
\end{equation*}
\end{prop}
\begin{proof}
The time reversibility equation
\begin{equation*}
\pi_iP_{ij}=\pi_{j}P_{ji}
\end{equation*}
reduces to 
\begin{equation*}
 \frac{\pi_i w_{ij}}{\sum_{k}w_{ik}}=\frac{\pi_j w_{ji}}{\sum_{k}w_{jk}}
\end{equation*}
But noting that $w_{ij}=w_{ji}$ and $\sum_{i}\pi_i = 1$, we get the desired result.
 
\end{proof}
\subsubsection*{Necessary condition for time reversibility}
If we try to prove the equations necessary for time reversibility, $X_iP_{ij}=X_jP_{jk}$ for all $i,j$, for any arbitrary Markov chain, one may not end up getting any solution. This is so because, if $P_{ij}P_{jk}>0$, then $\frac{X_i}{X_k}=\frac{P_{ji}P_{kj}}{P_{jk}P_{ij}} \neq \frac{P_{kj}}{P_{jk}}$.\\
Thus we see that a necessary condition for time reversibility is $P_{ij}P_{jk}P_{ki}=P_{ik}P_{kj}P_{ji},~ \forall i,j,k$. In fact we can show the following.
\begin{thm}
A stationary Markov chain is time reversible if and only if starting in state $i$
, any path back to state $i$ has the same probability as the reversed path, for all $i$. That is, if
\begin{flalign*}
P_{i i_1}P_{i_1 i_2}\hdots P_{i_k i}=P_{i,i_k}P_{i_k i_{k-1}} \hdots P_{i_1,i}.
\end{flalign*} 
\end{thm}
\begin{proof}
The proof of necessity is as indicated above. To see the sufficiency part, fix states $i,j$
\begin{flalign*}
&\sum_{i_1,i_2,\hdots i_{k}}P_{ii_1}\hdots P_{i_k,j}P_{j,i}=\sum_{i_1,i_2,\hdots i_{k}}P_{i,j}P_{j,i_k}\hdots P_{i_1 i}\\
&(P^k)_{ij}P_{ji}=P_{ij}(P^k)_{ji}\\
&\frac{\sum_{k=1}^{n}(P^k)_{ij}P_{ji}}{n}= \frac{\sum_{k=1}^{n}P_{ij}(P^k)_{ji}}{n}
\end{flalign*}
As limit $n \rightarrow \infty$, we get the desire result.
\end{proof}

\begin{thm}
Consider irreducible Markov chain with transition matrix $P$. If one can find non-negative vector $\pi$ and other transition matrix $P^*$ such that $\sum_j \pi_j =1$ and $\pi_iP_{ij}=\pi_jP^*_{ji}$ then $\pi$ is the stationary probability vector and $P^*$ is the transition matrix for the reversed chain.
\end{thm}
\begin{proof}
Summing $\pi_iP_{ij}=\pi_jP_{ji}^*$ over $i$ gives, $\sum_{i}\pi_iP_{ij}=\pi_j$. Hence $\pi_i$s are the stationary probabilities of the forward and reverse process. Since $P_{ji}^*=\frac{\pi_iP_{ij}}{\pi_j}$, $P_{ij}^*$ are the transition probabilities of the reverse chain.
\end{proof} 

\subsection{Example 4.7(E): Example 4.3(C) revisited}
Example 4.3(C) was with regard to the age of a renewal process. $X_n$ the forward process there was such that it increases in steps of 1 until it hits a value chosen by the inter arrival distribution. Hence the reverse process should be such that it decreases in steps of 1 until it hits 1 and then jumps to a state as chosen by the inter arrival distribution. Thus letting $P_i$ as the probability of inter arrival, it seems likely that  $P_{1i}*=P_i, ~ P_{i,i-1}=1,~ i > 1$. We have that $P_{i,1}=\frac{P_i}{\sum_{j \geq 1}P_j}=1-P_{i,i+1}, ~ i \geq 1$. For the reversed chain to be given as above, we would need 
\begin{flalign*}
&\pi_i P_{ij}=\pi_j P_{ji}^*\\
&\pi_i \frac{P_i}{\sum_j P_j}=\pi_1 P_i\\
&\pi_i=\pi_1 P(X \geq i)\\
&1=\sum_i \pi_i=\pi_1 E[X]; \pi_i=\frac{P(X \geq i)}{E[X]}, 
\end{flalign*}
where $X$ is the inter arrival time. We need to verify that $\pi_i P_{i,i+1}=\pi_{i+1}P^*_{i+1,i}$ or equivalently $P(X \geq i)(1-\frac{P_i}{P(X \geq i)})=P(X \geq i)$ to complete the proof that the reversed process is the excess process and the limiting distributions are as given above. But that is immediate.
\subsection{Semi Markov Processes}
Consider a stochastic process with states $0,1,2 \hdots$ such that whenever it enters state $i$,
\begin{enumerate}
\item {The next state it enters is state $j$ with probability $P_{ij}$.}\\
\item {Given the next state is going to be $j$, the time until the next transition from state $i$ to state $j$ has distribution $F_{ij}$. If we denote the state at time $t$ to be $Z(t)$, $\{Z(t), t \geq 0\}$ is called a Semi Markov process.}
\end{enumerate}

Markov chain is a semi Markov process  with 

\begin{flalign*} 
  F_{ij}(t) = \left\{
     \begin{array}{lr}
       0 & : t \leq 1 \\
       1 & : t > 1.
     \end{array}
   \right.
\end{flalign*}

Let $H_i$ the distribution of time the semi Markov process stays in state $i$ before transition. We have $H_j(t)= \sum_i P_{ij}F_{ij}(t)$, $\mu_i = \int_0 ^ \infty X dH_i(x)$. Let $X_n$ denote the $n^{\text{th}}$ state visited. Then $\{X_n\}$ is a Markov chain with transition probability $P$ called the embedded Markov chain of the semi Markov process. \\
\textbf{Definition:} If the embedded Markov chain is irreducible, then Semi Markov process is said to be irreducible. Let $T_{ii}$ denote the time between successive transitions to state $i$. Let $\mu_{ii}=E[T_{ii}]$.
\begin{thm}
If semi Markov process ius irreducible and if $T_{ii}$ has non-lattice distribution with $\mu_{ii}< \infty$ then, 
\begin{flalign*}
P_i=\lim_{t \rightarrow \infty}P(Z(t)=i|Z(0)=j)
\end{flalign*}
exists and is independent of the initial state. Furthermore, $P_i=\frac{\mu_i}{\mu_{ii}}$.
\end{thm}
\textbf{Corollary 4.8.2} If the semi-Markov process is irreducible and $\mu_{ii}<\infty$, then with probability 1,
$\frac{\mu_i}{\mu_{ii}}=\frac{\lim_{t \rightarrow \infty} \text{Amount of time in [0,t]}}{t}~\text{a.s}$.\\
\begin{thm}
Suppose conditions of the previous thmrem hold and the embedded Markov chain is positive recurrent. Then $P_i= \frac{\pi_i\mu_i}{\sum_{i}\pi_j \mu_j}$. 
\end{thm} 
\begin{proof}
Define the notation as follows:

$Y_i(j)=$ amount of time spent in state $i$ during $j^\text{th}$ visit to that state. $i,j \geq 0$. \\
$N_i(m)=$ number of visits to state $i$ in the first $m$ transitions of the semi-Markov process.\\

The proportion of time in $i$ during the first $m$ transitions:\\

\begin{flalign*}
P_{i=m}&= \frac{\sum_{j=1}^{N_i(m)}Y_i(j)}{\sum_l \sum_{j=1}^{N_i(m)}Y_i(j) }\\
&= \frac{\frac{N_i(m)}{m}\sum_{j=1}^{N_i(m)}Y_i(j)}{\sum_l \frac{N_i(m)}{m} \sum_{j=1}^{N_i(m)}Y_i(j) }\\
\end{flalign*}
Since $N_i(m)\rightarrow \infty$ as $m \rightarrow \infty$, it follows from the strong law of large numbers that $\frac{\sum_{i=2}^{N_i(m)}Y_i(j)}{N_i(m)}\rightarrow \mu_i$ and $\frac{N_i(m)}{m}\rightarrow (E[\text{number of transitions between visits to state }i])^{-1}=\pi_i$. Letting $m \rightarrow \infty$, result follows.
\end{proof}
\begin{thm}
If Semi Markov process is irreducible  and non lattice, then $\lim_{t \rightarrow \infty}P(Z(t)=i,Y(t)>x,S(t)=j|Z(0)=k)=\frac{P_{ij}\int_x^\infty F_{ij}^c(y)d(y)}{\mu_{ii}}$. Let $Y(t)$ denote the time from $t$ until the next transition. $S(t)$ state entered at the first transition after $t$. 
\end{thm}
\begin{proof}
The trick lies in defining the "ON" time. 
\begin{flalign*}
E[\text{ON time in a cycle}]=E[(X_{ij}-x)^+].
\end{flalign*}
\end{proof}
\textbf{Corollary:} 
\begin{flalign*}
\lim_{t \rightarrow \infty} P(Z(t)=i, Y(t) >x|Z(0)=k)= \frac{\int_{x}^{\infty}H_i^c(y)d(y)}{\mu_{ii}}.
\end{flalign*}

\end{document}
>>>>>>> 3ae3511b97af20b070d525442f60951d2a53ccda
