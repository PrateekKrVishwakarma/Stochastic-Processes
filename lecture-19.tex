% !TEX spellcheck = en_US
% !TEX spellcheck = LaTeX
\documentclass[a4paper,10pt,english]{article}
\usepackage{%
	amsfonts,%
	amsmath,%	
	etex,%
	amssymb,%
	amsthm,%
	babel,%
	bbm,%
	%biblatex,%
	caption,%
	centernot,%
	color,%
	enumerate,%
	epsfig,%
	epstopdf,%
	geometry,%
	graphicx,%
	hyperref,%
	latexsym,%
	mathtools,%
	multicol,%
	pgf,%
	pgfplots,%
	pgfplotstable,%
	pgfpages,%
	proof,%
	psfrag,%
	subfigure,%	
	tikz,%
	ulem,%
	url%
}	

\usepackage[mathscr]{eucal}
\usepgflibrary{shapes}
\usetikzlibrary{%
  arrows,%
	backgrounds,%
	chains,%
	decorations.pathmorphing,% /pgf/decoration/random steps | erste Graphik
	decorations.text,%
	matrix,%
  	positioning,% wg. " of "
  	fit,%
	patterns,%
  	petri,%
	plotmarks,%
  	scopes,%
	shadows,%
  	shapes.misc,% wg. rounded rectangle
  	shapes.arrows,%
	shapes.callouts,%
  	shapes%
}

\theoremstyle{plain}
\newtheorem{thm}{Theorem}[section]
\newtheorem{lem}[thm]{Lemma}
\newtheorem{prop}[thm]{Proposition}
\newtheorem{cor}[thm]{Corollary}

\theoremstyle{definition}
\newtheorem{defn}[thm]{Definition}
\newtheorem{conj}[thm]{Conjecture}
\newtheorem{exmp}[thm]{Example}
\newtheorem{assum}[thm]{Assumptions}
\newtheorem{axiom}[thm]{Axiom}

\theoremstyle{remark}
\newtheorem{rem}{Remark}
\newtheorem{note}{Note}

\newcommand{\norm}[1]{\left\lVert#1\right\rVert}
\newcommand{\indep}{\!\perp\!\!\!\perp}
\DeclarePairedDelimiter\abs{\lvert}{\rvert}%
%\DeclarePairedDelimiter\norm{\lVert}{\rVert}%
\newcommand{\tr}{\operatorname{tr}}
\newcommand{\R}{\mathbb{R}}
\newcommand{\Q}{\mathbb{Q}}
\newcommand{\N}{\mathbb{N}}
\newcommand{\E}{\mathbb{E}}
\newcommand{\Z}{\mathbb{Z}}
\newcommand{\B}{\mathscr{B}}
\newcommand{\C}{\mathcal{C}}
\newcommand{\T}{\mathscr{T}}
\newcommand{\F}{\mathcal{F}}
\newcommand{\G}{\mathcal{G}}
%\newcommand{\ba}{\begin{align*}}
%\newcommand{\ea}{\end{align*}}

\makeatletter
\def\th@plain{%
  \thm@notefont{}% same as heading font
  \itshape % body font
}
\def\th@definition{%
  \thm@notefont{}% same as heading font
  \normalfont % body font
}
\makeatother
\date{}
\title{Lecture 19 : Martingales}
\author{}
\begin{document}
\maketitle
\section{Martingales}
%A martingale is a type of stochastic process whose definition formalizes the concept of a fair game.
\begin{defn}
A stochastic process $\{Z_n,~n \in \N \}$ is said to be a \textbf{martingale} if 
\begin{enumerate}
\item $\E[|Z_n|]< \infty$, ~ \text{for all}~ n.
\item $\E[Z_{n+1}|Z_1,Z_2, \hdots Z_n]=Z_n$.
\end{enumerate}
If the equality in second condition is replaced by $\leq$ or $\geq$, then the process is called \textbf{supermartingale} or \textbf{submartingale}, respectively.
\end{defn}
\begin{rem} Taking expectation on both sides of part 2 of the above definition, we get $\E[Z_{n+1}]=\E[Z_n]$, and hence $\E[Z_{n+1}]=\E[Z_1]$, for all $n$.
\end{rem}
\begin{exmp}[Simple random walk]
Let $\{X_i\}$ be a sequence of independent random variables with mean $0$. Let $Z_n=\sum_{i=1}^n X_i$. Then, $\{Z_n,~n \in \N \}$ is a martingale. This is so because, $\E[Z_n]=0$ and 
\begin{align*}
\E[Z_{n+1}|Z_1,Z_2 \hdots Z_n] =\E[Z_{n}+X_{n+1}|Z_1,Z_2 \hdots Z_n]
%&=\E[Z_{n}|Z_1,Z_2 \hdots Z_n]+\E[X_{n+1}|Z_1,Z_2 \hdots Z_n]\\
&=Z_n.
\end{align*} 
\end{exmp}
\begin{exmp}
Let $\{X_i\}$ be a sequence of independent random variables with mean $1$. Let $Z_n=\Pi_{i=1}^n X_i$. Then, $\{Z_n,~n \in \N \}$ is a martingale. This is so because, $\E[Z_n]=1$ and 
\begin{align*}
\E[Z_{n+1}|Z_1,Z_2 \hdots Z_n] =\E[Z_{n}X_{n+1}|Z_1,Z_2 \hdots Z_n]
%&=Z_n\E[X_{n+1}|Z_1,Z_2 \hdots Z_n]\\
%&=Z_n\E[X_{n+1}]\\
&=Z_n.
\end{align*} 
\end{exmp}
\begin{exmp}[Branching Process] 
Let $\{X_n\}$ be a branching process. Let $X_0=1$. Then,
\begin{equation*}
X_n = \sum_{i=1}^{X_{n-1}}Z_i,
\end{equation*}
where $Z_i$ represents the number of offspring of the $i^{\text{th}}$ individual of the $(n-1)^{\text{st}}$ generation. conditioning on $X_{n-1}$ yields, $\E[X_n]= \mu^n$ where $\mu$ is the mean number of offspring per individual. Then $\{Y_n = X_n / \mu^n: n \in \N\}$ is a martingale because $\E[Y_n]= 1$ and 
\begin{align*}
\E[Y_{n+1}|Y_1, \hdots Y_n] %&= \frac{1}{\mu^{n+1}}\E[X_{n+1}|Y_1, \hdots Y_n]\\
&= \frac{1}{\mu^{n+1}}\E[\sum_{i=1}^{X_{n}}Z_i|Y_1, \hdots Y_n]
%&=  \frac{1}{\mu^{n+1}}X_{n}\E[Z_i]
= \frac{X_n}{\mu^n}=Y_n.
\end{align*}
\end{exmp}
\begin{exmp} [Doob's Martingale]
Let $X,Y_1,Y_2 \hdots$ be arbitrary random variables such that $\E[|X|]< \infty$. Then
\begin{equation*}
Z_n =\E[X|Y_1,Y_2, \hdots Y_n]
\end{equation*}
is a martingale. The integrability condition can be directly verified, and
\begin{align*}
\E[Z_{n+1}|Y_1,Y_2, \hdots Y_n]&= \E[\E[X|Y_1,\hdots Y_{n+1}]|Y_1,\hdots Y_{n}] = \E[X|Y_1,\hdots Y_{n}]]=Z_n.
\end{align*} 
%Thus the result follows. The above martingale is called the Doob type martingale.
\end{exmp}
\begin{exmp}
For any sequence of random variables $X_1,X_2 \hdots $, the random variables $X_i-\E[X_i|X_1 \hdots X_{i-1}]$ have zero mean. Define
\begin{equation*}
Z_n =\sum_{i=1}^n X_i -\E[X_i|X_1,X_2, \hdots X_{i-1}] 
\end{equation*}
 is  a martingale provided $\E[|Z_n|]< \infty$.  To verify the same, 
 \begin{flalign*}
\E[Z_{n+1}|Z_1 \hdots Z_n]&= \E[Z_n+X_n-\E[X_n|X_1 \hdots X_{n-1}]]\\
&= Z_n+\E[X_n-\E[X_n|X_1 \hdots X_{n-1}]]=Z_n.\\
\end{flalign*}
\end{exmp}
\subsection{Stopping Times}
\begin{defn}The positive integer values, possibly infinite, random variable $N$ is said to be a \textbf{random time} for the process $\{Z_n\}$ if the event $\{N=n\}$ is determined by the random variables $Z_1 \hdots Z_n$. If $\Pr\{N < \infty\}=1$, then the random time $N$ is said to be a \textbf{stopping time}.  
\end{defn}
\begin{defn}
Let $N$ be a random time for the process $\{Z_n\}$, then \textbf{stopped process} $\{\bar{Z}_n\}$ is defined as 
\begin{align*}
\bar{Z}_n = Z_n1_{\{n \leq N\}} + Z_N1_{\{n > N\}}.
\end{align*}
\end{defn}
\begin{prop}
If $N$ is a random time for the martingale $\{Z_n\}$, then the stopped process $\{\bar{Z}_n\}$ is also a martingale.
\end{prop}
\begin{proof}
We claim that 
\begin{equation*}
\bar{Z}_n= \bar{Z}_{n-1}+1_{\{n \leq N\}}(Z_n-Z_{n-1})
\end{equation*}
The above equation can be directly verified by considering the two cases separately by
\begin{align*}
\bar{Z}_n &= 
\begin{cases}
Z_n & n \leq N,\\
\bar{Z}_{n-1}=Z_N, & n > N.
\end{cases}
\end{align*}
Further, since $N$ is a random time, we see that 
\begin{align*}
\E[\bar{Z}_{n+1}|Z_1 \hdots \bar{Z}_n] %&=\E[\bar{Z}_{n}+1_{\{n \leq N\}}(Z_n-Z_{n-1})|Z_1 \hdots \bar{Z}_n]\\
&=\bar{Z}_{n}+1_{\{n \leq N\}} \E[(Z_n-Z_{n-1})|Z_1 \hdots \bar{Z}_n] =\bar{Z}_{n}.
\end{align*}
\end{proof}
\begin{rem}
For any martingale $\{\Z_n: n \in \N\}$, we have $\E[\bar{Z}_{n}]=\E[Z_1]$, for all $n$.  Now assume that $N$ is a stopping time. It is immediate that 
\begin{equation*}
\Pr\left\{\lim_{n\in \N}\bar{Z}_n = Z_N\right\} = 1.
\end{equation*}
\end{rem}
But  is it true that
\begin{equation*}
\lim_{n \in \N}\E[\bar{Z}_n] = \E[Z_N]?
\end{equation*}
It so turns out that the above is true under some additional regularity constraints only. %We state the following theorem without proof.
\begin{thm}[Martingale Stopping Theorem]
\label{MartStopThm}
If $N$ is a stopping time for a martingale $\{Z_n: n \in \N\}$ such that either of the following conditions is true:
\begin{enumerate}[(i)]
\item $N$ is bounded, 
\item $\bar{Z}_n$ is uniformly bounded,
\item $\E[N] < \infty$, and for some real positive $K$, we have $\sup_{n \in \N}\E[|Z_{n+1}-Z_n||Z_1 \hdots Z_n] < K$,
\end{enumerate}
then $Z_N$ is integrable and $\lim_{n \in \N}\E[\bar{Z}_n] = \E[Z_N]=\E[Z_1]$.
\end{thm}
\begin{proof} We show this is true for all three cases.
\begin{enumerate}[(i)] 
\item Let $K$ be the bound on $N$ then for all $n \geq K$, we have $\bar{Z}_n = \Z_N$, and hence it follows that
\begin{align*}
\E Z_1 = \E\bar{Z}_n &= \E Z_N, ~\forall n \geq K.
\end{align*}
\item Dominated convergence theorem implies the result. 
\item Since $N$ is integrable and  
\begin{align*} 
\bar{Z}_n \leq |Z_1| + K N,
\end{align*}
we observe that $\bar{Z}_n$ is bounded by an integrable random variable, and hence result follows from dominated convergence theorem.
\end{enumerate}
\end{proof}
\begin{cor}[Wald's Equation] If $N$ is a stopping time for $\{X_i,~ i \in \N\}$ \textit{iid} with $\E[|X|]< \infty$ and $\E[N]< \infty$, then
\begin{align*}
\E[\sum_{i=1}^{N}X_i]=\E[N]\E[X].
\end{align*}
\end{cor}
\begin{proof}
Let $\mu=\E[X]$. Then $\{Z_n = \sum_{i=1}^{n}(X_i-\mu): n \in \N\}$ is a martingale and hence from the Martingale stopping theorem, we have $\E[Z_N]=\E[Z_1]=0$. But 
\begin{align*}
\E[Z_N] %&= \E[\sum_{i=1}^N (X_i-\mu)]\\
%&=\E[\sum_{i=1}^N (X_i)-N\mu)]\\
&=\E\sum_{i=1}^N X_i- \mu\E N.
\end{align*}
Observe that condition $3$  for Martingale stopping theorem to hold can be directly verified. Hence the result follows. 
\end{proof}
%\section{ Submartingales, Supermartingales and the Martingale Convergence Theorem}
%\begin{defn}
%A stochastic process $\{Z_n,~  n \geq 1\}$ having $\E[|Z_n|]< \infty$ for all $n$ is said to be a submartingale if
%\begin{equation}
%\label{Submartingale}
%\E[Z_{n+1}|Z_1 \hdots Z_n] \geq Z_n
%\end{equation}
%and is said to be a supermartingale if
%\begin{equation}
%\label{Supermartingale}
%\E[Z_{n+1}|Z_1 \hdots Z_n] \leq Z_n
%\end{equation}
%\end{defn}
%From \ref{Submartingale}, for a submartingale
%\begin{equation*}
%\E[Z_{n+1}] \geq \E[Z_n]
%\end{equation*}
%where the inequality is reversed for a supermartingale. 
%\begin{thm}
%\label{Stoppingtime_theorem}
%If $N$ is a stopping time for $\{Z_n,~ n\geq 1\}$ such that any one of the following sufficient conditions is satisfied:
%\begin{enumerate}
%\item $\bar{Z}_n$ is uniformly bounded, or;
%\item $N$ is bounded, or;
%\item $\E[N]< \infty$, and there is an $M < \infty$ such that
%\begin{equation*}
%\E[|Z_{n+1}-Z_n| |Z_1, \hdots Z_n]<M,
%\end{equation*}
%then,
%\begin{eqnarray*}
%\E[Z_N] \geq \E[Z_1] ~ \text {for a submartingale}\\
%\E[Z_N] \leq \E[Z_1] ~ \text {for a supermartingale}.
%\end{eqnarray*}
%\end{enumerate}
%\end{thm}
%\begin{proof}
%We claim that 
%\begin{equation*}
%\bar{Z}_n= \bar{Z}_{n-1}+1_{N \geq n}(Z_n-Z_{n-1})
%\end{equation*}
%The above equation can be directly verified by considering the two cases separately viz. 
%\begin{enumerate}
%\item $N \geq n$: $\bar{Z}_n=Z_n$.
%\item $N < n:$ $\bar{Z}_{n-1}=\bar{Z}_{n}=Z_N$
%\end{enumerate}
%\begin{flalign*}
%\E[\bar{Z}_{n+1}|Z_1 \hdots \bar{Z}_n]&=\E[\bar{Z}_{n}+1_{n \leq N}(Z_n-Z_{n-1})|Z_1 \hdots \bar{Z}_n]\\
%&\stackrel{(a)}{=}\bar{Z}_{n}+1_{n \leq N} \E[(Z_n-Z_{n-1})|Z_1 \hdots \bar{Z}_n]\\
%& \geq \bar{Z}_{n},
%\end{flalign*}
%where in $(a)$ we have used the fact that $N$ is a random time. Also, we have $\E[\bar{Z}_{n}]=\E[Z_1]$, for all $n$.  Now assume that $N$ is a stopping time. It is immediate that
%\begin{equation*}
% \bar{Z}_n \rightarrow Z_N ~ \text{w.p}~ 1.
%\end{equation*}
%But  is it true that
%\begin{equation*}
% \E[\bar{Z}_n] \rightarrow \E[Z_N] ~ \text{as n}~ \rightarrow \infty.
%\end{equation*}
%which gives that 
%\begin{equation*}
%\E[Z_N ] \geq \E[Z_1].
%\end{equation*}
%\end{proof}

Before we state and prove martingale convergence theorem, we state some results which will be used in the proof of the theorem.
\begin{lem}
\label{StoppingTimeBound}
If $\{Z_i, i \in \N \}$ is  a submartingale and $N$ is a stopping time such that $\Pr\{N \leq n\}=1$ then
\begin{equation*}
 \E Z_1 \leq \E Z_N \leq \E Z_n.
\end{equation*}
\end{lem}
\begin{proof}
It follows from Theorem \ref{MartStopThm} that since $N$ is bounded, $\E[Z_N] \geq \E[Z_1]$. Now, since $N$ is a stopping time, we see that for $\{N = k\}$
\begin{eqnarray*}
\E[Z_n|Z_1, \hdots ,Z_N,N=k]&=\E[Z_n|Z_1 \hdots Z_k,N=k] = \E[Z_n|Z_1 \hdots Z_k] \geq Z_k = Z_N.
\end{eqnarray*}
%where $(a)$ follows from the fact that $N$ is  a stopping time. 
Result follows by taking expectation on both sides.
\end{proof}
\begin{lem}
\label{ConvexFuncSubmart}
If $\{Z_n,n \in \N\}$ is a martingale and $f$ is a convex function, then $\{f(Z_n),n \in \N\}$ is a submartigale.
\end{lem}
\begin{proof}
The result is a direct consequence of Jensen's inequality.
\begin{align*}
\E[f(Z_n)|Z_1, \hdots Z_n] &\geq f(\E[Z_{n+1}|Z_1, \hdots Z_n])=f(Z_n).
\end{align*}
\end{proof}
\begin{thm}[Kolmogorov's inequality for submartingales] If $\{Z_n,~ n \in \N \}$ is a submartingale, then
\begin{equation*}
\Pr\{\max\{Z_1,Z_2 \hdots Z_n\}>a\}\leq \frac{\E[Z_n]}{a},~ \text{for}~ a>0.
\end{equation*}
\end{thm}
\begin{proof}
Let $N = \min\{i \in [n]: Z_i >a\} \wedge n$. %, and define it to equal $n$ if $Z_i \leq a$ for all $i=1, \hdots n$. 
Then, $\{\max\{Z_1 \hdots Z_n\}>a\} = \{Z_N > a\}$. Since $N \leq n$, from Markov inequality, we have
\begin{align*}
\Pr\{\max\{Z_1 \hdots Z_n\}>a\}&=\Pr\{Z_N>a\} \leq \frac{\E[Z_N]}{a} \leq  \frac{\E[Z_n]}{a}.
\end{align*}
%where the last inequality follows from Lemma \ref{StoppingTimeBound} as $N \leq n$ and $(*)$ follows from Markov's inequality.
\end{proof}
\begin{cor}
\label{MartingaleBoundCor}
Let $\{Z_n,~n \geq 1\}$ be a martingale. Then, for $a>0$:
\begin{enumerate}
\item $\Pr\{\max\{|Z_1|, \hdots |Z_n|\}>a\} \leq \E[|Z_n|]/a$;
\item $\Pr\{\max\{|Z_1|, \hdots |Z_n|\}>a\} \leq \E[Z_n^2]/a^2$.
\end{enumerate} 
\end{cor}
\begin{proof}
The proof the above statements follow from Lemma \ref{ConvexFuncSubmart} and Kolmogorov's inequality for submartingales by considering the convex functions $f(x)=|x|$ and $f(x)=x^2$. 
\end{proof}
\begin{thm}[Martingale Convergence Theorem]
\label{MartingaleConvergenceTheorem}
If $\{Z_n,~n \geq 1\}$ is a martingale such that for some $M< \infty$
\begin{equation*}
\E[|Z_n|] \leq M, ~ \text{for all}~ n
\end{equation*}
then, with probability 1, $\lim_{n \rightarrow \infty}Z_n$ exists and is finite.
\end{thm}
\begin{proof}
Assume $\E[Z_n^2]< \infty$ which is stronger than $\E[|Z_n|]< \infty$ (as a consequence of Jensen's inequality). Observe that $\{Z_n^2\}$ is a submartingale (from Lemma \ref{ConvexFuncSubmart}). Thus $\E[Z_n^2]<\infty$ and is non-decreasing in $n$. Thus, as $n \rightarrow \infty$, $\E[Z_n^2]$ converges and let $\mu<\infty$ be given by $\mu=\lim_{n \rightarrow \infty}\E[Z_n^2]$.
\begin{equation}
\label{KolmoBound}
\Pr(\cup_{k \leq n} \{|Z_{m+k}-Z_m|> \epsilon\} )
\end{equation}  
\begin{eqnarray*}
&\stackrel{(a)}{\leq }\E[(Z_{m+n}-Z_m)^2]/\epsilon^2
&=\E[Z_{m+n}^2-2Z_mZ_{m+n}+Z_m^2]/\epsilon^2.
\end{eqnarray*}
Note that 
\begin{eqnarray*}
\E[Z_{m+n}Z_m]&=\E[\E[Z_mZ_{m+n}|Z_m]]\\
&=\E[Z_m\E[Z_{m+n}|Z_m]]\\
&=\E[Z_m^2].
\end{eqnarray*}
From \ref{KolmoBound}, 
\begin{equation*}
\Pr(\cup_{k \leq n} \{|Z_{m+k}-Z_m|> \epsilon\}) \leq \frac{\E[Z_{m+n}^2]-\E[Z_m^2]}{\epsilon^2}.
\end{equation*}
Letting $n \rightarrow \infty$
\begin{equation*}
\Pr(\cup_{k \leq 1} \{|Z_{m+k}-Z_m|> \epsilon\}) \leq \frac{\mu-\E[Z_m^2]}{\epsilon^2}.
\end{equation*}
Hence,
\begin{equation*}
\Pr(\cup_{k \leq n} \{|Z_{m+k}-Z_m|> \epsilon\}) \rightarrow 0 ~\text{as}~ m \rightarrow \infty.
\end{equation*}
Thus with probability 1, $\{Z_n\}$ will be  a Cauchy sequence, and thus $\lim_{n \rightarrow \infty}Z_n$ will exist and be finite.`
\end{proof}
\begin{cor}
If $\{Z_n,~m \geq 0\}$ is a non-negative martingale, then, with probability 1, $\lim_{n \rightarrow \infty}Z_n$ exists and is finite.
\end{cor}
\begin{proof}
Since $Z_n$ is non-negative,
\begin{equation*}
\E[|Z_n|]=\E[Z_n]=\E[Z_1].
\end{equation*}
\end{proof}

\end{document}
