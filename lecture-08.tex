% !TEX spellcheck = en_US
% !TEX spellcheck = LaTeX
\documentclass[a4paper,10pt,english]{article}
\usepackage{%
	amsfonts,%
	amsmath,%	
	etex,%
	amssymb,%
	amsthm,%
	babel,%
	bbm,%
	%biblatex,%
	caption,%
	centernot,%
	color,%
	enumerate,%
	epsfig,%
	epstopdf,%
	geometry,%
	graphicx,%
	hyperref,%
	latexsym,%
	mathtools,%
	multicol,%
	pgf,%
	pgfplots,%
	pgfplotstable,%
	pgfpages,%
	proof,%
	psfrag,%
	subfigure,%	
	tikz,%
	ulem,%
	url%
}	

\usepackage[mathscr]{eucal}
\usepgflibrary{shapes}
\usetikzlibrary{%
  arrows,%
	backgrounds,%
	chains,%
	decorations.pathmorphing,% /pgf/decoration/random steps | erste Graphik
	decorations.text,%
	matrix,%
  	positioning,% wg. " of "
  	fit,%
	patterns,%
  	petri,%
	plotmarks,%
  	scopes,%
	shadows,%
  	shapes.misc,% wg. rounded rectangle
  	shapes.arrows,%
	shapes.callouts,%
  	shapes%
}

\theoremstyle{plain}
\newtheorem{thm}{Theorem}[section]
\newtheorem{lem}[thm]{Lemma}
\newtheorem{prop}[thm]{Proposition}
\newtheorem{cor}[thm]{Corollary}

\theoremstyle{definition}
\newtheorem{defn}[thm]{Definition}
\newtheorem{conj}[thm]{Conjecture}
\newtheorem{exmp}[thm]{Example}
\newtheorem{assum}[thm]{Assumptions}
\newtheorem{axiom}[thm]{Axiom}

\theoremstyle{remark}
\newtheorem{rem}{Remark}
\newtheorem{note}{Note}

\newcommand{\norm}[1]{\left\lVert#1\right\rVert}
\newcommand{\indep}{\!\perp\!\!\!\perp}
\DeclarePairedDelimiter\abs{\lvert}{\rvert}%
%\DeclarePairedDelimiter\norm{\lVert}{\rVert}%
\newcommand{\tr}{\operatorname{tr}}
\newcommand{\R}{\mathbb{R}}
\newcommand{\Q}{\mathbb{Q}}
\newcommand{\N}{\mathbb{N}}
\newcommand{\E}{\mathbb{E}}
\newcommand{\Z}{\mathbb{Z}}
\newcommand{\B}{\mathscr{B}}
\newcommand{\C}{\mathcal{C}}
\newcommand{\T}{\mathscr{T}}
\newcommand{\F}{\mathcal{F}}
\newcommand{\G}{\mathcal{G}}
%\newcommand{\ba}{\begin{align*}}
%\newcommand{\ea}{\end{align*}}

\makeatletter
\def\th@plain{%
  \thm@notefont{}% same as heading font
  \itshape % body font
}
\def\th@definition{%
  \thm@notefont{}% same as heading font
  \normalfont % body font
}
\makeatother
\date{}
\title{Lecture 08: Branching Processes and Delayed Renewal Process}
\author{}

\begin{document}
\maketitle

\section{Age-dependent Branching Process }

Suppose an organism lives upto a time period of $X \sim F$ and produces $N \sim P$ number of  offspring. Let $X(t)$ denote the number of organisms alive at time $t$. The stochastic process $\{X(t),~ t \geq 0\}$ is called an age-dependent branching process. We are interested in computing $M(t)=\mathbb{E}[X(t)]$ when $m=\mathbb{E}[N] =\sum_{j \in \mathbb{N}}{j P_j}$. 

\begin{thm}
If $X(0)=1$, $m>1$ and $F$ is non lattice, then
\begin{flalign*}
\lim_{t \rightarrow \infty} e^{-\alpha t}M(t)= \frac{m-1}{m^2 \alpha \int_{0}^{\infty}xe^{-\alpha x dF(x)}},
\end{flalign*}

where $\alpha > 0$ is unique such that $\int_{0}^{\infty}xe^{-\alpha x } dF(x) = \frac{1}{m}$.
\end{thm}

\begin{proof}
Condition on $T_1$, the life time of first organism,
\begin{flalign*}
M(t)&=\int_{0}^{\infty}\mathbb{E}[X(t)|T_1=y]dF(y)\\
&\stackrel{(a)}{=}\int_{y=0}^{t} 1 dF(y) + \int_{y=t}^{\infty} m M(t-y)dF(y).
\end{flalign*}

Thus we get 

\begin{flalign}
\label{renew}
M(t)= F^c(t)+m\int_{0}^{t}M(t-y)dF(y)
\end{flalign}

Let $\alpha$ denote the unique positive number such that $\int_{0}^{\infty}xe^{-\alpha x } dF(x) = \frac{1}{m}$ and $G(y)=m\int_{0}^{y}e^{-alpha y} dF(y)$. Upon multiplying both sides of equation (\ref{renew}) by  $e^{-\alpha t}$ and defining $f(t)=e^{-\alpha t}M(t)$, $h(t)=e^{-\alpha t}F^{c}(t)$,  

\begin{flalign*}
f&=h+f*G\\
&=h+G*(h+f*G)\\
\vdots
&=h+h*\sum_{i=1}^{\infty}G_i\\
&=h+h*m_G.
\end{flalign*}

Or, $f(t)=h(t)+\int_{0}^{t}h(t-s)dm_G(s)$. It can be shown that $h(t)$ is dRi and hence by Key Renewal thmrem, 
\begin{flalign*}
f(t) \rightarrow \frac{\int_{o}^{\infty}e^{-\alpha t}F^c(t) dt }{\int_{0}^{\infty}xdG(x) }.
\end{flalign*}

\begin{flalign*}
\int_{o}^{\infty}e^{-\alpha t}F^c(t) dt &= \int_{0}^{\infty} e^{-\alpha t}\int_{t}^{\infty}dF(x)dt \\
&=  \int_{0}^{\infty} \int_{0}^{x} e^{-\alpha t} dt  dF(x)\\
&= \int_{0}^{\infty} (1- e^{-\alpha x}) dF(x) \\
&= \frac{1}{\alpha}(1-\frac{1}{m}) ~~ (\text{by the definition of} ~\alpha).
\end{flalign*}

Also $\int_{0}^{\infty}xdG(x) = m \int_{0}^{\infty}xe^{-\alpha x}dF(x)$. Hence the result follows.

\end{proof} 

\section{Delayed Renewal Process}
Let $\{X_n: n \in \mathbb{N}\}$ be independent but $X_1 \sim G$ and $X_i \sim F,~ i \geq 2$ then the counting process $\{N_D(t): t \geq 0\}$ is called general renewal process or delayed renewal process. Let $S_0=0$ and $S_n =\sum_{i=1}^{n}X_i$. We have 
\begin{flalign*}
N_D(t) = \sup \{n \in \mathbb{N}: S_n \leq t\},\\
P(N_D(t)=n) &= P(S_n \leq t)-P(S_{n+1} \leq t)\\
&=G*F^{n-1}(t)-G*F^n(t),\\
m_D(t)=\mathbb{E}[N_D(t)]= \sum_{n \in \mathbb{N}} G*F^{n-1}(t).
\end{flalign*}

Taking the Laplace transform of $m_D(t)$, denoted as $\tilde{m}_D(s) = \frac{\tilde{G}(s)}{1-\tilde{F}(s)}$.

\begin{prop}
The following holds:
\begin{enumerate}
\item $\lim_{t \rightarrow \infty} \frac{N_D(t)}{t} = \frac{1}{\mu} $.
\item $\lim_{t \rightarrow \infty} \frac{m_D(t)}{t} = \frac{1}{\mu} $.
\item If $F$ is non-lattice, $lim_{t \rightarrow \infty} m_D(t+a)-m_D(t)=\frac{a}{\mu_F}$.
\item If $F$ and $G$ are lattice with period $d$, $\mathbb{E}$[ $\#$of renewals at $nd$ ] =$\frac{d}{\mu_F}.$
\item If $F$ is nonlattice, $\mu < \infty$ and $h$ dRi, then 
\begin{flalign*}
\lim_{t \rightarrow \infty} \int_{0}^{t}h(t-x)dm_D(x) = \frac{\int_{0}^{\infty}h(t) dt}{\mu}.
\end{flalign*}
\end{enumerate}
\end{prop}

\subsubsection{Example:}
Let $\{X_n: n \in \mathbb{N}\}$ be iid discrete observed. A pattern $x_1,x_2 \hdots x_k$ is said to occur at time $n$ if $X_n=x_k,~X_{n-1}=x_{k-1}, \hdots X_{n-k+1}=x_1.$ This forms a renewal process with inter arrivals being lattice random variables with $d=1$. The probability of a renewal at $n$ is thus $\prod_{i=1}^k P[X=x_i].$ From Blackwell's theorem for a strictly positive lattice random variable, we have 
\begin{align}
\lim_{n\to \infty} P[\text{pattern at $n$}]= \frac{1}{\E[\text{time between renewals}]},
\end{align}
from which it follows that 
\begin{align}
\E[\text{time between renewals}] = \frac{1}{\prod_{i=1}^k P[X=x_i]}.
\end{align}

If we have iid tosses and consider $N(n)$ as the number of times pattern $0,1,0,1$ appear in $n$ tosses, with $P(H)=p=1-q,$~the process is a delayed renewal processes. To find the mean number of tosses for the first time the pattern $0,1,0,1$ appear, \\

\begin{flalign*}
\mathbb{E}[\text{first time pattern}~ 0,1,0,1 ~ \text{appears}]&= \mathbb{E}[\text{first time pattern}~ 0,1 ~ \text{appears}]  \\
&+\mathbb{E}[\text{time between patterns}~ 0,1,0,1 ]\\
&=  p^{-1}q^{-1}+ p^{-2}q^{-2}.
\end{flalign*}

Similarly we can show that $\mathbb{E}[\text{first time}~ k \text{heads} ] = \sum_{i=1}^{n} p^{-i}$.

\end{document}