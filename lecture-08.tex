% !TEX spellcheck = en_US
% !TEX spellcheck = LaTeX
\documentclass[a4paper,10pt,english]{article}
\usepackage{%
	amsfonts,%
	amsmath,%	
	etex,%
	amssymb,%
	amsthm,%
	babel,%
	bbm,%
	%biblatex,%
	caption,%
	centernot,%
	color,%
	enumerate,%
	epsfig,%
	epstopdf,%
	geometry,%
	graphicx,%
	hyperref,%
	latexsym,%
	mathtools,%
	multicol,%
	pgf,%
	pgfplots,%
	pgfplotstable,%
	pgfpages,%
	proof,%
	psfrag,%
	subfigure,%	
	tikz,%
	ulem,%
	url%
}	

\usepackage[mathscr]{eucal}
\usepgflibrary{shapes}
\usetikzlibrary{%
  arrows,%
	backgrounds,%
	chains,%
	decorations.pathmorphing,% /pgf/decoration/random steps | erste Graphik
	decorations.text,%
	matrix,%
  	positioning,% wg. " of "
  	fit,%
	patterns,%
  	petri,%
	plotmarks,%
  	scopes,%
	shadows,%
  	shapes.misc,% wg. rounded rectangle
  	shapes.arrows,%
	shapes.callouts,%
  	shapes%
}

\theoremstyle{plain}
\newtheorem{thm}{Theorem}[section]
\newtheorem{lem}[thm]{Lemma}
\newtheorem{prop}[thm]{Proposition}
\newtheorem{cor}[thm]{Corollary}

\theoremstyle{definition}
\newtheorem{defn}[thm]{Definition}
\newtheorem{conj}[thm]{Conjecture}
\newtheorem{exmp}[thm]{Example}
\newtheorem{assum}[thm]{Assumptions}
\newtheorem{axiom}[thm]{Axiom}

\theoremstyle{remark}
\newtheorem{rem}{Remark}
\newtheorem{note}{Note}

\newcommand{\norm}[1]{\left\lVert#1\right\rVert}
\newcommand{\indep}{\!\perp\!\!\!\perp}
\DeclarePairedDelimiter\abs{\lvert}{\rvert}%
%\DeclarePairedDelimiter\norm{\lVert}{\rVert}%
\newcommand{\tr}{\operatorname{tr}}
\newcommand{\R}{\mathbb{R}}
\newcommand{\Q}{\mathbb{Q}}
\newcommand{\N}{\mathbb{N}}
\newcommand{\E}{\mathbb{E}}
\newcommand{\Z}{\mathbb{Z}}
\newcommand{\B}{\mathscr{B}}
\newcommand{\C}{\mathcal{C}}
\newcommand{\T}{\mathscr{T}}
\newcommand{\F}{\mathcal{F}}
\newcommand{\G}{\mathcal{G}}
%\newcommand{\ba}{\begin{align*}}
%\newcommand{\ea}{\end{align*}}

\makeatletter
\def\th@plain{%
  \thm@notefont{}% same as heading font
  \itshape % body font
}
\def\th@definition{%
  \thm@notefont{}% same as heading font
  \normalfont % body font
}
\makeatother
\date{}
\title{Lecture 07: Inspection Paradox and Limiting Mean Excess Time}
\author{}

\begin{document}
\maketitle

%%%%%%%%%%%%%%%%%%%%%%%%%%%%%%%%%%%%%%%%%%%%%%%%%%%%%%%%%%5  
\section{The Inspection Paradox}
%%%%%%%%%%%%%%%%%%%%%%%%%%%%%%%%%%%%%%%%%%%%%%%%%%%%%%%%%%
Define $X_{N(t)+1}=A(t)+Y(t)$ as the length of the renewal interval containing $t$, in other words, the length of current renewal interval. Inspection paradox says that $P(X_{N(t)+1} >x)\geq \bar{F}(x)$. That is, for any $x$, the length of the current renewal interval to be greater than $x$ is always more likely than that for an ordinary renewal interval. Formally,
\begin{flalign*}
\Pr\{X_{N(t)+1}>x\}&= \int_{0}^t\Pr\{X_{N(t)+1} > x | S_{N(t)} = y, N(t)=n\}dF_{(S_{N(t)}, N(t))}.
%&= P(X_2 >x)\\
%&= \bar{F}(x).
\end{flalign*}
Now we have,
\begin{flalign*}
\Pr\{X_{N(t)+1}>x | S_{N(t)}=y, N(t)=n\} & = \Pr\{X_{N(t)+1}>x | X_1+\cdots+X_n=y, X_{n+1}>t-y\} \\
& = \Pr\{X_{n+1}>x | X_{n+1}>t-y\} \\
& = \frac{\Pr\{X_{n+1}>\text{max}(x,t-y)\}}{\Pr\{X_{n+1}>t-y\}} \\
& \geq \bar{F}(x). 
\end{flalign*}
So we get that,
\begin{flalign*}
\Pr\{X_{N(t)+1}>x\}\geq \Pr\{X_{1}>x\}.
\end{flalign*}
One can also look into a weaker version of inspection paradox involving
the limiting distribution of $X_{N(t)+1}$, consider an alternating 
renewal process for which the ON time is the total time of the cycle if that 
total time is greater than $x,$ and zero otherwise. The system is either totally ON 
during a cycle (if the renewal interval is greater than $x$), or totally OFF 
otherwise. Formally,
\begin{align*}
Z_n= &\text{ ON time in $n^{\text{th}}$ cycle} = X_n \mathbb{I}_{X_n>x} \\
Y_n= &\text{ OFF time in $n^{\text{th}}$ cycle} = X_n \mathbb{I}_{X_n\leq x}.
\end{align*}
Now we have,
\begin{flalign*}
\Pr \{X_{N(t)+1}>x\} &= \Pr\{\text{length of the interval containing } t>x\}\\
&= \Pr\{ \text{on at time } t \}.
\end{flalign*}

From alternating renewal process theorem, we conclude that 
\begin{flalign*}
\lim_{t\to \infty}\Pr\{X_{N(t)+1}>x\} &= \frac{\E[\text{on time in cycle}]}{\mu} \\
&= \frac{\E[X\mathbb{I}_{X>x}]}{\mu}\\
&= \frac{\int_{x}^\infty y dF(y)}{\mu}\\
&\geq \Pr[X_1\geq x],
\end{flalign*}
where the last step follows from Chebyshev's inequality stated below.

\begin{rem}
The inspection paradox states, in essence, that if we pick a point $t$, it is more likely that an inter-renewal interval with larger length will contain $t$ than the smaller ones. For instance, if $X_i$ were equally likely to be $\epsilon$ or $1-\epsilon$, we see that the mean of any inter arrival length is 1 for any value of $\epsilon\in(0,1)$. However, for small values $\epsilon$, it is more likely that a given $t$ will be in an interval of length $1-\epsilon$ than in an interval of length $\epsilon$.
\end{rem}

\subsubsection*{Chebyshev's Sum Inequality:}
 If $f:\mathbb{R} \rightarrow \mathbb{R}^{+}$ and $g : \mathbb{R} \rightarrow \mathbb{R}^{+}$ are functions with the same
 monotonicity then for any random variable $X$, $f(X)$ and $g(X)$ are positive and
 $$\E[f(X)g(X)] \geq \E[f(X)]\E[g(X)].$$
\textbf{Remark:} \\
 This inequality gives us that
 $$\E[X\mathbb{I}_{X\geq x}] \geq \E[X]\Pr[X\geq x].$$

\subsection{Example:}
 Suppose the number of commodities desired by a customer at a store follows a distribution $G$. The ordering policy of the store is as follows: For some fixed $s,~S$, if the inventory level after serving a customer is $x$, then the amount ordered is
 
 

     \begin{displaymath}
        \left\{
         \begin{array}{lr}
           S-x & \text{if } x <s\\
           0 & \text{if } x \geq s
         \end{array}
       \right.
    \end{displaymath} 

Let $L(t)$ denote the inventory level at time $t$. We are interested in finding $\lim_{t \rightarrow \infty}\mathbb{P}(L(t) \geq y)$. 
Let $X_n$ denote inter-restocking times. Let $\{L(t)\geq y\}$ denote ON period. $X_n$ forms an 
alternating renewal process with the above mentioned ON time. 
From alternating renewal process theorem, we have 

\begin{flalign*}
\lim_{t \rightarrow \infty}\mathbb{P}(L(t) \geq y) &= \frac{\mathbb{E}[\text{ON time}]}{\mathbb{E}[X_1]}\\
&=\frac{\mathbb{E}[\sum_{i=1}^{N_y}X_i]}{\mathbb{E}[\sum_{i=1}^{N_s}X_i]}=\frac{\mathbb{E}[N_x]}{\mathbb{E}[N_s]}.
\end{flalign*}

where $N_y= \min\{n \in \mathbb{N}: \sum_{i=1}^{n}D_i > S-y\}$  and $D_1,D_2 \hdots$ denote the successive customer demands. Since $D_i$ are iid, we can interpret $N_y-1$ as the number of renewals till time $S-y$. $D_i$ is the inter arrival time of the process. Thus   

\begin{flalign*}
\lim_{t \rightarrow \infty}\mathbb{P}(L(t) \geq y) =\frac{m_G(S-x)+1}{m_G(S-s)+1}, s \leq x \leq S.
\end{flalign*}
\section{Limiting Mean Excess Time}
Consider a nonlattice renewal process and we are interested in computing the mean excess time of the process. We start by writing the renewal equation of mean excess life time, $\mathbb{E}[Y(t)]$.
\begin{flalign*}
\mathbb{E}[Y(t)]&= \mathbb{E}[Y(t)|S_{N(t)}=0]F^c(t)+ \int_{0}^{t} \mathbb{E}[Y(t)|S_{N(t)}=y]F^c(t-y)dm(y)\\
&=\mathbb{E}[X_1-t |X_1>t]F^c(t)+ \int_{0}^{t} \mathbb{E}[X-(t-y)|X>t-y]F^c(t-y)dm(y).
\end{flalign*}
From Key Renewal theorem, we have 

\begin{flalign*}
\lim_{t \rightarrow \infty}\mathbb{E}[Y(t)]&=\frac{1}{\mu} \int_{0}^{\infty} \mathbb{E}[X-t|X-t >0]F^c(t) dt\\
&= \frac{1}{\mu} \int_{t=0}^{\infty} \int_{x=t}^{\infty}(x-t) dF(x) dt\\
&= \frac{1}{\mu} \int_{x=0}^{\infty} \int_{t=0}^{x}(x-t) dF(x) dt\\
&= \frac{\mathbb{E}[X^2]}{2\mu}.
\end{flalign*}

\begin{rem}
It can also be established that $\lim_{t\to \infty}\frac{\int_0^t Y(\tau) d\tau}{t} = \frac{EX^2}{2\mu}$ since, for each sample path $\{Y(t)= y(t)\}$ and sample values $\{x_i,i\in\mathbb{N}\}$, we have $\int_0^ty(\tau)d\tau = \frac{1}{2}\sum_{i=1}^{n(t)}x_i^2+\int_{\tau=s_{n(t)}}^t y(\tau)d\tau$.
\end{rem}



\begin{prop}
If the inter arrival time is nonlattice and $\mathbb{E}[X^2] < \infty$, we have 
\begin{flalign*}
\lim_{t \rightarrow \infty} \left(m(t)-\frac{t}{\mu}\right) = \frac{\mathbb{E}[X^2]}{2\mu^2}-1.
\end{flalign*} 
\begin{proof}
This follows since $\mu (m(t)+1) = t + \mathbb{E}[Y(t)].$
\end{proof}

\end{prop}












\end{document}