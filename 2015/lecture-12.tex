\documentclass[a4paper,10pt]{article}
\usepackage{%
	amsfonts,%
	amsmath,%	
	etex,%
	amssymb,%
	amsthm,%
	babel,%
	bbm,%
	%biblatex,%
	caption,%
	centernot,%
	color,%
	enumerate,%
	epsfig,%
	epstopdf,%
	geometry,%
	graphicx,%
	hyperref,%
	latexsym,%
	mathtools,%
	multicol,%
	pgf,%
	pgfplots,%
	pgfplotstable,%
	pgfpages,%
	proof,%
	psfrag,%
	subfigure,%	
	tikz,%
	ulem,%
	url%
}	

\usepackage[mathscr]{eucal}
\usepgflibrary{shapes}
\usetikzlibrary{%
  arrows,%
	backgrounds,%
	chains,%
	decorations.pathmorphing,% /pgf/decoration/random steps | erste Graphik
	decorations.text,%
	matrix,%
  	positioning,% wg. " of "
  	fit,%
	patterns,%
  	petri,%
	plotmarks,%
  	scopes,%
	shadows,%
  	shapes.misc,% wg. rounded rectangle
  	shapes.arrows,%
	shapes.callouts,%
  	shapes%
}

\theoremstyle{plain}
\newtheorem{thm}{Theorem}[section]
\newtheorem{lem}[thm]{Lemma}
\newtheorem{prop}[thm]{Proposition}
\newtheorem{cor}[thm]{Corollary}

\theoremstyle{definition}
\newtheorem{defn}[thm]{Definition}
\newtheorem{conj}[thm]{Conjecture}
\newtheorem{exmp}[thm]{Example}
\newtheorem{assum}[thm]{Assumptions}
\newtheorem{axiom}[thm]{Axiom}

\theoremstyle{remark}
\newtheorem{rem}{Remark}
\newtheorem{note}{Note}

\newcommand{\norm}[1]{\left\lVert#1\right\rVert}
\newcommand{\indep}{\!\perp\!\!\!\perp}
\DeclarePairedDelimiter\abs{\lvert}{\rvert}%
%\DeclarePairedDelimiter\norm{\lVert}{\rVert}%
\newcommand{\tr}{\operatorname{tr}}
\newcommand{\R}{\mathbb{R}}
\newcommand{\Q}{\mathbb{Q}}
\newcommand{\N}{\mathbb{N}}
\newcommand{\E}{\mathbb{E}}
\newcommand{\Z}{\mathbb{Z}}
\newcommand{\B}{\mathscr{B}}
\newcommand{\C}{\mathcal{C}}
\newcommand{\T}{\mathscr{T}}
\newcommand{\F}{\mathcal{F}}
\newcommand{\G}{\mathcal{G}}
%\newcommand{\ba}{\begin{align*}}
%\newcommand{\ea}{\end{align*}}

\makeatletter
\def\th@plain{%
  \thm@notefont{}% same as heading font
  \itshape % body font
}
\def\th@definition{%
  \thm@notefont{}% same as heading font
  \normalfont % body font
}
\makeatother
\date{}
\title{Lecture 12 : Continuous Time Markov Chains}
\author{Parimal Parag}
\begin{document}
\maketitle
\section{Continuous Time Markov Chains}

Consider a continuous time stochastic process $\{X(t),~ t \geq 0\}$ taking on values on the set of non-negative integers. The process $\{X(t),~t \geq 0\}$ is continuous time Markov chain if for all $s,~t \geq 0$ and $i,~j \in \mathbb{N}_0$, $P(X(t+s)|X(u),~ u \in [0,s])= P(X(t+s)|X(s))$. If the probability is independent of $s$, the continuous time Markov chain (CTMC) has homogeneous transitions and we denote the probability by $P_{ij}(t)$. Suppose $X(0)=i,~X(u)=i,~ \forall u \in [0,s]$. We are interested in knowing probabilities of the form $P(X(v)=i,~v \in [s,s+t]||X(u)=i, ~ u \in [0,s])$. To that end, $\tau_i \triangleq \{t \geq 0: X(t) \neq i|X(0)=i\}$. We could sample the process at these instants and construct a DTMC out of it and study the same. Observe that,
\begin{flalign*}
P(\tau_i \geq s+t | \tau_i > s)&=P(X(v)=i,~v \in [s,s+t]|X(0)=i)\\
&=P(\tau_i \geq t | X(0)=i)
\end{flalign*}

$\tau_i$ is memoryless and exponentially distributed. 

\subsection{Alternative way of constructing CTMC}
A CTMC is a stochastic process that each time it enters state $i$,\\
\begin{enumerate}
\item {$\tau_i \sim $ exp$(\nu_i)$. }\\
\item {P(Entering state $j$| state $i$ ) $\equiv$ $P_{ij}$ is such that $\sum_{i \neq j}P_{ij}=1$.}
\end{enumerate}
Note that $P_{i,j}$ and $\tau_i$ are not dependent. $\nu_i$ is called rate of state $i$. $\nu_i < \infty$. Else we call the state to be instantaneous. A CTMC is a DTMC with exponential sojourn time in each state.

\textbf{Definition:} A CTMC is called "regular" if $P(\# transitions in [0,t] is \text{finite})=1,~ \forall t < \infty$. Consider the following example of a non-regular CTMC. \\
\textbf{Example:} $P_{i,i+1}=1, \nu_i = i^2$. Show that it is regular.\\

Let $Q$ be a matrix such that \\
\begin{enumerate}
\item {$q_{ij}=\nu_i P_{ij}$, $\forall i \neq j$.} \\
\item { $q_{ii}= -\nu_i$.}
\end{enumerate}
 Properties of $Q:$
 \begin{enumerate}
\item {$0 \leq -q_{ii} < \infty,~ \forall i$.} \\
\item { $q_{ij} \geq 0,~ \forall i \neq j$.}\\
\item { $\sum_{j}q_{ij}=0,~ \forall i$.}
\end{enumerate}

From the $Q$ matrix, we can construct the whole CTMC.  In DTMC, we had the result $P^{(n)}(i,j)=(P^n)_{i,j}$. We can generalize this notion  in the case of CTMC as follows: $P=e^{Q}\triangleq \sum_{k \in \mathbb{N}_0}\frac{Q^k}{k !}$.  Observe that $e^{Q_1+Q_2}=e^{Q_1}e^{Q_2},~ e^{nQ}=(e^Q)^n=P^n$.\\
\begin{thm}
Let $Q$ be a finite sized matrix. Let $P(t)=e^{tQ}$. Then $\{P(t),~ t \geq 0\}$ has the following properties:\begin{enumerate}
\item {$P(s+t)=P(s)P(t),~ \forall s,~t$ (semi group property).}
\item {$P(t),~t \geq 0$ is the unique solution to the forward equation, $\frac{dP(t)}{dt}=P(t)Q,~P(0)=I$.}
\item {And the backward equation $\frac{dP(t)}{dt}=QP(t),~P(0)=I$.}\\
\item {For all $k \in \mathbb{N}$, $\frac{d^kP(t)}{d^k(t)}|_{t=0}=Q^k$.}
\end{enumerate}
\end{thm}  
\begin{proof}
$\frac{dM(t)e^{-tQ}}{dt}=0,$ $M(t)e^{-tQ}$ is constant. $M(t)$ is any matrix satisfying the forward equation.
\end{proof}
\begin{thm}
A finite matrix $Q$ is $Q$ matrix if and only if $P(t)=e^{tQ}$ is a stochastic matrix for all $t \geq 0$. 
\end{thm}
\begin{proof}
$P(t)=I+tQ+O(t^2)$ ($f(t)=O(t) \Rightarrow \frac{f(t)}{t} \leq c,$ for small $t,~c < \infty$ ). $q_{ij} \geq 0$ if and only if $P_{ij}(t) \geq 0,~ \forall i \neq j$ and $t \geq 0$ sufficiently small. $P(t)=P(\frac{t}{n})^n$. Note that if $Q$ has zero row sums, $Q^n$ also has zero row sums.\\
\begin{flalign*}
\sum_j [Q^n]_{ij}&= \sum_j \sum_k [Q^{n-1}]_{ik}Q_{kj}= \sum_j \sum_k Q_{kj}[Q^{n-1}]_{ik}=0.\\
\sum_{j}P_{ij}(t)&=1+\sum_{n \in \mathbb{N}} \frac{t^n}{n!}\sum_j [Q^n]_{ij}=1+0=1.
\end{flalign*}  
Conversely $\sum_{j}P_{ij}(t)=1,~ \forall t \geq 0$, then $\sum_jQ_{ij}= \frac{dP_{ij}(t)}{dt}=0$.
\end{proof}
\subsection{Kolmogorov Differential Equations}
\begin{lem}
\begin{enumerate}
\item{$\lim_{t \rightarrow 0} \frac{1-P_{ii}(t)}{t}=\nu_i$.}\\
\item{$\lim_{t \rightarrow 0} \frac{P_{ij}(t)}{t}=q_{ij}$.}
\end{enumerate}
\end{lem}
\begin{lem}
For all $s,~t \geq 0$, $P_{ij}(t+s)=\sum_{k \in \mathbb{N}_0 P_{ik}(t)P_{kj}(s)}$
\end{lem}
\subsection{Chapman Kolmogorov Equation for CTMC}
\begin{thm}
\textbf{Kolmogorov Backward equation:} For all $i,j,t \geq 0$, $P'_{ij}(t)= \sum_{k \neq i}q_{ik}P_{kj}(t)-\nu_iP_{ij}(t)$ ,  $\frac{dP(t)}{dt}=QP(t)$
\end{thm}
\begin{proof}
$P_{ij}(t+h)=\sum_kP_{ik}(h)P_{kj}(t).$
\begin{flalign*}
P_{ij}(t+h)-P_{ij}(t)=\sum_{k \neq 1}P_{ik}(h)P_{kj}(t)-(1-P_{ii}(h))P_{ij}(t), 
\end{flalign*}
divide by $h$, $h \rightarrow 0$, we get $\frac{dP_{ij}(t)}{dt}=\lim_{h \rightarrow 0}P'_{ij}(t)= \sum_{k \neq i}q_{ik}P_{kj}(t)-\nu_iP_{ij}(t)$. Now the exchange of limit and summation has to be justified. 
\begin{flalign*}
&\liminf_{h \rightarrow 0} \sum_{k \neq 1}\frac{P_{ik}(h)}{h}P_{kj}(t) \geq \liminf_{h \rightarrow 0}\sum_{k \neq 1}\frac{P_{ik}(h)}{h}P_{kj}(t), k <N\\
&= \sum_{k \neq 1}q_{jk}P_{kj}(t),~k <N.
\end{flalign*}
This is true for any finite $N$. Take supremum over all $N$. We get \\
$\liminf_{h \rightarrow 0} \sum_{k \neq 1}\frac{P_{ik}(h)}{h}P_{kj}(t) \geq \sum_{k \neq 1}q_{jk}P_{kj}(t)$. Suffices to show that $\limsup_{h \rightarrow 0} \sum_{k \neq 1}\frac{P_{ik}(h)}{h}P_{kj}(t) \leq \sum_{k \neq 1}q_{jk}P_{kj}(t)$. To that end, 
\begin{flalign*}
\text{LHS}& \leq \limsup_{h \rightarrow 0}[\sum_{k \neq 1, k<N}\frac{P_{ik}(h)}{h}P_{kj}(t)+\sum_{k \neq 1, k\geq N}\frac{P_{ik}(h)}{h} ]\\
& = \limsup_{h \rightarrow 0}[\sum_{k \neq 1, k<N}\frac{P_{ik}(h)}{h}P_{kj}(t)+\frac{1-P_{ii}(h)}{h} \sum_{k \neq 1, k\geq N}\frac{P_{ik}(h)}{h} ]\\
& = [\sum_{k \neq 1, k<N}q_{ik}P_{kj}(t)+\nu_i- \sum_{k \neq 1, k\geq N}q_{ik} ]\\
&=\sum_{k \neq 1}q_{ik}P_{kj}(t). 
\end{flalign*}
\end{proof}
\begin{thm}
\textbf{Kolmogorov Forward Equation:} Under suitablle reqularity conditions, $P'_{ij}(t)=\sum_{k \neq i}P_{ik}(t)q_{kj}-P_{ij}(t)\nu_i$, i.e. $\frac{dP(t)}{dt}=P(t)Q$.
\end{thm}
\end{document}
