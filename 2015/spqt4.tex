\documentclass[a4paper,10pt]{article}
\usepackage{amsmath,amssymb}
\usepackage[colorlinks,urlcolor=blue]{hyperref}

\title{SPQT- Asst 4 (CTMC)}
\author{Prof. Parimal Parag}

\begin{document}
\maketitle
\begin{enumerate}
	\item Consider a population where each individual gives birth at an exponential rate $\lambda$ and dies at an exponential rate $\mu$. Moreover, new members enter the population as a Poisson Process with rate $\theta$. Let $X(t)$ denote the population at time $t$. 
	\begin{enumerate}
		\item Show that $\{X(t)\}$ is a CTMC.
		\item Find the generator matrix of $X(t)$.
		\item Find conditions for it to have a stationary distribution. Under those conditions, find the stationary distribution.
		\item Find $E[X(t)|X(0)=i]$.
	\end{enumerate}
	
	\item Let $A$ be a subset of the state space of CTMC $X(t)$. Let $T_i(t)$ be the amount of time spent in $A$ during $[0,t]$ given that $X(0) = i$. Let $Y_1,Y_2,\cdots Y_n$ be iid $exp(\lambda)$ independent of $X(t)$. Let $t_i(n) = E[T_i(Y_1+Y_2+ \cdots + Y_n)]$.
	\begin{enumerate}
		\item Derive a set of linear equations for $t_i(1)$, $i\geq 0$.
		\item Derive a set of linear equations for $t_i(n)$, in terms of $t_j(1)$ and $t_i(n-1)$.
		\item When n is large, for what value of $\lambda$ is $t_i(n)$ a good approximation for $E[T_i(t)]$?
	\end{enumerate}
	
	\item Consider a CTMC $X(t)$ with stationary distribution $\pi$ and generator matrix $Q$. 
	\begin{enumerate}
		\item Compute the probability that its sojourn time in state $i$ is greater than $\alpha > 0$.
		\item Consider its jump chain $\{Y_n\}$. Find its TPM $P$. Find the first time $\{Y_n\}$ comes back to state $i$ if $X(0) = i$.
		\item Fix $\alpha > 0$. Use (a) and (b) to find $E[T|X(0) = i]$ where $T$ is the first time $X(t)$ has its sojourn time in state $i$ greater than $\alpha$.
	\end{enumerate}
	
	\item Consider an $M/M/1/N$ queue where $N=2$. Arrival rate is $3/$hr and successive service times are iid $exp(4)$.
	\begin{enumerate}
		\item Find the generator matrix of $\{q(t)\}$.
		\item Find proportion of customers that enter the queue.
		\item If service rate is increased to 8, compute (b).
		\item Find conditions for stationary distribution of $q(t)$.
		\item Compute mean queue length and mean delay of customer entering the queue.
		\end{enumerate}
		
   \item If $\{X(t)\}$ and $\{Y(t)\}$ are independent reversible CTMC, show that so is $\{(X(t),Y(t))\}$.
	
	\item $N$ customers move around $r$ servers in the following manner. Customers that exit server $i$, enter server $i+1$ where $i=1,2,\cdots r-1$. Exit of server $r$ is connected to server $1$. Basically they form a circle. Service times at server $i$ are iid $exp(\mu_i)$. Consider the process $\{(q_1(t), q_2(t),\cdots, q_r(t))\}$.
 Show that the process is reversible and find the stationary distribution. If instead of the above circular arrangement, it joined $j$ with probability $\frac{1}{r-1}$, is it reversible?
 
\item Consider an $M/M/\infty$ system with arrival rate $\lambda$ and service rate $\mu$. 
\begin{enumerate}
	\item Let $q(t)$ be the number of customers in the system at time t. Find the generator matrix. Also find conditions of stationarity and stationary distribution.  
	\item Now consider the following system. Whenever an arrival comes, it will join the lowest numbered server. That is, if server 1 is free, it joins that, if not then the next and so on. 
	\begin{enumerate}
		\item Find the fraction of time server 1 is busy.
		\item By considering $M/M/2$ loss system, find the fraction of time server 2 is busy.
		\item Given $c$, find the fraction of time server $c$ is busy.
		\item What is the overflow rate from server $c$ to $c+1$? Is it a renewal process? Is it Poisson? Justify accordingly.
	\end{enumerate}
\end{enumerate}

\item Consider an ergodic CTMC $\{X(t)\}$ with generator matrix Q. Let $\pi$ be its stationary distribution. Let $B \subset E$ where $E$ is the state space. Let $G=B^c$. Compute the following under stationarity
\begin{enumerate}
	\item Compute $P[X(t) = i|X(t) \in B]$ for $i \in B$.
	\item Compute $P[X(t) = i|X(t) \in B, X(t^-) \in G]$ for $i \in B$.
	\item Show that 
	\[ \sum_{i \in G}\sum_{j \in B} \pi_i q_{ij} = \sum_{i \in B}\sum_{j \in G} \pi_i q_{ij}\]
\end{enumerate}
	
\end{enumerate}

Mail any typos to \url{gautamshenoydmc@gmail.com}.
\end{document}