\documentclass[a4paper,10pt]{article}
\usepackage{amsmath,amssymb}
\usepackage[colorlinks,urlcolor=blue]{hyperref}

\title{SPQT- Asst 5 (Martingales)}
\author{Prof. Parimal Parag}

\begin{document}
\maketitle
\begin{enumerate}
	\item Let $\{X_n\}_{n \geq 1}$ be iid zero mean random variables. Let $Y_1 = X_1$ and
	\[Y_{n+1} = Y_n + X_{n+1}f_n(Y_1,Y_2,\cdots,Y_n)\]
	Here $f_n$ is any real valued deterministic function. Moreover assume $Y_i$ are integrable. Show that $Y_i$ is a martingale.
	
	\item Let $X_n$ be a zero mean martingale with $E[X_n^2] < \infty$ for every $n$. Then
	\begin{enumerate}
		\item Show that 
		\[E[(X_{n+r} -X_n)^2] = \sum_{k=1}^r E[(X_{n+k} - X_{n+k-1})^2] \]
		\item Assume $\sum_{n \in \mathbb{N}} E[(X_n -X_{n-1})^2] < \infty$. Then prove that $X_n$ converges w.p.$1$.
	\end{enumerate}
	
	\item Let $X_n$ be a martingale. Show that if it is bounded above or below, then $\sup_{n\geq 1} E[|X_n|] < \infty$.
	
	\item (Simple Random Walk) Let $\{Y_n\}$ be iid with $P[Y_i = 1] = p =1-q = 1-P[Y_i = -1]$. Let $S_0 = 0$ and $S_n = \sum_{k=1}^n Y_k$. Define $T$ as 
	\[T = \inf \{ n \geq 1: S_n = -a \mbox{ OR } S_n = b\}\]
	where $a$ and $b$ are nonnegative integers. Essentially $T$ is the first time the random walk $S_n$ hits either $-a$ or $b$. Prove that when $p \neq q$
	\[E[T] = \frac{b}{p-q} -\frac{(a+b)}{(p-q)}\frac{\left(1-(\frac{p}{q})^b\right)}{\left(1-(\frac{p}{q})^{a+b}\right)}\]
	
	\item (Hardy type inequality) Let $X_n$ be a martingale. Suppose for some $\alpha > 1$, $E[|X_n|^\alpha] < \infty$, then prove that
	\[E[\max_{0\leq k\leq n} |X_k|] \leq \frac{\alpha}{\alpha - 1} E[|X_n|^\alpha]^{\frac{1}{\alpha}}\]
	
	\item (Doob's Decomposition Theorem) Let $X_n$ be a submartingale. Show that it can be written as $X_n = Y_n + Z_n$ where $Y_n$ is a martingale and $Z_n$ is a nondecreasing sequence of random variables.
	
	\item Let $f$ be a convex function and $X_n$ be a martingale. If $f(X_n)$ is integrable for every $n$, prove that $f(X_n)$ is a submartingale. Suppose instead $X_n$ were a submartingale, what additional condition (if there is any) should be imposed on $f$ so that $f(X_n)$ becomes a submartingale?
	
	\item There are $n$ coins, each independent with probability of ith coin coming heads $= p_i$. After each coin is tossed and outcomes noted, let $X$ be the number of heads. Prove that
	\[P\left\{\left| X - \sum_{i=1}^np_i \right| \geq a\right\} \leq 2e^{-2a^2/n}\]
\end{enumerate}
Mail any typos to \url{gautamshenoydmc@gmail.com}.

\end{document}