% !TEX spellcheck = en_US
% !TEX spellcheck = LaTeX
\documentclass[a4paper,10pt,english]{article}
\usepackage{%
	amsfonts,%
	amsmath,%	
	etex,%
	amssymb,%
	amsthm,%
	babel,%
	bbm,%
	%biblatex,%
	caption,%
	centernot,%
	color,%
	enumerate,%
	epsfig,%
	epstopdf,%
	geometry,%
	graphicx,%
	hyperref,%
	latexsym,%
	mathtools,%
	multicol,%
	pgf,%
	pgfplots,%
	pgfplotstable,%
	pgfpages,%
	proof,%
	psfrag,%
	subfigure,%	
	tikz,%
	ulem,%
	url%
}	

\usepackage[mathscr]{eucal}
\usepgflibrary{shapes}
\usetikzlibrary{%
  arrows,%
	backgrounds,%
	chains,%
	decorations.pathmorphing,% /pgf/decoration/random steps | erste Graphik
	decorations.text,%
	matrix,%
  	positioning,% wg. " of "
  	fit,%
	patterns,%
  	petri,%
	plotmarks,%
  	scopes,%
	shadows,%
  	shapes.misc,% wg. rounded rectangle
  	shapes.arrows,%
	shapes.callouts,%
  	shapes%
}

\theoremstyle{plain}
\newtheorem{thm}{Theorem}[section]
\newtheorem{lem}[thm]{Lemma}
\newtheorem{prop}[thm]{Proposition}
\newtheorem{cor}[thm]{Corollary}

\theoremstyle{definition}
\newtheorem{defn}[thm]{Definition}
\newtheorem{conj}[thm]{Conjecture}
\newtheorem{exmp}[thm]{Example}
\newtheorem{assum}[thm]{Assumptions}
\newtheorem{axiom}[thm]{Axiom}

\theoremstyle{remark}
\newtheorem{rem}{Remark}
\newtheorem{note}{Note}

\newcommand{\norm}[1]{\left\lVert#1\right\rVert}
\newcommand{\indep}{\!\perp\!\!\!\perp}
\DeclarePairedDelimiter\abs{\lvert}{\rvert}%
%\DeclarePairedDelimiter\norm{\lVert}{\rVert}%
\newcommand{\tr}{\operatorname{tr}}
\newcommand{\R}{\mathbb{R}}
\newcommand{\Q}{\mathbb{Q}}
\newcommand{\N}{\mathbb{N}}
\newcommand{\E}{\mathbb{E}}
\newcommand{\Z}{\mathbb{Z}}
\newcommand{\B}{\mathscr{B}}
\newcommand{\C}{\mathcal{C}}
\newcommand{\T}{\mathscr{T}}
\newcommand{\F}{\mathcal{F}}
\newcommand{\G}{\mathcal{G}}
%\newcommand{\ba}{\begin{align*}}
%\newcommand{\ea}{\end{align*}}

\makeatletter
\def\th@plain{%
  \thm@notefont{}% same as heading font
  \itshape % body font
}
\def\th@definition{%
  \thm@notefont{}% same as heading font
  \normalfont % body font
}
\makeatother
\date{}
\title{Lecture 09: Equilibrium Renewal Processes and Renewal Reward Processes}
\author{}

\begin{document}
\maketitle
\section{Renewal theory Contd. -- Delayed Renewal processes }

\subsection{Example:}

\begin{color}{blue} {\bf (Optional -- not covered in class)} \end{color}

Consider two coins and suppose  that each time is coin flipped, it lands tail with some unknown probability $p_i,~i=1,2.$ We are interested in coming up with a strategy that ensures that long term proportion of tails is $\min\{p_1,~p_2\}.$ One strategy is as follows: Set $n = 1$. In the $n^\text{th}$ round of coin flipping, flip the first coin till $n$ consecutive tails are obtained. Then flip the second coin till $n$ consecutive tails are obtained. Increment $n$ and repeat. \\

{\bf Claim.} $\lim_{m \to \infty} \frac{\# \mbox{tails in the first
    $m$ tosses}}{m} = \min\{p_1, p_2\}$ with probability $1$.\\

The proof is as follows. Let $p=\max\{p_1,p_2\}$ and $\alpha p
=\min\{p_1,p_2\}$. There is nothing to prove if $\alpha = 1$, so let
$\alpha < 1$. Call the coin with $P(T)=p$, the bad coin and the other,
the good coin. Let $B_n$ denote the number of flips in the
$n^\text{th}$ round of tossing the bad coin, and $G_n$ the number of
flips in the $n^\text{th}$ round of tossing the good coin. We first
prove the following lemma.
 \begin{lem}
   For any $\epsilon > 0$ with $\epsilon^{-1} \in \N$,
 $P(B_n \geq \epsilon G_n ~\text{for infinitely many rounds}~ n)=0$.
 \end{lem} 
 \begin{proof}
   For any $n \in \N$,
 \begin{flalign*}
   P\left(G_n \leq  \frac{B_n}{\epsilon}\right)&=\E[P(G_n \leq \frac{B_n}{\epsilon}|B_n)]\\
   &=\E[ \sum_{i=1}^{\frac{B_n}{\epsilon}}P(G_n = i |B_n)]\\
   &\leq\E[ \sum_{i=1}^{\frac{B_n}{\epsilon}} (\alpha p)^n]\\
   &= \E[{\frac{B_n}{\epsilon}}](\alpha p)^n \\
   &=\epsilon^{-1} \left( \sum_{i=1}^n \frac{1}{p^{i}}\right) (\alpha
   p)^n = \epsilon^{-1} \frac{p^{-n} - 1}{1 - p} (\alpha p)^n,
 \end{flalign*}
 where the inequality follows from the fact that $\{G_m = i\}$ implies
 that $i \geq m$ and that in cycle $m$, the coin flips numbered
 $i-m+1$ to $i$ are all tails. Hence,
 \[ \sum_{n=1}^\infty P\left(G_n \leq \frac{B_n}{\epsilon}\right) \leq
 \epsilon^{-1} \sum_{n=1}^\infty \frac{\alpha^{n}}{1 - p} < \infty.\]
 By the Borel-Cantelli lemma, it
 follows that $P(B_n \geq \epsilon G_n \text{for infinitely many
   $n$}) = 0$.
 \end{proof}
 With probability $1$, all but a finite number of rounds have at most
 an $\epsilon$ fraction of bad coin tosses, implying that $\lim_{m \to
   \infty} \frac{\# \mbox{bad coin tosses in the first $m$ tosses}}{m}
 \leq \epsilon$. Now taking a decreasing sequence $\epsilon_k = 1/k$,
 $k = 1, 2, 3, \ldots$, and using the continuity of probability, we
 get that with probability $1$, $\lim_{m \to \infty} \frac{\#
   \mbox{bad coin tosses in the first $m$ tosses}}{m} = 0$. This
 proves the claim using the strong law of large numbers for tosses of
 the good coin.

 \subsection{Distribution of the Last Renewal Time for a Delayed Renewal Process}
 In the same manner as we derived the key lemma, refer Theorem 1.9 in lecture 6, for the last renewal time distribution of a standard renewal process, we can show for a delayed renewal process:
 \begin{flalign*}
 P(S_{N(t)} \leq s)&=G^c(t) P(S_{N(t)} \leq s|S_{N(t)} = 0) + \int_{0}^{t}P(S_{N(t)} \leq s|S_{N(t)}=u)F^c(t-u)dm(u)\\
 &= G^c(t) +\int_{0}^{s}F^c(t-u)dm(u).
 \end{flalign*}
 Let $F_e(x)=\frac{\int_{0}^{x}F^c(y)dy}{\mu},~ x \geq 0$, known as
 the \textit{equilibrium distribution} of $F$. Observe that the moment generating function of $F_e(x)$ is $\tilde{F}_e(s) =
 \frac{1-\tilde{F}(s)}{s\mu}$.
 \begin{proof}
 By definition, $\tilde{F}_e(s) = \E\left[\mathrm{e}^{-sX}\right]$, where $ X $ is a random variable with probability distribution function $ F_e(x) $. So,
 \begin{flalign*}
 \tilde{F}_e(s) &= \int_{0}^{\infty}\mathrm{e}^{-sx}dF_e(x)\\
 &= \frac{1}{\mu}\int_{0}^{\infty}\mathrm{e}^{-sx}F^c(x)dx\\
 &= \frac{1}{s\mu} - \frac{1}{\mu}\int_{0}^{\infty}\mathrm{e}^{-sx}F(x)dx\\
 &= \frac{1}{s\mu} - \frac{1}{s\mu}\int_{0}^{\infty}\mathrm{e}^{-sx}dF(x)\\
 &= \frac{1}{s\mu} - \frac{1}{s\mu}\tilde{F}(s),
 \end{flalign*}
 where the third and fourth equalities follows from the basic integration techniques.
 \end{proof}
 And also observe that $F_e$ is the limiting
 distribution of the age and the excess time for the renewal process
 governed by $F$. If $G=F_e$, then the delayed renewal process is
 called the \textit{equilibrium renewal process}. Suppose we start
 observing a renewal process at some arbitrary time $t$. Then, the
 observed renewal process is the equilibrium renewal process. Let
 $Y_e(t)$ denote the excess time for the (delayed) equilibrium renewal
 process.
 \begin{thm}
 For the equilibrium renewal process,
 \begin{enumerate}
 \item $m_e(t) =\frac{t}{\mu}$.\\
 \item $P(Y_e(t) \leq x) = F_e(x)$.\\
 \item $\{N_e(t), t \geq 0\}$ has stationary increments.
  \end{enumerate}
\begin{proof}
To prove $ (1) $, observe that $\tilde{m}_e(s)=\frac{\tilde{G}(s)}{1-\tilde{F}(s)} = \frac{\tilde{F_e}(s)}{1-\tilde{F}(s)} = \frac{1}{s\mu}$. Hence, if $m_e(t) = \frac{t}{\mu}$ then, 
\begin{flalign*}
\tilde{m}_e(s) &= \int_{0}^{\infty}\mathrm{e}^{-st}dm_e(t)\\ 
&= \frac{1}{\mu} \int_{0}^{\infty}\mathrm{e}^{-st}dt\\
&= \frac{1}{s\mu}.
\end{flalign*}	
Since moment generating function is a one-to-one map, $m_e(t) = \frac{t}{\mu}$ is unique. \\
$ (2) $ 
\begin{flalign*}
P(Y_e(t) >x) &= P(Y_e(t) >x|S_{N_e(t)}=0)P(S_{N_e(t)}=0) +P(Y_e(t) >x|S_{N_e(t)}=s)F^c(t-s)\frac{ds}{\mu}\\
 &= P(X_1 >t+x,X_1>t) +P(X_2 >t+x-s|X_2 >t-s)F^c(t-s)\frac{ds}{\mu}\\
&= {F_e}^c(t+x)+\int_{0}^{t}F^c(t+x-s)\frac{ds}{\mu}\\
&= 1 - \frac{1}{\mu} \int_{0}^{t+x}F^c(y)dy - \frac{1}{\mu} \int_{t+x}^{x}F^c(y)dy\\
&= 1 - \frac{1}{\mu} \int_{0}^{x}F^c(y)dy\\
&= F_e^c(x).
\end{flalign*}
$ (3) $ $N_e(t+s)-N_e(s) =$ Number of renewals in time interval of length $t$. When we start observing at $s$, the observed renewal process is delayed renewal process with initial distribution being the original distribution.
\end{proof}
\end{thm}
\emph{Question}: What can you say about the equilibrium renewal process when $ F $ is distributed exponentially with the parameter $ \lambda $?
\\
\emph{Answer}: Let's look at the distribution of the first inter-arrival distribution, $ F_e $. So,
\begin{flalign*}
  F_e(x) &= \frac{1}{\mu} \int_{0}^{x}F^c(y)dy \\
  &= \lambda \int_{0}^{x} \mathrm{e}^{-y\lambda} dy \\
  &= 1 - \mathrm{e}^{-x\lambda},
\end{flalign*}	
where the first equality follows from the definition of $ F_e $ for equilibrium renewal process, the second equality follows from the fact that the mean of exponential distribution is inverse of the parameter $ \lambda $.\\
Thus even $ F_e $ is distributed exponentially with the parameter $ \lambda $. So with all the properties of equilibrium renewal process, $ F_e $ and $ F $ being distributed exponentially with the same parameter $ \lambda $, says that this is a poisson process (not a delayed renewal process).

\subsection{Renewal Reward Process}
\textbf{Definition:} Consider a renewal process $\{N(t), t \geq 0\}$ with inter arrival times $\{X_n: n \in \N\}$ having distribution $F$ and rewards $\{R_n: n \in \N\}$ where $R_n$ is the reward at the end of $X_n$. Let $(X_n,R_n)$ be \emph{iid}. Then $R(t)=\sum_{i=1}^{N(t)}R_i$ is reward process (total reward earned by time $t$). 
\begin{thm}
	\label{theorem}
Let $\E[|R|]$ and $\E[|X|]$ be finite.
\begin{enumerate}
\item $\lim_{t \rightarrow \infty} \frac{R(t)}{t} = \frac{\E[R]}{\E[X]} ~a.s.$
\item  $\lim_{t \rightarrow \infty} \frac{\E[R(t)]}{t} = \frac{\E[R]}{\E[X]}$.
\end{enumerate}
\end{thm}

\begin{proof}
$ (1) $ Write
\begin{flalign*}
\frac{R(t)}{t}&=\frac{\sum_{i=1}^{N(t)}R_i}{t}\\
&=\left(\frac{\sum_{i=1}^{N(t)}R_i}{N(t)} \right) \left(\frac{N(t)}{t}\right).
\end{flalign*}
By the strong law of large numbers (almost sure convergence law) we obtain that, 
\begin{flalign*}
	\lim_{t \rightarrow \infty} \frac{\sum_{i=1}^{N(t)}R_i}{N(t)} = \E[R],
\end{flalign*}
and by the basic renewal theorem (almost sure convergence law) we obtain that, 
\begin{flalign*}
	\lim_{t \rightarrow \infty} \frac{N(t)}{t} = \frac{1}{\E[X]}.
\end{flalign*} 
Thus $ (1) $ is proven.
\\
$ (2) $ \\
Notice that $ N(t)+1 $ is a stopping time for the sequence \{$ R_1,R_2,\dots $\}. This is true since
\begin{flalign*}
	\{N(t)+1 = n\} &= \{X_1+X_2+\cdots+X_{n-1} \leq t , X_n > t \}\\
	&= \{R_1+R_2+\cdots+R_{n-1} = R(t) , R_n \neq 0 \}.
\end{flalign*}
Moreover $ N(t)+1 $ is a stopping time for the sequence \{$ X_1,X_2,\dots $\}. So by algebra and Wald's lemma,
\begin{flalign*}
\E[R(t)] &= \E\left[\sum_{i=1}^{N(t)}R_i\right] \\
&= \E\left[\sum_{i=1}^{N(t)+1}R_i\right]-\E[R_{N(t)+1}]\\ 
&= (m(t)+1)\E[R_1]-\E[R_{N(t)+1}].
\end{flalign*}
Let $g(t)=\E[R_{N(t)+1}].$ So 
\begin{equation}
 \frac{\E[R(t)]}{t} = \frac{(m(t)+1)}{t}\E[R_1]-\frac{g(t)}{t}, \nonumber
\end{equation}
and the result will follow from the elementary renewal theorem if we can show that $ \frac{g(t)}{t} \rightarrow 0 $ as $ t \rightarrow \infty. $ So,
\begin{flalign*}
g(t) &= \E[R_{N(t)+1}1\{S_{N(t)}=0\}]+\E[R_{N(t)+1}1\{S_{N(t)}>0\}]\\
&=\E[R_{N(t)+1}|S_{N(t)}=0]P(X_1>t)+\int_{0}^{t}\E[R_{N(t)+1}|S_{N(t)}=u]F^c(t-u)dm(u),
\end{flalign*}
where the second equality follows from the fact that the interarrival times $ X_n, n \in \N $, are \emph{iid} with distribution $ F $. \\ 
However,
\begin{flalign*}
	\E[R_{N(t)+1}|S_{N(t)}=0] &= \E[R_1|X_1>t], \\
	\E[R_{N(t)+1}|S_{N(t)}=u] &= \E[R_n|X_1>t-u], 
\end{flalign*}
and so
\begin{flalign*}
g(t) &= \E[R_1|X_1>t]F^c(t)+\int_{0}^{t}\E[R_n|X_1>t-u]F^c(t-u)dm(u)\\
&= \E[R_1|X_1>t]F^c(t)+\int_{0}^{t}\E[R_1|X_1>t-u]F^c(t-u)dm(u),
\end{flalign*}
where the second equality follows from the fact that $ R_n, n \in \N $, are \emph{iid}. \\
Now, let
\begin{equation}
h(t)=\E[R_1|X_1>t]F^c(t) = \int_{x=t}^{\infty} \E[R_1|X_1=x]dF(x), \nonumber
\end{equation}
and note that since
\begin{equation}
\E[|R_1|] = \int_{x=0}^{\infty} \E[|R_1||X_1=x]dF(x) < \infty, \nonumber
\end{equation}
it follows that $h(t) \rightarrow 0$ as $t \rightarrow \infty.$ Hence, choosing $T$ such that $|h(u)| \leq \epsilon$ whenever $ u \geq T$, we have for all $t \geq T$ that
\begin{flalign*}
\frac{|g(t)|}{t} &\leq \frac{|h(t)|}{t} +\int_{0}^{t-T}\frac{|h(t-s)|}{t}dm(s)+\int_{t-T}^{t}\frac{|h(t-s)|}{t}dm(s)\\
&\leq \frac{\epsilon}{t}+ \frac{\epsilon m(t-T)}{t}+ \E[|R_1|]\frac{(m(t)-m(t-T))}{t} .
\end{flalign*}
Hence $\lim_{t \rightarrow \infty}\frac{g(t)}{t}= \frac{\epsilon}{\E[X]}$ by the
elementary renewal theorem, and the result follows since $\epsilon >
0$ is arbitrary.
 \end{proof}

\begin{rem}
	 	$ (1) ~$ $R_{N(t)+1}$ has different distribution than $R_1$.\\
	 	\emph{Analysis:} Notice that $R_{N(t)+1}$ is related to $X_{N(t)+1}$ which is the length of the renewal interval containing the point $t$. Since larger renewal intervals have a greater chance of containing $t$, it follows that $X_{N(t)+1}$ tends to be larger than a ordinary renewal interval. Formally,
	 	\begin{flalign*}
	 		\Pr\{X_{N(t)+1} > x\}&= \sum_{n \in \N_0} \left( \left[ \int_{0}^t\Pr\{X_{N(t)+1} > x | S_{N(t)} = y, N(t)=n\}F^c(t-y)dm(y) \right] \Pr\{N(t)=n\} \right).
	 	\end{flalign*}
	 	Now we have,
	 	\begin{flalign*}
	 		\Pr\{X_{N(t)+1}>x | S_{N(t)}=y, N(t)=n\} & = \Pr\{X_{N(t)+1}>x | X_1+\cdots+X_n=y, X_{n+1}>t-y\} \\
	 		& = \Pr\{X_{n+1}>x | X_{n+1}>t-y\} \\
	 		& = \frac{\Pr\{X_{n+1}>\text{max}(x,t-y)\}}{\Pr\{X_{n+1}>t-y\}} \\
	 		& \geq F^c(x). 
	 	\end{flalign*}
	 	So we get that,
	 	\begin{flalign*}
	 		\Pr\{X_{N(t)+1}>x\}\geq \Pr\{X_1>x\}.
	 	\end{flalign*}
	 	Thus the remark follows.\\
	 	$ (2) ~$ $R(t)$ is the gradual reward during a cycle, 
	 	\begin{flalign*}
	 		\frac{\sum_{n=1}^{N(t)}R_n}{t} \leq  \frac{R(t)}{t} \leq \frac{\sum_{n=1}^{N(t)+1}R_n}{t}.
	 	\end{flalign*}
	 	\emph{Analysis:} The part 1 of the theorem \ref{theorem} under this regime follows since
	 	\begin{flalign*}
	 		\lim_{t \rightarrow \infty} \frac{\sum_{n=1}^{N(t)}R_n}{t} = \frac{\E\left[R\right]}{\E\left[X\right]},\\
	 		\lim_{t \rightarrow \infty} \frac{\sum_{n=1}^{N(t)+1}R_n}{t} = \frac{\E\left[R\right]}{\E\left[X\right]},
	 	\end{flalign*}
	 	by the similar arguments given in the proof of the theorem \ref{theorem}.\\
	 	The part 2 of the theorem \ref{theorem} under this regime follows since
	 	\begin{flalign*}
	 		\lim_{t \rightarrow \infty} \frac{\E\left[R_{N(t)+1}\right]}{t} = 0,
	 	\end{flalign*}
	 	by the similar arguments given in the proof of the theorem \ref{theorem}. Thus the remark follows. For more insights refer Chapter 3 in \textit{Stochastic Processes} by \textit{Sheldon M. Ross}.
\end{rem}
 
 \subsubsection{Example:} Suppose for an alternating renewal process, we earn at a rate of one per unit time  when the system is on and the reward for a cycle is the the time system is ON during that cycle.\\
 $ \lim_{t \rightarrow \infty} \frac{\text{Amount of ON time in } [0,t]}{t} = \lim_{t \rightarrow \infty} \frac{R(t)}{t}=\frac{\E[X]}{\E[X]+\E[Y]} = \lim_{t \rightarrow \infty}P(\text{ON at time t})$. 
\end{document}