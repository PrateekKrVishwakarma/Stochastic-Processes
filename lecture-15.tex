% !TEX spellcheck = en_US
% !TEX spellcheck = LaTeX
\documentclass[a4paper,10pt,english]{article}
\usepackage{%
	amsfonts,%
	amsmath,%	
	etex,%
	amssymb,%
	amsthm,%
	babel,%
	bbm,%
	%biblatex,%
	caption,%
	centernot,%
	color,%
	enumerate,%
	epsfig,%
	epstopdf,%
	geometry,%
	graphicx,%
	hyperref,%
	latexsym,%
	mathtools,%
	multicol,%
	pgf,%
	pgfplots,%
	pgfplotstable,%
	pgfpages,%
	proof,%
	psfrag,%
	subfigure,%	
	tikz,%
	ulem,%
	url%
}	

\usepackage[mathscr]{eucal}
\usepgflibrary{shapes}
\usetikzlibrary{%
  arrows,%
	backgrounds,%
	chains,%
	decorations.pathmorphing,% /pgf/decoration/random steps | erste Graphik
	decorations.text,%
	matrix,%
  	positioning,% wg. " of "
  	fit,%
	patterns,%
  	petri,%
	plotmarks,%
  	scopes,%
	shadows,%
  	shapes.misc,% wg. rounded rectangle
  	shapes.arrows,%
	shapes.callouts,%
  	shapes%
}

\theoremstyle{plain}
\newtheorem{thm}{Theorem}[section]
\newtheorem{lem}[thm]{Lemma}
\newtheorem{prop}[thm]{Proposition}
\newtheorem{cor}[thm]{Corollary}

\theoremstyle{definition}
\newtheorem{defn}[thm]{Definition}
\newtheorem{conj}[thm]{Conjecture}
\newtheorem{exmp}[thm]{Example}
\newtheorem{assum}[thm]{Assumptions}
\newtheorem{axiom}[thm]{Axiom}

\theoremstyle{remark}
\newtheorem{rem}{Remark}
\newtheorem{note}{Note}

\newcommand{\norm}[1]{\left\lVert#1\right\rVert}
\newcommand{\indep}{\!\perp\!\!\!\perp}
\DeclarePairedDelimiter\abs{\lvert}{\rvert}%
%\DeclarePairedDelimiter\norm{\lVert}{\rVert}%
\newcommand{\tr}{\operatorname{tr}}
\newcommand{\R}{\mathbb{R}}
\newcommand{\Q}{\mathbb{Q}}
\newcommand{\N}{\mathbb{N}}
\newcommand{\E}{\mathbb{E}}
\newcommand{\Z}{\mathbb{Z}}
\newcommand{\B}{\mathscr{B}}
\newcommand{\C}{\mathcal{C}}
\newcommand{\T}{\mathscr{T}}
\newcommand{\F}{\mathcal{F}}
\newcommand{\G}{\mathcal{G}}
%\newcommand{\ba}{\begin{align*}}
%\newcommand{\ea}{\end{align*}}

\makeatletter
\def\th@plain{%
  \thm@notefont{}% same as heading font
  \itshape % body font
}
\def\th@definition{%
  \thm@notefont{}% same as heading font
  \normalfont % body font
}
\makeatother
\date{}
\title{Lecture 15 : Limiting Probabilities and Uniformization }%Time Reversibility of Discrete Time Markov Chains}
\author{}
\begin{document}
\maketitle

\section{Limiting Probabilities}
We denote by $\{S_n: n \in \N_0\}$ the jump times of CTMC and the probability transition of the embedded Markov chain is denoted by $P = \{p_{ij}: i \neq j \in I\}$. 
\begin{defn}
If the embedded Markov chain is irreducible and positive recurrent then the stationary distribution $\alpha$ of the embedded Markov chain is a unique solution to
\begin{align}
\label{eq:StationaryDistribution}
\alpha = \alpha P,~ \sum_{i}\alpha_i = 1.
\end{align}
\end{defn}
We denote the mean of inter-arrival time by 
\begin{align*}
\frac{1}{\nu_i} = \E[S_{n+1} - S_{n}|X(S_n) = i].
\end{align*} 
\begin{defn} First return time to a state $i$ for a stochastic process $\{X(t), t \geqslant 0\}$ is 
\begin{align*}
T_{ii} = \inf\{t > S_1 : X(0) = i, X(t) = i\}.
\end{align*}
\end{defn}
\begin{thm} For a CTMC with the irreducible and positive recurrent embedded Markov chain with the stationary distribution $\alpha$, following holds for the limiting process,
\begin{align}
\label{eq:LimitingProbability}
 \pi_j \triangleq \lim_{t \rightarrow \infty } P_{ij}(t)= \frac{\alpha_j / \nu_j}{\sum_j \alpha_j /\nu_j},
\end{align}
From~\eqref{eq:LimitingProbability} and~\eqref{eq:StationaryDistribution}, we see that $\{\pi_j\}$ is the unique non-negative solution to 
\begin{equation}
\nu_j\pi_j=\sum_i \nu_i\pi_iP_{ij},~ \sum_j \pi_j =1.
\end{equation}
\end{thm}
\begin{proof} 
For any CTMC on state space $I$, return to a state $i \in I$ is a renewal process. 
We can define an alternating renewal process with on and off times characterized by the time CTMC spends in state $j$ and otherwise respectively.
Therefore, by alternating renewal process theory, we have
\begin{align*}
\lim_{t \to \infty}\Pr\{ \text{CTMC is on at time } t\} = \lim_{t \to \infty}P_{ij}(t) = \frac{\E \tau_j }{\E T_{jj}}.
\end{align*} 

The time spent in state $j \in I$ during $k^\text{th}$ visit to this state is defined by
\begin{align*}
\tau_j(k) = \inf\{ t > 0: X() = j\}.
\end{align*}
Number of visits to state $j$ in the first $m$ transitions of the CTMC is defined by
\begin{align*}
N_j(m) = \sum_{l=1}^m1_{\{X(S_l) = j\}}.
\end{align*}
The proportion of time in state $j$ during the first $m$ transitions of the CTMC is 
\begin{align*}
\beta(m) &= \frac{\frac{N_j(m)}{m}\sum_{k=1}^{N_j(m)}\frac{\tau_j(k)}{N_j(m)}}{\sum_j \frac{N_j(m)}{m}\sum_{k=1}^{N_j(m)}\frac{\tau_j(k)}{N_j(m)}}.%,\\
%&= \frac{\frac{N_i(m)}{m}\sum_{j=1}^{N_i(m)}Y_i(j)}{\sum_l \frac{N_i(m)}{m} \sum_{j=1}^{N_i(m)}Y_i(j) }.\\
\end{align*}
Since $N_j(m)\rightarrow \infty$ as $m \rightarrow \infty$, it follows from the strong law of large numbers that 
\begin{xalignat*}{3}
&\lim_{m \in \N}\frac{\sum_{k=1}^{N_j(m)}\tau_j(k)}{N_j(m)} = \E \tau_j, && \lim_{m \in \N}\frac{N_j(m)}{m} = \alpha_j.
\end{xalignat*}
%and $ (E[\text{number of transitions between visits to state }i])^{-1}=\alpha_i$. Letting $m \rightarrow \infty$, result follows.
\end{proof}
\begin{rem}
Limiting probability $\pi_j$ is long-run proportion of time the process is in state $j$.
\end{rem}

%\begin{defn} We denote \textbf{stationary distribution} of an irreducible and positive recurrent CTMC $P(t)$ by $\pi$ such that for all $t > 0$
%\begin{align}
%\label{eq:StationaryCTMCDistribution}
%\pi = \pi P(t),~~ \sum_{i}\pi_i = 1.
%\end{align}
%\end{defn}
%
%Since a Continuous Time Markov Chain (CTMC) is a semi-Markov chain with $F_{ij}(t)=1-e^{\nu_i t}$. From the theory of semi-Markov process, if the embedded Markov chain with transition probabilities $P_{ij}$ is irreducible and positive recurrent, then the limiting process,
%\begin{align}
%\label{eq:LimitingProbability}
%\pi_j \triangleq \lim_{t \rightarrow \infty } P_{ij}(t)= \frac{\alpha_j / \nu_j}{\sum_j \alpha_j /\nu_j},
%\end{align}
%where $\alpha_i$ is the stationary distribution of the embedded Markov chain. From \ref{eq:LimitingProbability} and \ref{eq:StationaryDistribution}, we see that $\{\pi_j\}$ is the unique non-negative solution to 
%\begin{equation}
%\nu_j\pi_j=\sum_i \nu_i\pi_iP_{ij},~ \sum_j \pi_j =1.
%\end{equation}

\begin{lem}
For an irreducible CTMC with transition probability $P(t)$, if the initial state is chosen according to the limiting probability distribution $\pi$, then the resultant process will be stationary. That is,
\begin{equation*}
\pi P(t)=\pi,~ \text{for all}~ t > 0.
\end{equation*}
\end{lem}
\begin{proof}
\begin{flalign*}
\sum_{i}P_{ij}(t)\pi_i &=\sum_{i}P_{ij}(t)\lim_{s \rightarrow \infty}P_{ki}(s) =\lim_{s \rightarrow \infty} \sum_{i}P_{ij}(t)P_{ki}(s) =\lim_{s \rightarrow \infty}P_{kj}(t+s) =\pi_j.
\end{flalign*}
\end{proof} 

\begin{rem}
Another way of arriving at the limiting probabilities are by forward equations
\begin{equation}
P_{ij}'(t)=\sum_{k \neq j}q_{kj}P_{ik}(t)-\nu_iP_{ij}(t).
\end{equation}
Assume that the limiting probabilities exist. Then, it is easy to observe that $P_{ij}'(t) \rightarrow 0$ as $t \rightarrow \infty$. Letting $t \rightarrow \infty$, assuming that the limit and summation can be exchanged, we get the expression for $\pi_j$. 
\end{rem}
\begin{rem}
In any interval $(0,t)$, the number of transitions into state $j$ must equal to within 1 the number of transitions out of state $j$. Hence, in the long run, The rate at which transitions occur into state $j$ = The rate at which transitions occur out of state $j$. That is,
\begin{equation}
\nu_i \pi_i =\sum_{ i}\pi_i q_{ij}. 
\end{equation} 
 Hence,
 \begin{equation}
 \nu_iP_{ij}=\sum_{i}\pi_iq_{ij},~ \sum_j \pi_j =1.
 \end{equation}
 are called balance equations.
\end{rem}

\section{Uniformization}
Consider a continuous-time Markov chain in which the mean time spent in a state is the same for all states. That is, say $\nu_i=\nu$ for all states $i$. Let $N(t)$ denote the number of state transitions by time $t$. Since the amount of time spent in each state is  exponential $\nu$, $\{N(t), ~ t \geq 0 \}$ is a Poisson process with parameter $\nu$. To compute the transition probabilities $P_{ij}(t)$, we can condition on $N(t)$ as follows:
\begin{flalign*}
P_{ij}(t)&=Pr(X(t)=j|X(0)=i)\\
&=\sum_{n \in \mathbb{N}_0} Pr(X(t)=j,N(t)=n|X(0)=i)\\
&=\sum_{n \in \mathbb{N}_0} Pr( N(t)=n|X(0)=i) Pr(X(t)=j|X(0)=i,N(t)=n).
\end{flalign*}
Hence,
\begin{equation*}
P_{ij}(t)= \sum_{n \in \mathbb{N}_0} P_{ij}^{(n)}e^{-\nu t}\frac{(\nu t)^n}{n !}.
\end{equation*}
The above equation helps to compute $P_{ij}(t)$ approximately by computing an appropriate partial sum. But its application is limited as the rates are all assumed to be equal. But it so turns out that most Markov chains can be put in that form by allowing hypothetical transitions from a state to itself.
\subsection{Uniformization step}
Consider a CTMC with bounded $\nu_i$s. Choose $\nu$ such that 
\begin{equation}
\label{eq: UniformizationBound}
\nu_i \leq \nu,
\end{equation}
for all $i$. Since from each stage, the Markov chain leaves at rate $\nu_i$, we could equivalently assume that the transitions occur at a rate $\nu$ but only $\frac{\nu_i}{\nu}$ are real transitions and the remaining transitions are fictitious. Any Markov chain satisfying \ref{eq: UniformizationBound} can be thought of as being in a process that spends an exponential amount of time with rate $\nu$ in state $i$ and then makes a transition to state $j$ with probability $P_{ij}^*$, where
\begin{equation}
P_{ij}^* = \left\{
     \begin{array}{lr}
       1-\frac{\nu_i}{\nu} & : j =i\\
       \frac{\nu_i}{\nu}P_{ij} & : j \neq i.
     \end{array}
   \right.
\end{equation}   
The transition probabilities are computed by 
\begin{equation*}
P_{ij}(t)=\sum_{n=0}^{\infty}P_{ij}^{*n}e^{\nu}t \frac{{(\nu t)}^n}{n!}
\end{equation*}
This technique of uniformizing the rate in which a transition occurs from each state to state by introducing self transitions is called uniformization.
\subsection{Reversibility}
\begin{cor}
Consider an M/M/s queue with Poisson$(\lambda)$ arrivals and each server having exponential service time exp$(\mu)$ service. If $\lambda > s \mu$, then the output process in steady state is Poisson$(\lambda)$.
\end{cor}
\begin{proof}
Let $X(t)$ denote the number of customers in the system at time $t$. Since M/M/s process is a birth and death process, it follows from the previous proposition that $\{X(t),~t \geq 0\}$ is time reversible. Now going forward in time, the time instants at which $X(t)$ increases by 1 are the arrival instants of a Poisson process. Hence, by time reversibility, the time Points at which $X(t)$ increases by 1 when we go backwards in time also constitutes a Poisson process. But these instants are exactly the departure instants of the forward process. Hence the result.
\end{proof}
\begin{prop}
A time-reversible chain with limiting probabilities $\pi_j,~ j \in S$, that is truncated to the set $A\subset S$ and remains irreducible is also time reversible and has limiting probabilities 
\begin{equation}
\pi_j^A=\pi_j/(\sum_{i \in A}\pi_i),~ j \in A.
\end{equation}
\end{prop}
\begin{proof}
We must show that 
\begin{equation*}
\pi_i^Aq_{ij}=\pi_j^Aq_{ji},~ i \in A,~ j \in A,
\end{equation*}
or equivalently,
\begin{equation*}
\pi_iq_{ij}=\pi_j^Aq_{ji},~ i \in A,~ j \in A.
\end{equation*}
But this is true as the original chain is time reversible.
\end{proof}


\subsection{Time Reversibility of Discrete Time Markov Chains}

Consider a discrete time Markov chain with transition probability matrix $P$ and stationary probability vector $\alpha$.\\
\textbf{Claim:} The reversed process is a Markov chain.
\begin{proof}
\begin{flalign*}
&P(X_{m-1}=i|X_m=j,X_{m+1}=i_{m+1} \hdots )=\frac{P(X_{m-1}=i,X_m=j, \hdots )}{P(X_m=j, X_{m+1}=i_{m+1}\hdots )}\\
&=\frac{P(X_{m-1}=i,X_m=j)P(X_{m+1}=i_{m+1}\hdots |X_{m-1}=i,X_m=j  )}{P(X_m=j)P(X_{m+1}=i_{m+1} \hdots |X_m=j)}\\
&\stackrel{(a)}{=}\frac{P(X_{m-1}=i,X_m=j)P(X_{m+1}=i_{m+1}\hdots |X_m=j  )}{P(X_m=j)P(X_{m+1}=i_{m+1} \hdots |X_m=j)}\\
&=P(X_{m-1}=i|X_m=j).\\
\end{flalign*}
where $(a)$ follows from the Markov property.
\end{proof}
The sequence $\{X_n,X_{n-1} \hdots \}$ is called reverse process. Let $P^*$ denote the transition probability matrix. 
\begin{flalign*}
P_{ij}^*&=P(X_{n-1}=j|X_n=i)=\frac{P(X_{n-1}=j,X_n=i)}{P(X_n=i)}\\
&=\frac{P(X_{n-1}=j)P(X_n=i|X_{n-1}=j)}{P(X_{n}=i)}\\
&=\frac{P(X_{n-1}=j)}{P(X_{n}=i)}P_{ji}
\end{flalign*}
suppose we are considering a stationary Markov chain, $P(X_n=l)=P(X_{n-1}=l)=\alpha(l),~\forall l$, $\alpha(i){P^*}_{ij}=\alpha(j)(P)_{ji}$. If $P^*=P^T$ then the Markov chain is called time reversible. Thus the condition for time reversibility is given by $\alpha P_{ij}=\alpha_jP_{ji}$. Any non-negative vector $X$ satisfying $X_iP_{ij}=X_jP_{ji},~\forall i,j$ and $\sum_{j \in \mathcal{N}_0}X_j=1$ is stationary distribution of time-reversible Markov chain. This is true because,
\begin{flalign*}
\sum_{i} X_i P_{ij}=\sum_i X_j P_{ji}=X_j \sum X_i = 1.
\end{flalign*}
Since stationary probabilities are the unique solution of the above, the claim follows.\\

\textbf{Example 4.7(A) An Ergodic Random Walk}: Any ergodic, positive recurrent random walk is time reversible. The transition probability matrix is $P_{i,i+1}+P_{i-1,i}=1$. For every two transitions from $i+1$ to $i$, there must be one transition from $i$ to $i+1$. The rate of transitions from $i+1$ to $i$ must hence be same as the number of transitions from $i$ to $i+1$. So the process is time reversible.

\textbf{Example 4.7(B) The Metropolis Algorithm: } Let $a_j,~ j=1, \hdots m$ be positive numbers and let $A=\sum_{i=1}^{m}a_i$. Suppose that $m$ is large and $A$ is difficult to compute. One way to compute it by simulating a sequence of random variables is by generating a Markov chain whose limiting probabilities are $\pi_j$s. The Metropolis algorithm provides a convenient approach. \\
Let $Q$ be an irreducible transition probability matrix on the integers $1, \hdots n$ such that $q_{ij}=q_{ji}$ and for all $i$ and $j$. Generate a Markov chain $\{X_n\}$ such that the transition probabilities are given by 
\begin{flalign*}
P_{ij} = \left\{
     \begin{array}{lr}
       q_{ij}\min(1,a_j/a_i) & : j \neq i\\
       q_{ii}+\sum_{j \neq i}q_{ij}\{1-\min(1,a_j/a_i)\} & : j = i.
     \end{array}
   \right.
\end{flalign*} 
It can be directly verified that the chain is irreducible and hence verify that $\pi_j$s are the limiting probabilities.

\textbf{Edge weighted graphs:} Consider a graph having a positive number $w_{ij}$ associated with each edge $(i,j)$, and suppose that a particle moves from vertex to vertex in the following manner: If the particle is presently at vertex  $i$ then it will next move to vertex $j$ with probability
\begin{equation*}
P_{ij}=\frac{w_{ij}}{\sum_{j}w_{ij}}
\end{equation*}
where $w_{ij}$ is 0 if $(i,j)$ is not an edge of the graph. The Markov chain describing the sequence of vertices visited by the particle is called a random walk on an edge weighted graph. 
\begin{prop}
Consider a random walk on an edge weighted graph witha finite number of vertices. If this Markov chain is irreducible then it is, in steady state, time reversible with stationary probabilities given by 
\begin{equation*}
\alpha_i = \frac{\sum_{i}w_{ij}}{\sum_{j}\sum_{i}w_{ij}}.
\end{equation*}
\end{prop}
\begin{proof}
The time reversibility equation
\begin{equation*}
\alpha_iP_{ij}=\alpha_{j}P_{ji}
\end{equation*}
reduces to 
\begin{equation*}
 \frac{\alpha_i w_{ij}}{\sum_{k}w_{ik}}=\frac{\alpha_j w_{ji}}{\sum_{k}w_{jk}}
\end{equation*}
But noting that $w_{ij}=w_{ji}$ and $\sum_{i}\alpha_i = 1$, we get the desired result.
 
\end{proof}
\subsubsection*{Necessary condition for time reversibility}
If we try to prove the equations necessary for time reversibility, $X_iP_{ij}=X_jP_{jk}$ for all $i,j$, for any arbitrary Markov chain, one may not end up getting any solution. This is so because, if $P_{ij}P_{jk}>0$, then $\frac{X_i}{X_k}=\frac{P_{ji}P_{kj}}{P_{jk}P_{ij}} \neq \frac{P_{kj}}{P_{jk}}$.\\
Thus we see that a necessary condition for time reversibility is $P_{ij}P_{jk}P_{ki}=P_{ik}P_{kj}P_{ji},~ \forall i,j,k$. In fact we can show the following.
\begin{thm}
A stationary Markov chain is time reversible if and only if starting in state $i$
, any path back to state $i$ has the same probability as the reversed path, for all $i$. That is, if
\begin{flalign*}
P_{i i_1}P_{i_1 i_2}\hdots P_{i_k i}=P_{i,i_k}P_{i_k i_{k-1}} \hdots P_{i_1,i}.
\end{flalign*} 
\end{thm}
\begin{proof}
The proof of necessity is as indicated above. To see the sufficiency part, fix states $i,j$
\begin{flalign*}
&\sum_{i_1,i_2,\hdots i_{k}}P_{ii_1}\hdots P_{i_k,j}P_{j,i}=\sum_{i_1,i_2,\hdots i_{k}}P_{i,j}P_{j,i_k}\hdots P_{i_1 i}\\
&(P^k)_{ij}P_{ji}=P_{ij}(P^k)_{ji}\\
&\frac{\sum_{k=1}^{n}(P^k)_{ij}P_{ji}}{n}= \frac{\sum_{k=1}^{n}P_{ij}(P^k)_{ji}}{n}
\end{flalign*}
As limit $n \rightarrow \infty$, we get the desire result.
\end{proof}

\begin{thm}
Consider irreducible Markov chain with transition matrix $P$. If one can find non-negative vector $\alpha$ and other transition matrix $P^*$ such that $\sum_j \alpha_j =1$ and $\alpha_iP_{ij}=\alpha_jP^*_{ji}$ then $\alpha$ is the stationary probability vector and $P^*$ is the transition matrix for the reversed chain.
\end{thm}
\begin{proof}
Summing $\alpha_iP_{ij}=\alpha_jP_{ji}^*$ over $i$ gives, $\sum_{i}\alpha_iP_{ij}=\alpha_j$. Hence $\alpha_i$s are the stationary probabilities of the forward and reverse process. Since $P_{ji}^*=\frac{\alpha_iP_{ij}}{\alpha_j}$, $P_{ij}^*$ are the transition probabilities of the reverse chain.
\end{proof} 

\subsection{Example 4.7(E): Example 4.3(C) revisited}
Example 4.3(C) was with regard to the age of a renewal process. $X_n$ the forward process there was such that it increases in steps of 1 until it hits a value chosen by the inter arrival distribution. Hence the reverse process should be such that it decreases in steps of 1 until it hits 1 and then jumps to a state as chosen by the inter arrival distribution. Thus letting $\pi_i$ as the probability of inter arrival, it seems likely that  $P_{1i}*=\pi_i, ~ P_{i,i-1}=1,~ i > 1$. We have that $P_{i,1}=\frac{\pi_i}{\sum_{j \geq 1}\pi_j}=1-P_{i,i+1}, ~ i \geq 1$. For the reversed chain to be given as above, we would need 
\begin{flalign*}
&\alpha_i P_{ij}=\alpha_j P_{ji}^*\\
&\alpha_i \frac{\pi_i}{\sum_j \pi_j}=\alpha_1 \pi_i\\
&\alpha_i=\alpha_1 P(X \geq i)\\
&1=\sum_i \alpha_i=\alpha_1 E[X]; \alpha_i=\frac{P(X \geq i)}{E[X]}, 
\end{flalign*}
where $X$ is the inter arrival time. We need to verify that $\alpha_i P_{i,i+1}=\alpha_{i+1}P^*_{i+1,i}$ or equivalently $P(X \geq i)(1-\frac{\pi_i}{P(X \geq i)})=P(X \geq i)$ to complete the proof that the reversed process is the excess process and the limiting distributions are as given above. But that is immediate.
\subsection{Semi Markov Processes}
Consider a stochastic process with states $0,1,2 \hdots$ such that whenever it enters state $i$,
\begin{enumerate}
\item {The next state it enters is state $j$ with probability $P_{ij}$.}\\
\item {Given the next state is going to be $j$, the time until the next transition from state $i$ to state $j$ has distribution $F_{ij}$. If we denote the state at time $t$ to be $Z(t)$, $\{Z(t), t \geq 0\}$ is called a Semi Markov process.}
\end{enumerate}

Markov chain is a semi Markov process  with 

\begin{flalign*} 
  F_{ij}(t) = \left\{
     \begin{array}{lr}
       0 & : t \leq 1 \\
       1 & : t > 1.
     \end{array}
   \right.
\end{flalign*}

Let $H_i$ the distribution of time the semi Markov process stays in state $i$ before transition. We have $H_j(t)= \sum_i P_{ij}F_{ij}(t)$, $\mu_i = \int_0 ^ \infty X dH_i(x)$. Let $X_n$ denote the $n^{\text{th}}$ state visited. Then $\{X_n\}$ is a Markov chain with transition probability $P$ called the embedded Markov chain of the semi Markov process. \\
\textbf{Definition:} If the embedded Markov chain is irreducible, then Semi Markov process is said to be irreducible. Let $T_{ii}$ denote the time between successive transitions to state $i$. Let $\mu_{ii}=E[T_{ii}]$.
\begin{thm}
If semi Markov process ius irreducible and if $T_{ii}$ has non-lattice distribution with $\mu_{ii}< \infty$ then, 
\begin{flalign*}
\pi_i=\lim_{t \rightarrow \infty}P(Z(t)=i|Z(0)=j)
\end{flalign*}
exists and is independent of the initial state. Furthermore, $\pi_i=\frac{\mu_i}{\mu_{ii}}$.
\end{thm}
\textbf{Corollary 4.8.2} If the semi-Markov process is irreducible and $\mu_{ii}<\infty$, then with probability 1,
$\frac{\mu_i}{\mu_{ii}}=\frac{\lim_{t \rightarrow \infty} \text{Amount of time in [0,t]}}{t}~\text{a.s}$.\\
\begin{thm}
Suppose conditions of the previous thmrem hold and the embedded Markov chain is positive recurrent. Then $\pi_i= \frac{\alpha_i\mu_i}{\sum_{i}\alpha_j \mu_j}$. 
\end{thm} 
\begin{proof}
Define the notation as follows:

$Y_i(j)=$ amount of time spent in state $i$ during $j^\text{th}$ visit to that state. $i,j \geq 0$. \\
$N_i(m)=$ number of visits to state $i$ in the first $m$ transitions of the semi-Markov process.\\

The proportion of time in $i$ during the first $m$ transitions:\\

\begin{flalign*}
P_{i=m}&= \frac{\sum_{j=1}^{N_i(m)}Y_i(j)}{\sum_l \sum_{j=1}^{N_i(m)}Y_i(j) }\\
&= \frac{\frac{N_i(m)}{m}\sum_{j=1}^{N_i(m)}Y_i(j)}{\sum_l \frac{N_i(m)}{m} \sum_{j=1}^{N_i(m)}Y_i(j) }\\
\end{flalign*}
Since $N_i(m)\rightarrow \infty$ as $m \rightarrow \infty$, it follows from the strong law of large numbers that $\frac{\sum_{i=2}^{N_i(m)}Y_i(j)}{N_i(m)}\rightarrow \mu_i$ and $\frac{N_i(m)}{m}\rightarrow (E[\text{number of transitions between visits to state }i])^{-1}=\alpha_i$. Letting $m \rightarrow \infty$, result follows.
\end{proof}
\begin{thm}
If Semi Markov process is irreducible  and non lattice, then $\lim_{t \rightarrow \infty}P(Z(t)=i,Y(t)>x,S(t)=j|Z(0)=k)=\frac{P_{ij}\int_x^\infty F_{ij}^c(y)d(y)}{\mu_{ii}}$. Let $Y(t)$ denote the time from $t$ until the next transition. $S(t)$ state entered at the first transition after $t$. 
\end{thm}
\begin{proof}
The trick lies in defining the "ON" time. 
\begin{flalign*}
E[\text{ON time in a cycle}]=E[(X_{ij}-x)^+].
\end{flalign*}
\end{proof}
\textbf{Corollary:} 
\begin{flalign*}
\lim_{t \rightarrow \infty} P(Z(t)=i, Y(t) >x|Z(0)=k)= \frac{\int_{x}^{\infty}H_i^c(y)d(y)}{\mu_{ii}}.
\end{flalign*}



Since a Continuous Time Markov Chain (CTMC) is a semi-Markov chain with $F_{ij}(t)=1-e^{\nu_i t}$. From the theory of semi-Markov process, if the embedded Markov chain with transition probabilities $P_{ij}$ is irreducible and positive recurrent, then the limiting process,
\begin{align}
\label{eq:LimitingProbability}
\pi_j \triangleq \lim_{t \rightarrow \infty } P_{ij}(t)= \frac{\alpha_j / \nu_j}{\sum_j \alpha_j /\nu_j},
\end{align}
where $\alpha_i$ is the stationary distribution of the embedded Markov chain. From \ref{eq:LimitingProbability} and \ref{eq:StationaryDistribution}, we see that $\{\pi_j\}$ is the unique non-negative solution to 
\begin{equation}
\nu_j\pi_j=\sum_i \nu_i\pi_iP_{ij},~ \sum_j \pi_j =1.
\end{equation}

\begin{rem}
From the theory of semi-Markov process, it also follows that $\pi_j$ also equals the long-run proportion of time the process is in state $j$.
\end{rem}
\begin{rem}
If the initial state is chosen according to the limiting probabilities $\{\pi_j\}$ then the resultant process will be stationary. That is,
\begin{equation*}
\sum_i \pi_iP_{ij}(t)=\pi_j,~ \text{for all}~ t.
\end{equation*}
\begin{proof}
\begin{flalign*}
\sum_{i}P_{ij}(t)\pi_i &=\sum_{i}P_{ij}(t)\lim_{s \rightarrow \infty}P_{ki}(s)\\
&=\lim_{s \rightarrow \infty} \sum_{i}P_{ij}(t)P_{ki}(s)\\
&=\lim_{s \rightarrow \infty}P_{ki}(t+s)\\
&=\pi_j.
\end{flalign*}
\end{proof} 
\end{rem}
\end{document}
