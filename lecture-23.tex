% !TEX spellcheck = en_US
% !TEX spellcheck = LaTeX
\documentclass[a4paper,10pt,english]{article}
\usepackage{%
	amsfonts,%
	amsmath,%	
	etex,%
	amssymb,%
	amsthm,%
	babel,%
	bbm,%
	%biblatex,%
	caption,%
	centernot,%
	color,%
	enumerate,%
	epsfig,%
	epstopdf,%
	geometry,%
	graphicx,%
	hyperref,%
	latexsym,%
	mathtools,%
	multicol,%
	pgf,%
	pgfplots,%
	pgfplotstable,%
	pgfpages,%
	proof,%
	psfrag,%
	subfigure,%	
	tikz,%
	ulem,%
	url%
}	

\usepackage[mathscr]{eucal}
\usepgflibrary{shapes}
\usetikzlibrary{%
  arrows,%
	backgrounds,%
	chains,%
	decorations.pathmorphing,% /pgf/decoration/random steps | erste Graphik
	decorations.text,%
	matrix,%
  	positioning,% wg. " of "
  	fit,%
	patterns,%
  	petri,%
	plotmarks,%
  	scopes,%
	shadows,%
  	shapes.misc,% wg. rounded rectangle
  	shapes.arrows,%
	shapes.callouts,%
  	shapes%
}

\theoremstyle{plain}
\newtheorem{thm}{Theorem}[section]
\newtheorem{lem}[thm]{Lemma}
\newtheorem{prop}[thm]{Proposition}
\newtheorem{cor}[thm]{Corollary}

\theoremstyle{definition}
\newtheorem{defn}[thm]{Definition}
\newtheorem{conj}[thm]{Conjecture}
\newtheorem{exmp}[thm]{Example}
\newtheorem{assum}[thm]{Assumptions}
\newtheorem{axiom}[thm]{Axiom}

\theoremstyle{remark}
\newtheorem{rem}{Remark}
\newtheorem{note}{Note}

\newcommand{\norm}[1]{\left\lVert#1\right\rVert}
\newcommand{\indep}{\!\perp\!\!\!\perp}
\DeclarePairedDelimiter\abs{\lvert}{\rvert}%
%\DeclarePairedDelimiter\norm{\lVert}{\rVert}%
\newcommand{\tr}{\operatorname{tr}}
\newcommand{\R}{\mathbb{R}}
\newcommand{\Q}{\mathbb{Q}}
\newcommand{\N}{\mathbb{N}}
\newcommand{\E}{\mathbb{E}}
\newcommand{\Z}{\mathbb{Z}}
\newcommand{\B}{\mathscr{B}}
\newcommand{\C}{\mathcal{C}}
\newcommand{\T}{\mathscr{T}}
\newcommand{\F}{\mathcal{F}}
\newcommand{\G}{\mathcal{G}}
%\newcommand{\ba}{\begin{align*}}
%\newcommand{\ea}{\end{align*}}

\makeatletter
\def\th@plain{%
  \thm@notefont{}% same as heading font
  \itshape % body font
}
\def\th@definition{%
  \thm@notefont{}% same as heading font
  \normalfont % body font
}
\makeatother
\date{}
\title{Lecture 23: Brownian Motion}
\author{}

\begin{document}
\maketitle
\section{Introduction}
\begin{defn} A continuous time stochastic process $\{X(t), t \geqslant 0\}$ is called \textbf{standard Brownian motion process} if the following hold.
\begin{enumerate}[i)]
\item $X(0) = 0$.
\item Process has stationary and independent increments.
\item For every $t > 0$, process is normally distributed with mean $0$ and variance $t$.
\end{enumerate}
\end{defn}

\begin{defn}
Let $S_n$ be a simple symmetric random walk with step sizes $\{X_i: i \in \N\}$ being iid Rademacher random variables. That is,
\begin{align*}
\Pr\{X_i = 1\} &= \Pr\{X_i = -1\} = \frac{1}{2}.
\end{align*}
Let $\Delta t > 0$, and we choose $\Delta x = \sqrt{\Delta t}, n(t) = \frac{t}{\Delta t}$ to define \textbf{limiting scaled symmetric random walk}
\begin{align*}
X(t) \triangleq \lim_{\Delta t \to 0}\Delta xS_{n(t)}.
\end{align*}
\end{defn}
\begin{lem} For scaled symmetric random walk $\Delta xS_{n(t)}$, we have 
\begin{xalignat*}{3}
&\E \Delta xS_{n(t)} = 0,&&\Var \Delta xS_{n(t)} = \frac{(\Delta x)^2}{\Delta t} t.
\end{xalignat*}
\end{lem}
\begin{thm} Limiting scaled symmetric random walk $X(t)$ is a standard Brownian motion.
\end{thm}
\begin{proof} We will show that the limiting process satisfies all three criterion in the definition of standard Brownian motion. 
\begin{enumerate}[i)]
\item First condition holds trivially since $X(0) = S_0$. 
\item Since step sizes are independent random variables, it follows that the random walk $\Delta xS_{n(t)}$ has independent increments. 
Further, since step sizes have identical distribution, distribution of $S_{n(t)} - S_{n(s)}$ is same as distribution of $S_{n(t-s)}$ for $s,t$ multiples of $\Delta t$. 
By continuity of probability, we conclude that $X(t)$ as stationary independent increments. 
\item From previous lemma, it is clear that mean and variance of $X(t)$ are zero and $t$ respectively. 
Further, it follows from central limit theorem that limit of $\frac{S_{n(t)}}{\sqrt{n(t)}}= \frac{X(t)}{\sqrt{t}}$ is standard normal random variable, as $n(t) \to \infty$.
\end{enumerate}
\end{proof}
\begin{thm} Standard Brownian motion $X(t)$ is a.s. path-wise continuous and nowhere differentiable function.
\end{thm}
\begin{rem} Since standard Brownian motion satisfies independent increment property, it is a Markov process.
\end{rem}
\begin{rem} Due to stationary increments property, distribution of $X(t) - X(s)$ is same as distribution of $X(t-s)$.
\end{rem}
\begin{cor} Finite dimensional density of a standard Brownian motion at times $t_1 < t_2 < \ldots < t_n$ is given by 
\begin{align*}
f(x_1, x_2, \ldots, x_n) = \Pr\{X(t_i) = x_i, i \in [n]\} = \prod_{i \in [n]}f_{t_i - t_{i-1}}(x_i - x_{i-1}),
\end{align*}
where $t_0 = x_0 = 0$ and 
\begin{align*}
f_t(x) = \frac{1}{\sqrt{2 \pi t}}\exp\left(-\frac{x^2}{2t}\right).
\end{align*}
\end{cor}
\begin{defn} A stochastic process $\{X(t), t\geqslant 0\}$ is called a \textbf{Gaussian process} if all finite dimensional distributions are multivariate normal.
\end{defn}
\begin{cor} Standard Brownian motion process is a Gaussian process.
\end{cor}
\begin{prop} Conditional density of standard Brownian motion $X(s)$ given $X(t) = B$ for $s < t$ is given by
\begin{align*}
f_{s|t}(x|B) &= \frac{1}{\sqrt{2\pi s(t-s)/t}} \exp\left(-\frac{(x-Bs/t)^2}{2s(t-s)/t}\right), 
\end{align*}
%for some positive normalization constant $K$.
\end{prop}
\begin{proof} Result follows from Bayes rule for densities applied to standard Brownian motion to get
\begin{align*}
f_{s|t}(x|B) &= \frac{f_s(x)f_{t-s}(B-x)}{f_t(B)} = \frac{1}{\sqrt{2\pi s(t-s)/t}}\exp\left(-\frac{x^2}{2s}-\frac{(B-x)^2}{2(t-s)} + \frac{B^2}{2t}\right)\\
&=  \frac{1}{\sqrt{2\pi s(t-s)/t}}\exp\left(-\frac{t(x-sB/t)^2}{2s(t-s)}\right).
\end{align*}
\end{proof}
\begin{cor} Conditional mean and variance of standard Brownian motion $X(s)$ given $X(t) = B$ for $s < t$ is given by 
\begin{xalignat*}{3}
&\E[X(s)|X(t) = B] = B \frac{s}{t}, && \Var[X(s)|X(t) = B] = s\left(1-\frac{s}{t}\right).
\end{xalignat*}
\end{cor}
\begin{cor} Let $s/t = \alpha < 1$. Then, the conditional distribution of $X(s)$ given $X(t)$ is normal with mean $\alpha X(t)$ and variance $t \alpha(1-\alpha)$.
\end{cor}
\begin{thm} A continuous time stochastic process $\{X(t), t \geqslant 0\}$ is a standard Brownian motion process iff it is a zero mean Gaussian process with 
\begin{align*}
\Cov(X(t), X(s)) &= s \wedge t.
\end{align*}
\end{thm}
\begin{proof} Let $X(t)$ be a standard Brownian motion, then it is a zero mean Gaussian process. 
Therefore, by linearity of covariance for independent random variables, we have
\begin{align*}
\Cov(X(t),X(s))  = \Cov(X(t)-X(s),X(s)) + \Var X(s).
\end{align*}
Let $s \leq t$, then result follows from that $X(t) - X(s)$ is independent of $X(s)$ by independent increments property and that $X(s)$ is zero mean. 
\end{proof}
\begin{defn} Consider a standard Brownian motion process $\{X(t), t \geqslant 0\}$ in duration $[0,1]$. Conditional stochastic process $\{X(t), t \in [0,1]| X(1) = 0\}$ is called a \textbf{Brownian bridge}.
\end{defn}
\begin{prop} Brownian bridge is a zero-mean Gaussian process with 
\begin{align*}
\Cov(X(t), X(s)|X(1) = 0) &= s(1- t),\text{ for } s \in [t, 1].
\end{align*}
\end{prop}

\section{Martingales for Random Walks}
%Let
%\begin{align*}S_n = \sum_{k=1}^n X_k \quad n \geq 1\end{align*}
%denote a random walk. Then we have the following results.
\begin{prop}
A random walk $S_n$ with step size $X_n \in [-M,M]\cap \Z$ for some finite $M$ is a recurrent DTMC iff $\E X = 0$.
\end{prop}
\begin{proof}
If $\E X \neq 0$, the random walk is clearly transient since, it will diverge to $\pm \infty$ depending on the sign of $\E X$. 
Conversely, if $\E X =0$, then $S_n$ is a martingale. 
Assume that the process starts in state $i$. 
We define
\begin{xalignat*}{3}
&A = \{-M, -M+1,\cdots,-2,-1\}, &&A_j = j + [M], ~j > i.
\end{xalignat*}
%\begin{align*}A_j = \{j, j+1,\cdots,j+M\}\end{align*}
Let $N$ denote the hitting time to $A$ or $A_j$ by random walk $S_n$. 
Since $N$ is a stopping time, by optional stopping theorem, we have
\begin{align*}
\E_i[S_N] =\E_i[S_0] = i.
\end{align*}
Thus we have
\begin{align*}
i &= \E_i[S_N] %= \E_i[S_N|S_N \in A]\Pr\{S_N \in A\} + \E_i[S_N|S_N \in A_i]\Pr\{S_N \in A_j]\\
\geq -M\P_i\{S_N \in A\} + j(1-\P_i\{S_N \in A_j\}). 
\end{align*}
Rearranging this, we get a bound on probability of random walk $S_n$ hitting $A$ over $A_j$ as 
\begin{align*}
\P_i\{S_n \in A \text{ for some } n\} \geq \P_i\{S_N \in A\} \geq \frac{j-i}{j+M}.
\end{align*}
%Thus we have 
%\begin{align*}P[\mbox{process ever enters }A] \geq P[S_N \in A] \geq \frac{j-i}{j+M}\end{align*}
Taking limit $j \to \infty$, we see that for any $i \geq 0$, we have 
%\begin{align*}
$\P_i\{S_n \in A \text{ for some }n \} = 1$.
%\end{align*}
Similarly, taking $B = \{1,2,\cdots, M\}$, we can show that for any $i \geq 0$, 
%\begin{align*}
$\P_i\{S_n \in B \text{ for some }n \} = 1.$
%\end{align*}
Result follows from combining the above two arguments to see that for any $i \geq 0$,
\begin{align*}
\P_i\{S_n \in A\cup B \text{ for some } n\} = 1.
\end{align*}
\end{proof}
\begin{prop} Consider a random walk $S_n$ with mean step size $\E[X] \neq 0$. 
For $A,B > 0$, let $P_A$ denote the probability that the walk hits a value greater than $A$ before it hits a value less than $-B$. Then, 
\begin{align*}
P_A \approx \frac{1-e^{-\theta B}}{e^{\theta A}-e^{\theta B}}.
\end{align*}
Approximation is an equality when step size is unity and $A$ and $B$ are integer valued.
\end{prop}
\begin{proof}
Now  For $A,B > 0$, we wish to compute the probability $P_A$ that the walk hits at least $A$ before it hits a value $\leq -B$. Let $\theta \neq 0$ s.t
\begin{align*}E[e^{\theta X}] = 1\end{align*}
Now let $Z_n = e^{\theta S_n}$. We can see that $Z_n$ is a martingale with mean 1. Define $N$ as
\begin{align*}N = \min \{S_n \geq A \mbox{ or } S_n \leq -B\}\end{align*}
From Doob's Theorem, $E[e^{S_N}] = 1$. Thus we get
\begin{align*} 1 = E[e^{\theta S_N}|S_N \geq A]P_A + E[e^{\theta S_N}|S_N \leq -B](1-P_A)\end{align*}

We can obtain an approximation for $P_A$ byneglecting the overshoots past $A$ or $-B$. Thus we get
\begin{flalign*}
E[e^{\theta S_N}|S_N\geq A] &\approx e^{\theta A} \\
E[e^{\theta S_N}|S_N\leq -B] &\approx e^{-\theta B} \\
\end{flalign*}
Hence we get, 
\end{proof}
As an assignment, show that 
\begin{align*}E[N] \approx \frac{AP_A - B(1-P_A)}{E[X]}\end{align*}

\begin{exmp}\textbf{Gambler Ruin}
Consider a simple random walk with probability of increment = $p$. As an exercise, show that $E\left[(q/p)^X\right] = 1$ and thus $e^\theta = q/p$. If $A$ and $B$ are integers, then there is no overshoot and hence, our approximations are exact. Thus
\begin{align*}P_A = \frac{(q/p)^B - 1}{(q/p)^{A+B} -1}\end{align*}
 Suppose $E[X]<0$ and we wish to know if the random walk ever crosses $A$. Then
\begin{flalign*}
1 &= E[e^{\theta S_N}|S_N\geq A]P[\mbox{process crossed $A$ before $-B$}] \\
&+ E[e^{\theta S_N}|S_N\leq -B]P[\mbox{process crossed $-B$ before $A$}]
\end{flalign*}
Now $E[X]<0$ implies $\theta > 0$ (Why?). Hence we have
\begin{align*}1 \geq e^{\theta A} P[\mbox{process crossed $A$ before $-B$}]\end{align*}
Taking $B$ to $\infty$ yields
\begin{align*}P[\mbox{Random walk ever crosses A}] \leq e^{-\theta A}\end{align*}

\end{exmp}
\section{Application to G/G/1 Queues and Ruin}
\subsection{The G/G/1 Queue}
For the G/G/1 queue, the limiting distribution of delay is
\begin{align*}P[D_\infty \geq A] = P[S_n \geq A \mbox{ for some } n]\end{align*}
where 
\begin{align*}S_n = \sum_{k=1}^n U_k, \quad U_k = Y_k-X_{k+1} \end{align*}
Here $Y_i$ is the service time of the ith customer and $X_i$ is the interarrival duration between customer $i-1$ and customer $i$.
Thus when $E[U] = E[Y] - E[X] < 0$, letting $\theta > 0$ such that
\begin{align*}E[e^{\theta U}] = E[e^{\theta(Y-X)}] = 1\end{align*}
We get
\begin{align*}P[D_\infty \geq A] \leq e^{-\theta A}\end{align*} 

Now the exact distribution of $D_\infty$ can be calculated when services are exponential. Hence assume $Y_i \sim exp(\mu)$. Once again,
\begin{flalign*}
1 &= E[e^{\theta S_N}|S_N\geq A]P[S_n\mbox{ crossed $A$ before $-B$}] \\
&+ E[e^{\theta S_N}|S_N\leq -B]P[S_n\mbox{ crossed $-B$ before $A$}]
\end{flalign*}
Let us compute $E[e^{\theta S_N}|S_N \geq A]$ first. Let us condition this on $N=n$ and $X_{n+1} - \sum_{i=1}^{n-1} (Y_i - X_{i+1}) = c$. By the memoryless property, the conditional distribution of $Y_n$ given $Y_n > c+A$ is just $c+A$ plus an exponential with rate $\mu$. Thus we get
\begin{flalign*}
E[e^{\theta S_N}|S_N \geq A] &= E[e^{\theta(A+Y)}] \\
&=\frac{\mu e^{\theta A}}{\mu - \theta}
\end{flalign*}
Now substituting back, we get
\begin{flalign*}
1 &= \frac{\mu e^{\theta A}}{\mu - \theta}P[S_n\mbox{ crossed $A$ before $-B$}] \\
&+ E[e^{\theta S_N}|S_N\leq -B]P[S_n\mbox{ crossed $-B$ before $A$}]
\end{flalign*}
Now as $\theta > 0$, let $B\to \infty$ to get
\begin{align*}1 = \frac{\mu e^{\theta A}}{\mu - \theta} P[S_n \mbox{ ever crosses }A]\end{align*}
And hence
\begin{align*}P[D_\infty \geq A] = \frac{\mu - \theta}{\mu}e^{-\theta A}\end{align*}

\subsection{A Ruin Problem}
Suppose claims made to an insurance company follow a renewal process with iid interarrival times $\{X_i\}$. Let the values of the claims also be iid and independent of the renewal process $N(t)$ of their occurence. Let $Y_i$ be the ith claim value. Thus the total value of claims till time $t$ is $\sum_{k=1}^{N(t)}Y_i$. Now let us suppose the insurance company receives money at constant rate $c$ per unit time, $c>0$. We wish to compute the probability of the insurance company, starting with capital $A$, will eventually be wiped out or \textbf{ruined}. Thus we require
\begin{align*}p = P\left\{ \sum_{k=1}^{N(t)}Y_i > ct + A \mbox{ for some } t\geq 0\right\}\end{align*}
As an assignment, show that the company will be ruined if $E[Y] \geq cE[X]$. So let us assume that $E[Y] < cE[X]$. Also the ruin occurs when a claim is made. After the $n$th claim, the company's fortune is
\begin{align*}A + c\sum_{k=1}^n X_k -\sum_{k=1}^n Y_k \end{align*}
Letting $S_n = \sum_{k=1}^n Y_i - cX_i$ and $p(A) = P[S_n > A \mbox{ for some }n]$. As $S_n$ is a random walk, we see that 
\begin{align*}p(A) = P[D_\infty > A]\end{align*}

Now the results from the G/G/1 queue apply.

\section{Blackwell Theorem on the Line}
Let $S_n$ denote a random walk where $0<\mu=E[X] < \infty$. Let 
\begin{align*}U(t)=\#\{n: S_n \leq t\} = \sum_{n=1}^\infty I_n\end{align*}
Where $I_n=1$ if $S_n \leq t$ and zero else. Observe that if $X_n$ are nonnegative, then $U(t) = N(t)$. Let $u(t) = E[U(t)]$. Now we prove an analog of Blackwell Renewal Theorem. 

\begin{thm}
\textbf{(Blackwell renewal theorem)} If $\mu > 0$ and $X_i$ are not lattice, then
\begin{align*}u(t+a) - u(t) \to a/\mu \quad t\to \infty \quad \mbox{for }a>0\end{align*}
\end{thm}
Let us define a few concepts. We say an \textbf{ascending ladder variable of ladder height $S_n$} occurs at time $n$ when
\begin{align*}S_n > \max(S_0,S_1,\cdots, S_{n-1})\end{align*}
where $S_0 = 0$. We may deduce that since $X_i$ are iid random variables, then the random variables $(N_i,S_{N_i}-S_{N_{i-1}})$ are iid; where $N_i$ denotes the time between the $(i-1)$th and ith random variable. We may analogously define descending ladder variables. Now let $p(p_*)$ denote the probability of ever achieving an ascending/descending ladder variable.
\begin{align*}p = P\{S_n > 0 \mbox{ for some }n\},\quad p_* = P\{S_n < 0 \mbox{ for some }n\}\end{align*}
At each ascension/descension there is a probability $p$ (resp $p_*$) of achieving another one. Hence the number of ascensions/descensions is geometrically distributed. The number of ascending ladder variables (ascensions) will have finite mean iff $p < 1$. Now as $E[X] > 0$, by SLLN, we deduce that $w.p.1$, there will be infinitely many ascending ladder variables but finitely many descending ones. That is $p =1$ and $p_* < 1$.

\begin{proof}
The successive ascending ladder heights are a renewal process. Let $Y(t)$ be the excess time. Now given the value of $Y(t)$, the distribution of $U(t+a) - U(t)$ is independent of $t$. (Why?). Hence let us denote
\begin{align*}E[U(t+a) - U(t)|Y(t)] = g(Y(t))\end{align*}
for some function $g$. Now taking expectations yields
\begin{align*}u(t+a) - u(t) = E[g(Y(t))]\end{align*}

Now since $Y(t) \to^d Y_\infty$ where $Y_\infty$ has the equilibrium distribution, we have $E[g(Y(t))] \to E[g(Y_\infty)]$. The result would be true if we show $g$ is continuous and bounded. We leave that as an exercise. For now, we deduce that the limit exists. Let
\begin{align*}h(a) = \lim_{t \to \infty}u(t+a) - u(t)\end{align*}
This also implies $h(a+b) = h(a) + h(b)$. Thus for some constant $c$, 
\begin{align*}h(a) = ca\end{align*}

Now to get $c$, let $N_t$ denote the first $n$ for which $S_n > t$. If $X_i$ are upper bounded by $M$, then
\begin{align*}t < \sum_{i=1}^{N_t} X_i \leq t+M\end{align*}
Taking expectations, and using Wald's Lemma, yields
\begin{align*}t < E[N_t]\mu \leq t+M\end{align*}
Thus 
\begin{align*}\frac{E[N_t]}{t} \to \frac{1}{\mu}\end{align*}
If $X_i$ are unbounded, use the truncation arguments done while proving Elementary renewal theorem. Now $U(t)$ can be expressed as
\begin{align*}U(t) = N_t -1 +N_t^*\end{align*}
where $N_t^*$ is the number of times $S_n \leq t$ after having crossed $t$. Since $N_t^*$ is not greater than the number of points occuring after $N_t$ when the random walk is less than $S_{N_t}$, we get
\begin{align*}E[N_t^*] \leq E[\mbox{number of $n$ such that $S_n < 0$}]\end{align*}

Hence if we argue that RHS of above is finite, then
\begin{align*}\frac{u(t)}{t} \to \frac{1}{\mu}\end{align*}

From the first proposition in Random walks, we have $E[N] <\infty$ where $N$ is the first value of $n$ for which $S_n > 0$. At time $N$, with positive probability $1-p^*$, no future value of random walk will fall below $S_N$. Thus,
\begin{align*}E[\mbox{number of $n$ where $S_n < 0$}] \leq \frac{E[N|X_1<0]}{1-p^*} < \infty\end{align*}

Now follow the steps illustrated in the Blackwell renewal theorem (original) proof to arrive at the desired result.

\end{proof}
\end{document}
