% !TEX spellcheck = en_US
% !TEX spellcheck = LaTeX
\documentclass[a4paper,10pt,english]{article}
\usepackage{%
	amsfonts,%
	amsmath,%	
	etex,%
	amssymb,%
	amsthm,%
	babel,%
	bbm,%
	%biblatex,%
	caption,%
	centernot,%
	color,%
	enumerate,%
	epsfig,%
	epstopdf,%
	geometry,%
	graphicx,%
	hyperref,%
	latexsym,%
	mathtools,%
	multicol,%
	pgf,%
	pgfplots,%
	pgfplotstable,%
	pgfpages,%
	proof,%
	psfrag,%
	subfigure,%	
	tikz,%
	ulem,%
	url%
}	

\usepackage[mathscr]{eucal}
\usepgflibrary{shapes}
\usetikzlibrary{%
  arrows,%
	backgrounds,%
	chains,%
	decorations.pathmorphing,% /pgf/decoration/random steps | erste Graphik
	decorations.text,%
	matrix,%
  	positioning,% wg. " of "
  	fit,%
	patterns,%
  	petri,%
	plotmarks,%
  	scopes,%
	shadows,%
  	shapes.misc,% wg. rounded rectangle
  	shapes.arrows,%
	shapes.callouts,%
  	shapes%
}

\theoremstyle{plain}
\newtheorem{thm}{Theorem}[section]
\newtheorem{lem}[thm]{Lemma}
\newtheorem{prop}[thm]{Proposition}
\newtheorem{cor}[thm]{Corollary}

\theoremstyle{definition}
\newtheorem{defn}[thm]{Definition}
\newtheorem{conj}[thm]{Conjecture}
\newtheorem{exmp}[thm]{Example}
\newtheorem{assum}[thm]{Assumptions}
\newtheorem{axiom}[thm]{Axiom}

\theoremstyle{remark}
\newtheorem{rem}{Remark}
\newtheorem{note}{Note}

\newcommand{\norm}[1]{\left\lVert#1\right\rVert}
\newcommand{\indep}{\!\perp\!\!\!\perp}
\DeclarePairedDelimiter\abs{\lvert}{\rvert}%
%\DeclarePairedDelimiter\norm{\lVert}{\rVert}%
\newcommand{\tr}{\operatorname{tr}}
\newcommand{\R}{\mathbb{R}}
\newcommand{\Q}{\mathbb{Q}}
\newcommand{\N}{\mathbb{N}}
\newcommand{\E}{\mathbb{E}}
\newcommand{\Z}{\mathbb{Z}}
\newcommand{\B}{\mathscr{B}}
\newcommand{\C}{\mathcal{C}}
\newcommand{\T}{\mathscr{T}}
\newcommand{\F}{\mathcal{F}}
\newcommand{\G}{\mathcal{G}}
%\newcommand{\ba}{\begin{align*}}
%\newcommand{\ea}{\end{align*}}

\makeatletter
\def\th@plain{%
  \thm@notefont{}% same as heading font
  \itshape % body font
}
\def\th@definition{%
  \thm@notefont{}% same as heading font
  \normalfont % body font
}
\makeatother
\date{}
\title{Lecture 01: Introduction to Stochastic Processes}
\author{}

\begin{document}
\maketitle

\section{Probability Review}
%\begin{defn}
A \textbf{probability space} $(\Omega, \F, P)$ consists of set of all possible outcomes denoted by $\Omega$ and called a \textbf{sample space}, a collection of subsets $\F$ of sample space, and a non-negative set function probability $P: \F \to [0,1]$, with the following properties.
\begin{enumerate}
\item The collection of subsets of $\F$ is a $\sigma$-algebra, that is it contains an empty set and is closed under complements and countable unions.
\item Set function $P$ satisfies $P(\Omega) = 1$, and for every countable pair-wise disjoint collection $\{A_n \in \F: n \in \N \}$, we have
\begin{align*}
P(\bigcup_{n}A_n) = \sum_nP(A_n).
\end{align*}
\end{enumerate}
%\end{defn}
There is a natural order of inclusion on sets through which we can define monotonicity of probability set function $P$. To define continuity of this set function, we %need to 
define limits of sets. 
%\begin{defn} 
For a sequence of sets $\{A_n: n \in \N\}$, we define \textbf{limit superior} and \textbf{limit inferior} of this sequence respectively as 
\begin{xalignat*}{3}
&\lim\sup_nA_n = \bigcap_n\bigcup_{k \geq n}A_k, && \lim\inf_nA_n = \bigcup_n\bigcap_{k \geq n}A_k.
\end{xalignat*}
We say that limit exists if the limit superior and limit inferior are equal, and is equal to the limit of the sequence of sets.
%\end{defn}
\begin{thm} Probability set function is monotone and continuous.
\end{thm}
\begin{proof}
Let $A \subseteq B$ both subsets be elements of $\mathcal{F}$, then from the additivity of probability over disjoint sets $A$ and $B \setminus A$, we have 
\begin{align*}
P(B) &= P(A \cup B\setminus A) = P(A) + P(B\setminus A) \geq P(A). 
\end{align*}
Monotonicity follows from non-negativity of probability set function, that is since $P(B \setminus A) > 0$. 
For continuity from below, we take an increasing sequence of sets $\{A_n: n \in \N\}$, such that $A_n \subseteq A_{n+1}$ for all $n$. 
Then, it is clear that $A_n \uparrow A = \cup_{n}A_n$. 
We can define disjoint sets $\{E_n: n \in \N\}$, where 
\begin{xalignat*}{3}
&E_1 = A_1,&& E_n = A_{n} \setminus A_{n-1},~n \geq 2.
\end{xalignat*}
The disjoint sets $E_n$'s satisfy $\cup_{i=1}^nE_i = A_n$ for all $n \in \N$ and $\cup_{n}E_n = \cup_nA_n$.
From the above property and the additivity of probability set function over disjoint sets, it follows that 
\begin{align*}
P(A) &= P\cup_{n}E_n = \sum_{n \in \N}P(E_n) = \lim_{n \in \N}\sum_{i=1}^nP(E_i) =  \lim_{n \in \N}P\cup_{i=1}^nE_i =  \lim_{n \in \N}P(A_n).
\end{align*}
For continuity from below, we take decreasing sequence of sets $\{A_n: n \in \N\}$, such that $A_{n+1} \subseteq A_{n}$ for all $n$. 
We can form increasing sequence of sets $\{B_n: n \in \N\}$ where $B_n = A_n^c$. 
Then, the continuity from above follows from continuity from above. 
\end{proof}
%\begin{defn} 
A real valued \textbf{random variable} $X$ on a probability space $(\Omega, \F, P)$ is a function $X: \Omega \to \R$ such that for every $x \in  \R$, we have $\{ \omega \in \Omega: X(\omega) \leq x \} = X^{-1}(-\infty, x] \in \F$.
%\end{defn}
%\begin{defn} 
%For a random variable $X$ defined on probability space $(\Omega, \F, P)$, 
The \textbf{distribution function} $F: \R \to [0,1]$ for this random variable $X$ is defined as 
\begin{align*}
F(x) = (P \circ X^{-1})(-\infty,x],~\forall x \in \R.
\end{align*}
Let $g: \R \to \R$ be a function. Then, the \textbf{expectation} of $g(X)$ is defined as 
\begin{align*}
\E g(X) = \int_{x \in \R} g(x) dF(x) .%= \int_{\omega \in \Omega} g(x) d(P \circ X^{-1})(-\infty, x].
\end{align*}
%\end{defn}
\begin{thm} Distribution function $F$ of a random variable $X$ is non-negative, monotone increasing, continuous from the right, and has countable points of discontinuities. Further, if $P \circ X^{-1}(\R) = 1$, then
\begin{xalignat*}{5}
&\lim_{x \to -\infty}F(x) = 0,&& \lim_{x \to \infty}F(x) = 1.
\end{xalignat*}
\end{thm}
\begin{proof}
Non-negativity and monotonicity of distribution function follows from non-negativity and monotonicity of probability set function, 
and the fact that for $x_1 < x_2$
\begin{align*}
X^{-1}(-\infty, x_1] \subseteq X^{-1}(-\infty, x_2]. 
\end{align*}
Let $x_n \downarrow x$ be a decreasing sequence of real numbers. 
Then, the right continuity of distribution function follows from the continuity from above of probability set functions. 
We take decreasing sets $\{A_n: n \in \N\}$, where 
\begin{align*}
A_n &= \{\omega \in \Omega: X(\omega) \leq x_n\}.
\end{align*}
\end{proof}
%\begin{defn} 
%Let $X$ be a random variable on a probability space $(\Omega, \F, P)$ with a distribution function $F$. 
%Let $g: \R \to \R$ be a function. Then, the \textbf{expectation} of $g(X)$ is defined as 
%\begin{align*}
%\E[g(X)] = \int_{x \in \R} g(x) dF(x) .%= \int_{\omega \in \Omega} g(x) d(P \circ X^{-1})(-\infty, x].
%\end{align*}
%\end{defn}

\section{Deterministic and stochastic models}
Evolution of a \textbf{deterministic} system is characterized by a set of equations, with each run leading to the same outcome given the same initial conditions. 
Evolution of a \textbf{stochastic} system is at least partially random, and each run of the process leads to potentially a different outcome. 
Each of these different runs are called a \textbf{realization} or a \textbf{sample path} of the stochastic process.
\begin{shaded*}
We are interested in modeling, analysis, and design of stochastic systems. 
Following are some of the stochastic systems from different disciplines of science and engineering. 
\begin{itemize}
%\item Physics
%\begin{exmp}
%\item %[Movement of gas molecules] 
%Movements of gas molecules when environment has fluctuating densities due to temperature variations.
%\end{exmp}
%\item Chemistry
%\begin{exmp}
\item %[Chemical reactions] 
Evolution of number of molecules due to chemical reaction, where the time to form new molecules is uncertain and it depends on density of other molecules. 
%\end{exmp}
%\begin{exmp}
\item %[Financial commodities] 
Financial commodities like stock prices, currency exchange rates fluctuate with time. 
These can be modeled by random walks. 
One can provide probabilistic predictions and optimal buying and selling strategies using these models.
%\end{exmp}
%\begin{exmp}
\item %[Photon Detection] 
Machines that detect photons, have a dead time post a successful detection. 
This adds uncertainty in estimating photon density. 
These processes can be modeled by an \emph{on-off} process. 
%\end{exmp}
%\begin{exmp}
\item %[Epidemics] 
A contagious disease can spread very quickly across a region. 
This is similar to a content getting viral on internet. 
One can model spread of epidemics on network by Urn models. 
%\end{exmp}
%\begin{exmp}
\item %[Earthquake occurrences] 
Counting number of earthquakes that occur everyday at a certain location. 
These can be modeled by a counting process, and inter-arrival time of the quakes can be estimated to make probabilistic predictions. 
%\end{exmp}
%\begin{exmp}
\item %[Population growth] 
A mother cell takes a random amount of time to subdivide and create a daughter cell. 
A daughter cell takes certain random time to mature, and become a mother cell. 
A mother cell dies after certain number of sub-divisions. 
One is interested in finding out the asymptotic behavior of population density. 
%\end{exmp}
%\begin{exmp}
\item %[Page Rank] 
Popularity of a page depends on how quickly one can reach it from other pages on the Internet. 
Equilibrium distribution of certain random walks on graphs can be used to estimate page ranks on the web.
%\end{exmp}
\end{itemize}
\end{shaded*}

\section{Stochastic Processes}
%\begin{defn} 
A  collection of random variables $\{X_t \in \mathcal{X}: t \in T\}$ each defined on the same probability space $(\Omega, \F, P)$ is called a \textbf{random process} for an arbitrary index set $T$ and arbitrary state space $\mathcal{X}$. 
%\end{defn}

\subsection{Classification}
If the state space $T$ is countable, the stochastic process is called \textbf{discrete}-time stochastic process or random sequence. 
When the state space $T$ is uncountable, it is called \textbf{continuous}-time stochastic process. 
However, $T$ doesn't have to be time, if the index set is space, and then the stochastic process is spatial process. 
When $T = \R^n \times [0, \infty)$, stochastic process $X(t)$ is a spatio-temporal process. 
State space $\mathcal{X}$ can also be countable or uncountable. 
\begin{shaded*}
We list some examples of each such stochastic process. 
\begin{itemize} 
\item Discrete random sequence: brand switching, discrete time queues, number of people at bank each day.
\item Continuous random sequence: stock prices, currency exchange rates, waiting time in queue of $n$th arrival, workload at arrivals in time sharing computer systems.
\item Discrete random process:  counting processes, population sampled at birth-death instants, number of people in queues.
\item Continuous random process: water level in a dam, waiting time till service in a queue, location of a mobile node in a network.
\end{itemize}
\end{shaded*}

\subsection{Specification}
To define a measure on collection of random variables, we need to know it's joint distribution $F: \R^T \to [0,1]$. 
To this end, for any $x \in \R^T$we need to know
\begin{align*}
F(x) = P\left(\displaystyle {\bigcap_{t \in T}\{\omega \in \Omega: X_t(\omega) \leq x_t\}}\right).
\end{align*}
When the index set $T$ is infinite, any function of the above form would be zero if $x_t$ is finite for all $t \in T$. 
Therefore, we only look at the values of $F(x)$ when $x_t \in \R$ for indices $t$ in a finite set $S$ and $x_t = \infty$  for all $t \notin S$. 
%This leads to finite-dimensional distributions as defined below. 
%\begin{defn}  
We can define a \textbf{finite dimensional distribution} for any finite set $S \subseteq T$ and $x_S = \{x_s \in \R : s \in S\}$, 
\begin{align*}
F_S(x) = P\left(\displaystyle {\bigcap_{s \in S}\{\omega \in \Omega: X_s(\omega) \leq x_s\}}\right).
\end{align*}
%\end{defn}
Set of all finite dimensional distributions of the stochastic process $\{X_t: t \in T\}$ characterizes its distribution completely.
Simpler characterizations of a stochastic process $X(t)$ are in terms of its moments. 
That is, the first moment such as mean, and the second moment such as correlations and covariance functions. 
\begin{xalignat*}{5}
&m_X(t) \triangleq \E X_t, && R_X(t,s) \triangleq \E X_tX_s,&& C_X(t,s) \triangleq \E (X_t - m_X(t))(X_s-m_X(s)).
\end{xalignat*}
\begin{shaded*}
Some examples of simple stochastic processes. 
\begin{enumerate}[i\_]
\item $X_t = A \cos 2\pi t$, where $A$ is random. 
The finite dimensional distribution is given by 
\begin{align*}
F_S(x) &= P\left(\left\{A\cos 2\pi s \leq x_s, s \in S\right\}\right). %P\left(\left\{A \leq \min_{s \in S\setminus\{(2k+1)\frac{\pi}{2}, k \in \Z\}}\frac{x_s}{\cos 2\pi_s}\right\}\right).
\end{align*}
The moments are given by 
\begin{xalignat*}{5}
&m_X(t) = (\E A)\cos 2\pi t, && R_X(t,s) =  (\E A^2) \cos 2\pi t\cos 2\pi s,&& C_X(t,s) =\text{Var}(A) \cos 2\pi t\cos 2\pi s.
\end{xalignat*}
\item $X_t = \cos(2\pi t+ \Theta)$, where $\Theta$ is random and uniformly distributed between $(-\pi, \pi]$. 
The finite dimensional distribution is given by 
\begin{align*}
F_S(x) &= P\left(\left\{\cos(2\pi s + \Theta) \leq x_s, s \in S\right\}\right). %P\left(\left\{A \leq \min_{s \in S\setminus\{(2k+1)\frac{\pi}{2}, k \in \Z\}}\frac{x_s}{\cos 2\pi_s}\right\}\right).
\end{align*}
The moments are given by 
\begin{xalignat*}{5}
&m_X = 0, && R_X(t,s) =  \frac{1}{2}\cos2\pi (t-s),&& C_X(t,s) = R_X(t,s).
\end{xalignat*}
\item $X_n = U^n$ for $n \in \N$, where $U$ is uniformly distributed in the open interval $(0,1)$.
\item $Z_t = At +B$ where $A$ and $B$ are independent random variables. 
\end{enumerate}
\end{shaded*}

\subsection{Independence}
Recall, given the probability space $(\Omega, \F, P)$, two events $A, B \in \F$ are \textbf{independent events} if 
\begin{align*}
P(A\cap B) = P(A)P(B).
\end{align*}
Random variables $X,Y$ defined on the above probability space, are \textbf{independent random variables} if for all $x,y \in \R$
\begin{align*}
P\{X(\omega) \leq x, Y(\omega) \leq y\} = P\{X(\omega) \leq x\}P\{Y(\omega) \leq y\}.
\end{align*}
Two stochastic process $X_t, Y_t$ for common index set $T$ are \textbf{independent stochastic processes} if for all %$m,n \in \N$ and 
finite subsets $I, J \subseteq T$ %such that $|I| =m, |J| = n$, 
\begin{align*}
P\left(\{X_i \leq x_i, i \in I \}\cap\{Y_j \leq y_j, j \in J\}\right) = P\left(\{X_i \leq x_i, i \in I\}\right)P\left(\{Y_j \leq y_j, j \in J\}\right).
\end{align*}

\subsection{Examples of Tractable Stochastic Processes}
In general, it is very difficult to characterize a stochastic process completely in terms of its finite dimensional distribution. 
However, we have listed few analytically tractable examples below, where we can completely characterize the stochastic process. 
\subsubsection{Independent and identically distributed processes}
Let $\{X_t: t \in T\}$ be an independent and identically distributed (\emph{iid}) random process, with common distribution $F(x)$. 
Then, the finite dimensional distribution for this process for any finite $S \subseteq T$ can be written as 
\begin{align*}
F_S(x) &= P\left(\{X_s(\omega) \leq x_s, s \in S\}\right) = \prod_{s \in S}F(x_s).
\end{align*}
It's easy to verify that the first and the second moments are independent of time indices. 
Since $X_t = X_0$ in distribution, 
\begin{xalignat*}{5}
&m_X = \E X_0 , && R_X = \E X_0^2, && C_X = \text{Var}(X_0).
\end{xalignat*}

%\subsubsection{Independent increments process}
\subsubsection{Stationary Processes}
A stochastic process $X_t$ is \textbf{stationary} if all finite dimensional distributions are shift invariant, that is for finite $S \subseteq T$ and $t > 0$, we have 
\begin{align*}
F_S(x) &= P(\{X_s(\omega) \leq x_s, s \in S\}) = P(\{X_{s+t}(\omega) \leq x_s, s \in S\}) = F_{t+S}(x).
\end{align*}
In particular, all the moments are shift invariant. 
Since $X_t = X_0$ and $(X_t,X_s) = (X_{t-s}, X_{0})$ in distribution, we have 
\begin{xalignat*}{5}
&m_X = \E X_0 , && R_X(t-s,0) = \E X_{t-s}X_0, && C_X(t-s,0) = R_X(t-s,0) - m_X^2.
\end{xalignat*}

\subsubsection{Markov Processes}
A stochastic process $X_t$ is \textbf{Markov} if conditioned on the present state, future is independent of the past.  
That is, for any ordered index set $T$ containing any two indices $u > t$, we have  %such that $t < u$, % \in T$, 
\begin{align*}
P(\{X_u(\omega) \leq x_u| X_s, s \leq t\}) = P(\{X_{u}(\omega) \leq x_u| X_t\}).
\end{align*}
We will study this process in detail in coming lectures. 

\subsubsection{L\'evy Processes}
A stochastic process $X_t$ indexed by positive reals is \textbf{L\'evy} if the following conditions hold. 
\begin{enumerate}[i\_]
\item $X_{0}=0$, almost surely.
\item The increments are independent: For any $ 0\leq t_{1}<t_{2}<\cdots <t_{n}<\infty$, $X_{t_{2}}-X_{t_{1}},X_{t_{3}}-X_{t_{2}},\ldots ,X_{t_{n}}-X_{t_{n-1}}$ are independent.
\item The increments are stationary: For any $s<t$,  $X_{t}-X_{s}$, is equal in distribution to $X_{t-s}$.
\item Continuous in probability: For any $\epsilon > 0$ and $t\geq 0$ it holds that $\lim _{h\rightarrow 0}P(|X_{t+h}-X_{t}|>\epsilon )=0$. 
\end{enumerate}
\begin{shaded*}
Two examples of L\'evy processes are Poisson process and Wiener process. 
The distribution of Poisson process at time $t$ is Poisson with rate $\lambda t$ and the distribution of Wiener process at time $t$ is zero mean Gaussian with variance $t$. 
\end{shaded*}
\begin{thm} A L\'evy process has infinite divisibility. 
That is, for all $n \in \N$ %the following equality holds 
\begin{align*}
\E e^{\theta X_t} &= \left(\E e^{\theta X_{{t}/{n}}}\right)^n.
\end{align*}
Further, if the process has finite moments $\mu_n(t) = \E X_t^n$ then the following Binomial identity holds
\begin{align*}
\mu_n(t+s) &= \sum_{k=0}^n\binom{n}{k}\mu_k(t)\mu_{n-k}(s).
\end{align*}
\end{thm}
\begin{proof}
The first equality follows from the independent and stationary increment property of the process, 
and the fact that we can write 
\begin{align*}
X_t &= \sum_{k=1}^nX_{\frac{kt}{n}}-X_{\frac{(k-1)t}{n}}.
\end{align*}
Second property also follows from the the independent and stationary increment property of the process, 
and the fact that we can write 
\begin{align*}
X_{t+s}^n &= (X_{t} + X_{t+s}-X_{t})^n = \sum_{k=0}^n\binom{n}{k}X_t^k(X_{t+s}-X_t)^{n-k}.
\end{align*}

\end{proof}
%In particular, all the moments are shift invariant. 
%Since $X_t = X_0$ and $(X_t,X_s) = (X_{t-s}, X_{0})$ in distribution, we have 
%\begin{xalignat*}{5}
%&m_X = \E X_0 , && R_X(t-s,0) = \E X_{t-s}X_0, && C_X(t-s,0) = R_X(t-s,0) - m_X^2.
%\end{xalignat*}
\end{document}
\section{Examples of Stochastic Processes}
\begin{exmp}[Queues] Queues are complex stochastic processes and consist of two stochastic processes arrival and service, coupled through a buffer. 
Number of arrivals and arrival instants could be discrete or continuous random variable. 
For a discrete arrival case, arrival process can be characterized by the time epochs of discrete arrivals, denoted $\{A_n: n \in \N \}$. 
Similarly, service requirement of each incoming arrival can also be a discrete or continuous random variable.
Service of each discrete arrival can be considered to be a random amount of time, $\{S_n: n \in \N\}$. 
Queue can have a finite or infinite waiting area, and can be served by single or multiple servers. 
Important performance metrics for queues are mean waiting time of arrivals and mean queue length. 
These metrics are affected by the service policy that determines how to serve incoming arrivals. 
Few important service policies are first come first out, last in first out, processor sharing etc.
%If a queue can only be served at a fixed rate, then arrivals would have to wait in the buffer till their service time. 
Queues have applications in operations research, industrial engineering, telecommunications networks, among others. 
\end{exmp}
\begin{exmp}[Gambler's ruin] One can model many gambling games with random walks, where wins or losses on each bet can be thought of as a random step. If a gambler starts with certain capital, and he wants to quit gambling after he makes a certain amount of money, one is interested in probability of a gambler getting bankrupt before it can quit gambling. These questions are related to hitting times of a random walk. Random walks have deep relations to Brownian motion.
\end{exmp}
\begin{exmp}[Urn Models] In these models, one is interested in ball and urns. One is interested in distribution of balls in urns, when one can randomly throw balls into urns. Balls can be of multiple colors and may or may not be replaced after putting into urns. These models have applications in influence maximization and epidemic control.
\end{exmp}
\begin{exmp}[Branching Processes] This is used by biologists to model population. In this model, one assumes that every individual in a population has a probability distribution over number of children it can have. Each children can be assumed to have identical and independent distribution for their progenies in next generation. These type of models can answer questions related to survival of species.
\end{exmp}
\begin{exmp}[Random Graphs] A typical graph $G$ consists of vertex set $V$ and edge set $E \subseteq V \times V$. Both of these can be random in general. In classical settings, usually $V = [n]$ and $E$ is selected randomly from set of all possible edges $[n] \times [n]$, without self-loops. These models are exploited in study of various type of networks, and can be used to answer questions regarding network properties.
\end{exmp}
\end{document}
\section{Poisson processes}
\begin{defn}[Point Process] A stochastic process $\{N(t), t\geqslant 0\}$ is a \textbf{point process} if
\begin{enumerate}
  \item $N(0) = 0$.
  \item $t\mapsto N(t) (\omega)$ is non-decreasing, integer valued, right continuous and at points of discontinuity (wherever it has jumps) $(N(t)- N(t^{-}))\leqslant 1, \hspace{0.1cm} \forall \quad \omega \in \Omega$. 
\end{enumerate}
\end{defn} 
\begin{defn}[Simple Point Process] A \textbf{simple point process} is a point process of jump size 1.
\end{defn}

\begin{defn}[Stationary Increment Point Process] A point process $\{N(t), t\geqslant 0\}$ is called \textbf{stationary increment point process }, if for any collection of $0\leqslant t_{1}<t_{2}...,<t_{n}$, we have $(N(t_{n})-N(t_{n-1}),N(t_{n-1})-N(t_{n-2}),...,N(t_{1}))$ having the same joint distribution as $(N(t_{n}+t)-N(t_{n-1}+t),...,N(t_{1}+t)), ~ \forall t \geqslant 0$.
\end{defn}
\begin{defn}[Stationary Independent Increment Point Process]   A point process $\{N(t), t\geqslant 0\}$ is called \textbf{stationary independent increment process}, if it has stationary increments and the increments are independent random variables.
\end{defn}

\begin{figure}[hhhh]
\center
	\begin{tikzpicture}
[node distance=1cm, draw=black, thick, >=stealth',
axes/.style=,
codeword/.style={rectangle, draw=black, inner sep=0pt, minimum height=0.8cm, minimum width=0.5cm}]

\begin{scope}[axes]
\draw[->] (0,0) -- (8,0) node[right] {$t$} coordinate (time);
\draw[->] (0,0) -- (0,5) node[above] {$N(t)$} coordinate (number of  events);
\foreach \x/\xtext in {1/S_1, 2/S_2, 5/S_3,7/S_4}
	\draw[xshift=\x cm] (0pt,1pt) -- (0pt,-1pt) node[below,fill=white] {$\xtext$};
\foreach \y/\ytext in {1,...,4}
	\draw[yshift=\y cm] (1pt,0pt) -- (-1pt,0pt) node[left,fill=white] {$\ytext$};
\end{scope}
\draw[] (0,0)--(1,0)--(1,1)--(2,1)--(2,2)--(5,2)--(5,3)--(7,3)--(7,4)--(8,4);
\end{tikzpicture}

	\caption{Sample path of a Poisson process.}
	\label{Fig:Poisson}
\end{figure}
\noindent The points of discontinuity corresponds to the arrival instants of the point process. Let $X_{n}$  denote the inter arrival time between $(n-1)^{th}$ and $n^{th}$ arrival. Further, let, $S_{0}=0,~ S_{n}= \sum^{n}_{k=1}X_{k}$. $S_{n}$ is the arrival instant of of $n^{th}$ point. The arrival at time zero is not counted.

\begin{defn}[Poisson Process]
A simple point process $\{N(t),~ t\geqslant0\} $ is called a \textbf{Poisson process} with rate $0< \lambda< \infty$, if inter-arrival times $\{X_{n},~n\geqslant 1\}$ are \emph{iid} $\exp(\lambda)$ random variables, i.e.
 \begin{equation*}
  P\{X_{1}\leqslant x\} = 
	\begin{cases}
		1-e^{-\lambda x}, & x\geqslant 0   \\
		0,  & \text{ else}.
	\end{cases}
\end{equation*}
\end{defn}

\textbf{Remarks:} Observe that 
\begin{align*}
\{S_{n}\leqslant t\} &=  \{N(t)\geqslant n \},\\
\{S_{n}\leqslant t, S_{n+1}> t \} &= \{N(t)= n\},\quad\mathrm{and} \\
P\{X_{n} = 0\} &= P\{X_n\leqslant 0\} = 0.
\end{align*}
Also, by Strong Law of Large Numbers (SLLN), 
\begin{equation*}
\lim_{n \to \infty} \frac{S_{n}}{n} = E[X_{1}] = \frac{1}{\lambda}\quad\mathrm{a.s.} 
\end{equation*}
Therefore, we have $S_n \rightarrow \infty$, a.s. This implies $P\{\omega: N(t)(\omega) < \infty\} =1$. To see this, let's pick an $\omega \in \Omega$ such that $ N(t)(\omega) = \infty$, then $S_{n}(\omega)\leqslant t,\quad \forall n$. This implies $S_{\infty}(\omega)\leqslant t$  and $\omega \not\in \{\omega: S_{n}(\omega) \rightarrow \infty \}.$ Hence, probability measure for such $omega$'s is zero and the claim follows. 


\subsection{Moment Generating Function and Density Function of $S_n$}
We know that time of $n^{\mathrm{th}}$ event $S_n$ is sum of $n$ consecutive \emph{iid} inter-arrival times $X_k$, i.e. $S_n = \sum^{n}_{k=1}X_{k}$. Therefore, moment generating function $\mathbb{E} [ e^{\alpha S_{n}} ]$ of $S_n$ is given by 
 \begin{equation*}
  \mathbb{E} [ e^{\alpha S_{n}} ] = \left(\mathbb{E}[e^{\alpha X_{1}}]\right)^{n}. 
 \end{equation*} 
Since each $X_k$ is \emph{iid} exponential with rate $\lambda$, it is easy to see that moment generating function of intr-arrival time $X_1$ is 
 \begin{equation*}
  \mathbb{E} [ e^{\alpha X_1} ] = 
		\begin{cases}
		\frac{\lambda}{\lambda-\alpha}, & \alpha < \lambda \\
		\infty, & \alpha \geqslant \lambda.
		\end{cases} 
 \end{equation*} 
   %\begin{eqnarray*}
%\mathbb{ E}[e^{\alpha X_{1}}]  &=& \int^{\infty}_{0}\lambda e^{-\lambda t} e^{\alpha t} dt \\
  %&=& \lambda \int^{\infty}_{0} e ^{-(\lambda-\alpha)t} dt\\
  %\end{eqnarray*}
   %If $\alpha \geqslant \lambda,~\mathbb{E} [e^{\alpha X_{1}}] = \infty$. Else, 
  %\begin{eqnarray*}
  %\mathbb{E} [e^{\alpha X_{1}}]&=&\lambda \left[ \frac{e^{-(\lambda-\alpha)t}}{-(\lambda-\alpha)}\right]^{\infty}_{0} \\
   %&=& \lambda\left[0-(-\frac{1}{\lambda-\alpha})\right] = \frac{\lambda}{\lambda-\alpha}. 
   %\end{eqnarray*}
Substituting the moment generating function of inter-arrival time $X_1$ in moment generating function of $n^{\mathrm{th}}$ event time $S_n$, we obtain
\begin{equation*}
   \mathbb{E}[e^{\alpha S_{n}}] = 
	\begin{cases} 
	\left(\frac{\lambda}{\lambda-\alpha}\right)^{n}, & \alpha < \lambda, \\
   \infty, &\text{else}.
	\end{cases}
\end{equation*}

\begin{thm}[Arrival Time] Density function of $S_n$ is Gamma distributed with parameters $n$ and $\lambda$. That is,
\begin{equation*}
f_{S_n}(s) =\frac{\lambda (\lambda s)^{n-1}} {(n-1)!} e^{-\lambda s}.
\end{equation*}
\end{thm}
\begin{proof} Notice that $X_i$'s are \emph{iid} and $S_1 = X_1$. In addition, we know that $S_n = X_n + S_{n-1}$. Since, $X_n$ is independent of $S_{n-1}$, we know that distribution of $S_n$ would be convolution of distribution of $S_{n-1}$ and $X_1$. Since $X_n$ and $S_1$ have identical distribution, we have $f_{S_{n}}=f_{S_{n-1}}*f_{S_1}$. The result follows from straightforward induction.
\end{proof}

Process $N(t)$ is of real interest, and we can compute the distribution of $N(t)$ for each $t$ from the distribution of $S_n$ in the following.
%\begin{figure}[hhhh]
%\center
	%\begin{tikzpicture}
[node distance=1cm, draw=black, thick, >=stealth',
axes/.style=,
codeword/.style={rectangle, draw=black, inner sep=0pt, minimum height=0.8cm, minimum width=0.5cm}]

\begin{scope}[axes]
\draw[->] (0,0) -- (8,0) node[right] {$t$} coordinate (time);
\draw[->] (0,0) -- (0,5) node[above] {$N(t)$} coordinate (number of  events);
\foreach \x/\xtext in {1/S_1, 2/S_2, 5/S_3,7/S_4}
	\draw[xshift=\x cm] (0pt,1pt) -- (0pt,-1pt) node[below,fill=white] {$\xtext$};
\foreach \y/\ytext in {1,...,4}
	\draw[yshift=\y cm] (1pt,0pt) -- (-1pt,0pt) node[left,fill=white] {$\ytext$};
\end{scope}
\draw[] (0,0)--(1,0)--(1,1)--(2,1)--(2,2)--(5,2)--(5,3)--(7,3)--(7,4)--(8,4);
\end{tikzpicture}

 %% \caption{}\label{}
%\end{figure}
\begin{thm}[Poisson process] Process $N(t)$ is Poisson distributed with parameter $\lambda$ for each $t$. That is,
	\begin{equation*}
	P\{N(t)=n)\}= e^{-\lambda t}\frac{(\lambda t)^{n}}{n!}.
	\end{equation*}
\end{thm}
\begin{proof}

\begin{eqnarray*}
   P\{N(t) =n\}&=&  P\{S_{n}\leqslant t, S_{n+1} >t)\\
   &=&  \int^{t}_{0} P\left\{ {S_{n+1}>t}|{S_{n}=s}\right\}f_{S_n}(s)  ds\\
   &\stackrel{(a)}{=}& \int^{t}_{0} P\{X_{n+1}>t-s\} f_{S_n}(s) ds\\
   &=&  \int^{t}_{0}e^{-\lambda(t- s)} \frac{\lambda^{n}s^{n-1}}{(n-1)!}e^{-\lambda s}  ds\\
   &=&\frac{e^{-\lambda t} (\lambda t)^{n}}{n !}.
\end{eqnarray*}
 where (a) follows from the memoryless property of exponential distribution. %($P\{X_{n+1}>s+t|X_{k+1}>t)=P\{X_{k+1}>s)$).\\
\end{proof}

\textbf{Remark:} The Poisson process is not a stationary process. That is, the finite dimensional distributions (fdd) are not shift invariant. In the following section, we show that the Poisson process is a \textit{stationary,  independent increment} process. To this end, we will use an important property of exponential distribution- namely memoryless property. Memoryless property of exponential distribution will facilitate the computation of fdd of the Poisson process via one dimensional marginal distribution. 

\begin{prop}[Memoryless Distribution] Exponential distribution with  continuous support is the only distribution satisfying memoryless property.
\end{prop}

%\begin{eqnarray*}    
%            P\{N(t1}& = &n_{1}),..., N(tk}=n_{k}\hspace{1.0cm} fdd
%             \end{eqnarray*}
%
%We will S.t Poisson process has another very important property- Stationary indep. increment process\\
%
%$ 0 \leqslant t_{1} \leqslant  t_{2} \hspace{0.5cm} P[N(t1}=n_{1}, N(t2}=n_{2}]=0$ \hspace{1.0cm} if $n_{2}< n_{1}$\\
%$n_{2}\geqslant n_{1} \hspace{1.5cm} P[N(t1}=n, N(t2}-N(t1}=n_{2}-n_{1}]$
%
%
%If $N(t)$ is independent increment process.
%\begin{eqnarray*}
%&=&P [N(t1}=n_{1}]\\
%P (N(t2}-N(t1} &=& n_{2}-n_{1}) n2-n1\\
% P (N(t2}-N(t1} &=&\frac{[\lambda (t2-t1)]}{(n2-n1)!} e^{-\lambda (t2-t1)!}\\
%\end{eqnarray*}
%Need\\
%\begin{eqnarray*}
% N(t1}&=& n1\overrightarrow(n2-n1) n2\\
% arrivals &=& \phi \frac{(\lambda t1)^{n1}}{(n1)!}e^{-\lambda t1}\\
%\end{eqnarray*}
%What it means all fdd can be computed from $1-d$ marginal in a stationary indep. incr. property.\\
%Its not stationary $ \rightarrow N(t1}$ and $N(t2}$ not same distr.\\
%$N(t2}$ in some sense $> N(t1}$ so not same distr. intution\\
%To s.t ${N(t), t\geqslant 0}$ stationery ind. increment process.\\
%First we show indep. increment.\\
%We need an important property of exponential property of memorylessness.\\
%$X\sim exp (\lambda$\\
%$S>0, t> 0$\\
%\begin{eqnarray*}
%% \nonumber to remove numbering (before each equation)
%  P\{X>S +t / X>t) &=& P\{X>S) \\
%  In \hspace{0.2cm}reliability\hspace{0.2cm} theory & x& life time\\
%  P\{X>t+S/ X>t) &\leqslant& P\{X>S)
%\end{eqnarray*}
%Class of distr. satisfying ``New better than used distr''\\
%\begin{eqnarray*}
%X>t+s>S\\
%P\{X>s+t)/ X>t)\\
%&=& \frac{P\{X>S+t, X>t)}{P\{X>t)}\\
%&=& \frac{P\{X>S+t)}{P\{X>t)}\\
%&=& e^{-\lambda S}\\
%&=& P\{X>S)
%\end{eqnarray*}
\begin{proof}
Let $X$ be a random variable with a distribution function $F$ with memoryless property defined on $\mathbb{R}^{+}$. Let $g(t) \triangleq P\{X > t\} = 1 - F(t)$. Due to memoryless property of $F$, we notice that
\begin{eqnarray*}
  P\{X>s\} &=& P\{ X > t+s| X>t\} \\
  P\{X>s\} &=& \frac{ P\{ X>t+s, X>t\}}{P\{X>t\}}.
\end{eqnarray*}
Since $\{X > t + s\} =\{ X>t+s, X>t\}$, we have $g(t+s) = g(t)g(s)$ and hence $g(0) = g^2(0)$. Therefore, $g(0)$ is either unity or zero. Note, that $g$ is a right continuous (RC) function and is non-negative. 

We will show that $g$ is an exponential function. That is, $g(t) = e^{\alpha t}$ for some $\alpha \geqslant 0$. We will prove this in stages. First, we show this is true for $t \in \mathbb{Z}^+$. Notice that we can obtain via induction
\begin{eqnarray*}
	g(2) &=& g(1) g(1) = g^{2}(1), \mathrm{ and }\\
	g(m) &=& [g(1)]^{m}.
\end{eqnarray*}
Since $g(1)$ is non negative, there exists a $\beta$ such that $g(1)=e^{\beta}$ and $g(m)= e^{m \beta}, m \in \mathbb{Z}_{+}$. Next we show that for any $n \in \mathbb{Z}_{+}$,        
\begin{equation*}
	g(1) =  g\left(\frac{1}{n}+..., +\frac{1}{n}\right) = \left[g\left(\frac{1}{n}\right)\right]^{n}.
\end{equation*}
Therefore, for same $\beta$ we used for $g(1)$, we have $g\left(\frac{1}{n}\right) = e^{\frac{\beta}{n}}$. Now, we show that $g$ is exponential for any $t \in \mathbb{Q}^+$. To this end, we see that for any $m, n \in \mathbb{Z}_{+}$, we have 
\begin{equation*}
	g\left(\frac{m}{n}\right) = \left[g\left(\frac{1}{n}\right)\right]^{m}= e^{\frac{m \beta}{n}}.
\end{equation*}
Now, we can show that $g$ is exponential for any real positive $t$ by taking a sequence of rational numbers $\{t_n\}$ decreasing to t. From right continuity of $g$, we obtain 
\begin{equation*}
	g(t) \stackrel{(a)}{=} \lim_{n\rightarrow \infty} g(t_n) =   \lim_{n\rightarrow \infty} e^{\beta t_{n}}= e^{\beta t}.
\end{equation*}
Since $P\{X > x\}$  is decreasing  with $x$, $\beta $ is negative.  
\end{proof}
\begin{figure}[hhhh]
\center
	\begin{tikzpicture}
[node distance=1cm, draw=black, thick, >=stealth',
axes/.style=,
codeword/.style={rectangle, draw=black, inner sep=0pt, minimum height=0.8cm, minimum width=0.5cm}]

\begin{scope}[axes]
\draw[->] (0,0) -- (9,0) node[right] {$t$} coordinate (time);
\draw[->] (0,0) -- (0,8) node[above] {$N(t)$} coordinate (number of  events);
\foreach \x/\xtext in {0.5/S_{n-3}, 1.5/S_{n-2}, 2.2/S_{n-1},3.2/S_{n}, 4/t_1, 5.1/S_{n+1}, 6/S_{n+2}, 7/t}
	\draw[xshift=\x cm] (0pt,1pt) -- (0pt,-1pt) node[below,fill=white] {$\xtext$};
\foreach \y/\ytext in {1/n-3,2/n-2, 3/n-1, 4/n,5/n+1,6/n+2,7/n+3}
	\draw[yshift=\y cm] (1pt,0pt) -- (-1pt,0pt) node[left,fill=white] {$\ytext$};
\end{scope}
\draw[] (0,0)--(.5,0)--(.5,1)--(1.5,1)--(1.5,2)--(2.2,2)--(2.2,3)--(3.2,3)--(3.2,4)--(5.1,4)--(5.1,5)--(6,5)--(6,6)--(7.5,6)--(7.5,7)--(9,7);
\draw[<->] (4,0) -- node[midway,left]{$N({t_1})$}(4,4);
\draw[<->] (7,0) -- node[midway,left]{$N(t)$}(7,6);
\draw[<->] (3.2,4.1)--node[near start, above]{$X'_{n+1}$}(4,4.1);
\draw[<->] (4,4.1) --node[midway,above]{$X''_{n+1}$}(5.1,4.1);
\end{tikzpicture}

  \caption{Stationary independent increment property of Poisson process.}
	\label{Fig:IndependentIncrements}
\end{figure}
\begin{prop}[Stationary Independent Increment Property] Poisson process ${N(t), t\geqslant 0}$ has stationary independent increment property.
\end{prop}
\begin{proof}
To show that $N(t)$ has stationary independent increment property, it suffices to show that $N_t-N(t_{1}) \perp N(t_1)$ and $N(t) - N(t_1) \sim N(t-t_1)$. Since, we can use induction to show this stationary independence increment property for for any finite disjoint time-intervals. The memoryless property of exponential distribution is crucially used. And, we see that independent increment holds only if inter-arrival time is exponential. We can see in Figure~\ref{Fig:IndependentIncrements} that $t_1$ divides $X_{n+1}$ in two parts such that, $X_{n+1} = X_{n+1}^{'} + X_{n+1}^{''}$. Here,  $X_{n+1}^{''}$ is independent of $X_{n+1}^{'}$ and has same distribution as $X_{n+1}$. Therefore, 
\begin{align*}
\{ N(t_1) = n \} &\iff  \{ S_n = t_1 + X'_{n+1} \}, \\
\{ N(t) - N(t_1) \geqslant m \} &\iff \{ X''_{n+1} + \sum_{i=n+2}^{n+m} X_i \leqslant t - t_1 \}.
\end{align*}
Since, $\{X_i: i \geqslant n+2\}\cup\{X_{n+1}^{''}\}$ are independent of $\{X_i: i \leqslant n\}\cup{X_{n+1}^{'}}$, we have $N(t)-N(t_{1}) \perp N(t_1)$. Further, since $X_{n+1}^{''}$ has same distribution as $X_{n+1}$, we get $N(t) - N(t_1) \sim N(t-t_1)$. By induction we can extend this result to $(N(t_{n})-N(t_{n-1}),...,N(t_{1}))$. 
\end{proof}
\end{document}